\documentclass{report}
\title{Bachelorproef: richtlijnen}
\author{Dylan Cluyse}
\documentclass[a4paper,12pt,twoside]{report}

\usepackage[margin=1in]{geometry}
\usepackage{amsmath}
\usepackage{titling}
\usepackage{fontspec}
\usepackage{listings}
\usepackage{hyperref}
\usepackage{graphicx}
\newfontfamily\headingfont[]{Montserrat-Black}
\usepackage[dutch]{babel}
\usepackage{fontspec}
\setmainfont{Montserrat}

\begin{document}
	\chapter{Algemene richtlijnen}
	Voor de vereenvoudigde versie van het artikel moet er geen rekening worden gehouden met het digitale formaat, marges, lettertypes -of ruimte. Een Word-document of PDF-document is voldoende. Daarnaast moet er ook geen rekening worden gehouden met de afbeeldingen in het artikel. Deze digitale aspecten vallen buiten de scope van het logopedische -en onderwijsaspect. 
	
	\medspace
	
	De literatuurstudie wees volgende aanpassingen en richtlijnen uit, die scholieren met dyslexie kunnen ondersteunen. In de vereenvoudiging moeten de volgende elementen terug te vinden zijn:
	
	\section{Lexicale vereenvoudiging}
	
	Het doelpubliek is een derde graad middelbaar onderwijs. Kennis die in de eerste en tweede graad gekend zijn, hoeft niet aangepast te worden en wordt vernomen als 'gekend'.
	
	\begin{itemize}
		\item De tekst bevat geen andere synoniemen buiten het synoniem dat een moeilijk woord vervangt. Na het vervangen van een moeilijk woord met een synoniem, wordt er niet met andere synoniemen gewerkt. De woordfrequentie moet zo hoog mogelijk zijn.
		\item Indien een woord geen eenvoudiger synoniem heeft, mag het woord identiek blijven. Tussen haakjes wordt de betekenis van het desbetreffende woord meegegeven. 
		
	\end{itemize}
	
	\section{Syntactische vereenvoudiging}
	\begin{itemize}
		\item Zinnen zijn niet langer dan zeven woorden. Te lange zinnen worden opgebroken.
		\item Verwijswoorden zoals 'zij', 'hun' of 'hij' worden verandert naar namen, zoals voornamen of de namen van de entiteit(en). 
		\item Tangconstructies worden aangepast. Dit kan door de bijzin naar het begin of het einde van een zin te plaatsen, de zin splitsen in twee kortere zinnen of door het onderwerp en de persoonsvorm dichter bij elkaar plaatsen door minder informatie tussenin te plaatsen.
		\item Samengestelde werkwoorden worden uitgesloten.
		\item Voorzetseluitdrukkingen worden vermeden.
	\end{itemize}

	\section{Samenvatten}
	
\begin{itemize}
	\item De bronvermelding per samengevatte tekst moet overgenomen worden. De referentie gebeurt zoals die uit het oorspronkelijke document (Vancouver).
	\item De limiet op het vereenvoudigde artikel is 500 woorden. 
	\item Titels en subtitels worden overgenomen.
	\item Per paragraaf zijn er hoogstens zes zinnen.
	\item Zinnen, indien ze voldoen aan de criteria en kort en bondig genoeg zijn, mogen geciteerd worden in de vereenvoudigde tekst.
	\item Indien meerdere punten of pijlers in een tekst worden aangehaald, mogen deze punten als opsomming worden geschreven.
\end{itemize}


	\chapter{Specifieke richtlijnen voor het gegeven artikel}
	Na de tekstvereenvoudiging moeten de volgende vragen in hoogstens twee paragrafen beantwoord kunnen worden:
	\begin{itemize}
		\item \textbf{Inleiding}: Wat is het doel van dit onderzoek? Uit welk eerder onderzoek of uit welke probleemstelling vloeide dit onderzoek voort?
		\item \textbf{Socio-technische ontwikkeling}: Welke drie technische ontwikkelingen worden aangehaald in het onderzoek? Wat zijn de sociotechnische ontwikkelingen die het traditionele controle- en handhavingskader onder druk zetten als gevolg van de opkomst van algoritmische surveillance in het politiewerk?
		\item \textbf{Juridisch kader}: Wat zijn de tekortkomingen van het huidige juridisch kader en de controle-instrumenten die momenteel worden ingezet voor de verwerking van gegevens door middel van AI, en biedt het recente voorstel van de EU voor een AI-wet voldoende bescherming van grondrechten en handhavingsmechanismen?
		\item \textbf{Herdenken van algoritmische surveillance-controle}: Hoe kan de visie van Ubuntufilosofie en relationele ethiek bijdragen aan een herziening? Hoe kan relationele controle helpen bij het beschermen van kwetsbare groepen tegen schendingen van mensenrechten door algoritmische surveillance?
		\item \textbf{Concrete stappen}: Welke concrete stappen omtrent ethiek worden er aangehaald? Hoe kan relationele controle helpen bij het herdenken van controlemechanismen en rekening houden met sociaal-technische ontwikkelingen zoals asymmetrische machtsrelaties en de toenemende macht van technologiebedrijven?
		\item \textbf{Conclusies}: Wat besluiten de onderzoekers? Indien verder onderzoek vereist is, naar welk onderzoek kijken ze specifiek uit?
	\end{itemize} 
\end{document}