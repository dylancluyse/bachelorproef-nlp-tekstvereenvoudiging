\documentclass{report}
\title{Bachelorproef: richtlijnen rond AI-tekstvereenvoudiging}
\author{Dylan Cluyse}
\documentclass[a4paper,12pt,twoside]{report}

\usepackage[margin=1in]{geometry}
\usepackage{amsmath}
\usepackage{titling}
\usepackage{fontspec}
\usepackage{listings}
\usepackage{hyperref}
\usepackage{graphicx}
\newfontfamily\headingfont[]{Montserrat-Black}
\usepackage[dutch]{babel}
\usepackage{fontspec}
\setmainfont{Montserrat}

\begin{document}
	\chapter{Algemene richtlijnen}
	
	Het onderzoek achterhaalt hoe scholieren met dyslexie in de derde graad middelbaar onderwijs ondersteund kunnen worden bij het intensief lezen van een wetenschappelijk artikel. De ondersteuning wordt aangeboden in de vorm van tekstvereenvoudiging met AI. Tekstvereenvoudiging omvat het lexicaal en syntactisch vereenvoudigen, alsook het samenvatten van de kerngedachte per hoofdstuk. Om tekstvereenvoudiging met AI te testen, moeten handmatig en automatisch vereenvoudigde teksten met elkaar worden vergeleken. 
	
	\medspace
	
	De opdracht voor deze bijdrage is het manueel vereenvoudigen van een gekregen wetenschappelijk artikel. Dit wetenschappelijk artikel is zes pagina's (voorpagina uitgesloten) lang. Het doelpubliek voor dit vereenvoudigd artikel zijn scholieren met dyslexie in de derde graad ASO/TSO middelbaar onderwijs. Concreet zou dit een artikel moeten zijn dat tijdens een STEM-les wordt gegeven aan deze doelgroep. Op pagina 2 vindt u tekstvereenvoudigingstechnieken terug. Deze aanpassingen hebben een beneficieel effect op scholieren met dyslexie een wetenschappelijk artikel bij het intensief lezen van wetenschappelijke teksten. U dient deze gekregen aanpassingen te volgen voor deze bijdrage. De beschreven technieken en elementen dienen in de manuele vereenvoudigde tekst terug te vinden zijn. 
	
	\medspace
	
	Op basis van de richtlijnen op pagina 2 worden dezelfde instructies aan een AI-model gegeven. Met de richtlijnen en de door u handmatig vereenvoudigde tekst kan het onderzoek evalueren of AI-taalmodellen capabel zijn om manuele tekstvereenvoudigingstechnieken, specifiek voor scholieren met dyslexie, toe te passen op wetenschappelijke artikelen. De tekst dat een AI-model vereenvoudigd wordt afgetoetst op basis van bestaande metrieken en de kenmerken van uw vereenvoudigde versie. % Woordenschat die in de eerste en tweede graad gekend zijn, hoeven niet aangepast te worden en wordt vernomen als 'gekend'. 
	
	\medspace
	
	Voor de vereenvoudigde versie van het artikel moet u als taaldocent of auteur geen rekening houden met marges, lettertypes of spatiëring. Deze aanpassingen mogen, maar enkel de tekstuele inhoud van het gekregen document wordt in het experiment opgenomen. Een Word-document of PDF-document is voldoende. Daarnaast moet er ook geen rekening worden gehouden met de afbeeldingen in het artikel.  
	
	\medspace
	
	Aanpassingen die niet op pagina 2 omschreven zijn om de tekst eenvoudiger te maken, zijn vrijblijvend. Indien deze aanpassing volgens u een meerwaarde biedt, dan moet de werkwijze voor de start van het document kort beschreven worden. De aanpassing moet eenmalig bovenaan het document worden vermeld. Bijvoorbeeld: 'De zin werd gesplitst omdat deze langer is dan tien woorden.' Zo kunnen wij bij het onderzoek rekening houden met deze extra handeling. Het AI-model wordt dan met deze extra parameter in het achterhoofd beoordeeld.
	
	\medspace
	
	Namens mijn promotor, copromotors en mezelf wil ik u hartelijk bedanken voor uw interesse in dit onderzoek.
	
	\newpage
	
	\section{Lexicale vereenvoudiging}
	
	\begin{itemize}
		\item Een moeilijk woord achterhalen gebeurt op basis van intuïtie en inschatting van de doelgroep. De woordenschat die zelden voorkomt in de dagelijkse lees- en schrijftaal van STEM-vakken voor scholieren tussen 16 en 18 jaar oud, moet worden aangepast. Vakjargon die al in de tweede graad ASO en TSO aan bod is gekomen, mag behouden blijven.
		\item Een woord dat langer is dan achttien letters, wordt als moeilijk beschouwd en moet vervangen worden door een korter (en eenvoudiger) alternatief.
		\item Acroniemen worden voluit geschreven.
		\item Vervang een moeilijk woord in het artikel door slechts één synoniem. Bijvoorbeeld, als het woord 'adhesief' wordt vervangen door 'klevend', gebruik dan geen andere synoniemen voor 'klevend' in de rest van het artikel. 
		\item Indien een woord geen eenvoudiger synoniem heeft, mag het woord kort worden uitgelegd. Dit kan tussen ronde haakjes, of in een aparte zin. Bijvoorbeeld: "Ik voelde me melancholisch." wordt aangepast naar "Ik had een diep gevoel van droefheid en verlies.".		
		\item Vermijd het directe overnemen van percentages indien deze voorkomen in het artikel. Vervang dit door benamingen zoals 'een kwart', 'de helft'. 
	\end{itemize}
	
	\section{Syntactische vereenvoudiging}
	
	\begin{itemize}
		\item Te lange zinnen worden opgebroken of gesplitst. De zinnen in het vereenvoudigde artikel zijn hoogstens tien woorden lang.
		\item Verwijswoorden zoals 'zij', 'hun' of 'hij' worden naar namen veranderd. Bijvoorbeeld voornamen of entiteitsnamen (bijvoorbeeld Nationale Bank). 
		\item Tangconstructies worden vervangen. Dit kan door de bijzin naar het begin of het einde van een zin te plaatsen, de zin te splitsen in twee kortere zinnen of door het onderwerp en de persoonsvorm dichter bij elkaar te plaatsen door minder informatie tussenin te plaatsen.
		\item Voorzetseluitdrukkingen en samengestelde werkwoorden worden vervangen indien mogelijk. Indien er geen eenvoudigere alternatieven zijn, mogen deze onaangepast blijven.
	\end{itemize}
	
	\section{Samenvatten}
	
	\begin{itemize}
		\item Het vereenvoudigde artikel volgt dezelfde structuur en chronologische volgorde zoals dat van het oorspronkelijk artikel. Iedere hoofdstuk in het wetenschappelijk artikel is hoogstens twee paragrafen lang. Per paragraaf zijn er hoogstens vijf zinnen.
		\item Het vereenvoudigd artikel is hoogstens 500 woorden lang. 
		\item Citeren mag indien deze zinnen aan de bovenstaande criteria (lexicale en syntactische vereenvoudiging) voldoen.
		\item Het gebruik van opsommingen of \textit{bullet-points} wordt aangemoedigd.
		\item De bronvermelding wordt overgenomen. De referentie gebeurt zoals die uit het oorspronkelijke document (Vancouver) en mag direct overgenomen worden: '[4]' blijft '[4]'.
	\end{itemize}
	
	
	\chapter{Specifieke richtlijnen voor het gegeven artikel}
De kerngedachte van iedere paragraaf moet terug te vinden zijn in de vereenvoudigde tekst. Na de tekstvereenvoudiging moeten de volgende vragen in hoogstens twee paragrafen beantwoord kunnen worden:
	
	\begin{itemize}
		\item \textbf{Inleiding}: Wat is het doel van dit onderzoek? Uit welk eerder onderzoek of uit welke probleemstelling vloeide dit onderzoek voort?
		\item \textbf{Socio-technische ontwikkeling}: Welke drie technische ontwikkelingen worden aangehaald in het onderzoek? Wat zijn de sociotechnische ontwikkelingen die het traditionele controle- en handhavingskader onder druk zetten als gevolg van de opkomst van algoritmische surveillance in het politiewerk?
		\item \textbf{Juridisch kader}: Wat zijn de tekortkomingen van het huidige juridisch kader en de controle-instrumenten die momenteel worden ingezet voor de verwerking van gegevens door middel van AI, en biedt het recente voorstel van de EU voor een AI-wet voldoende bescherming van grondrechten en handhavingsmechanismen?
		\item \textbf{Herdenken van algoritmische surveillance-controle}: Hoe kan de visie van Ubuntufilosofie en relationele ethiek bijdragen aan een herziening? Hoe kan relationele controle helpen bij het beschermen van kwetsbare groepen tegen schendingen van mensenrechten door algoritmische surveillance?
		\item \textbf{Concrete stappen}: Welke concrete stappen omtrent ethiek worden er aangehaald? Hoe kan relationele controle helpen bij het herdenken van controlemechanismen en rekening houden met sociaal-technische ontwikkelingen zoals asymmetrische machtsrelaties en de toenemende macht van technologiebedrijven?
		\item \textbf{Conclusies}: Wat besluiten de onderzoekers? Indien verder onderzoek vereist is, naar welk onderzoek kijken ze specifiek uit?
	\end{itemize} 
	
\end{document}