%%=============================================================================
%% Discussie
%%=============================================================================

\chapter{\IfLanguageName{dutch}{Discussie}{Discussie}}%
\label{ch:discussie}

Dit onderzoek gebruikt drie onderzoeksmethoden om te bepalen hoe ontwikkelaars een optimale vorm van gepersonaliseerde ATS kunnen bieden aan scholieren met dyslexie in de derde graad van het middelbaar onderwijs.

\medspace

Uit de resultaten van de requirementsanalyse blijkt dat zowel erkende toepassingen als online tools onvoldoende functionaliteiten bieden voor gepersonaliseerde ATS en te weinig opties bieden voor personalisering van opmaak. Dit resultaat komt overeen met de verwachting dat bestaande tools niet specifiek gericht zijn op gepersonaliseerde ATS voor scholieren met dyslexie in de derde graad van het middelbaar onderwijs. Mogelijke verklaringen hiervoor zijn de populariteit van samenvattingstools in vergelijking met vereenvoudigingstools, de complexiteit die gepaard gaat met de ontwikkeling van dergelijke toepassing en het gebrek aan initiatief binnen dit vakgebied.

Hoewel ChatGPT en Bing Chatbot functionaliteiten bieden voor gepersonaliseerde ATS, ontbreken eenduidige handelingen waardoor gebruikers moeite kunnen hebben met het vereenvoudigen van wetenschappelijke artikelen. Daarnaast houdt het model van ChatGPT geen rekening met referenties buiten de getrainde data, wat problemen kan veroorzaken op het gebied van data-integriteit. Hiertegenover staat Bing AI, dat wel rekening houdt met externe referenties en daarmee een goede basis vormt voor ontwikkelaars om meer referentiemateriaal aan te bieden in ondersteunende onderwijstoepassingen. Verder onderzoek naar de toepassing van deze AI via een mogelijke API is absoluut noodzakelijk en kan baanbrekend zijn voor de onderwijssector. Aan de andere kant is er de mogelijkheid om bestaande toepassingen zoals Kurzweil uit te breiden met functionaliteiten die gepersonaliseerde ATS aanbieden aan scholieren met dyslexie in de derde graad van het middelbaar onderwijs.

\medspace

Het experiment in de vergelijkende studie maakte gebruik van het GPT-3-model met de Da Vinci 3 engine en is alleen gefinetuned op basis van API-parameters. Zo bevat het ook geen extra vooraf getrainde data van wetenschappelijke artikelen. Uit een vergelijkende studie van taalmodellen voor gepersonaliseerde ATS bleek dat zowel de drie geteste HuggingFace-modellen via API als het geteste GPT-3-model via API in staat waren om moeilijke woorden te identificeren en te vervangen. Hoewel de vrij beschikbare modellen op HuggingFace lexicale vereenvoudiging mogelijk kunnen maken, staan ze in de schaduw van GPT-3, dat als API vrij beschikbaar is voor ontwikkelaars. Het GPT-3-model kan een baanbrekende oplossing bieden voor gepersonaliseerde ATS van wetenschappelijke artikelen, omdat het snel en efficiënt moeilijke woorden kan herkennen in doorlopende tekst en structurele aanpassingen kan maken aan de oorspronkelijke tekst.

Dit resultaat bevestigt de verwachting dat GPT-3 beter in staat is om gepersonaliseerde ATS aan te bieden in vergelijking met vrij beschikbare HuggingFace-taalmodellen. Een mogelijke verklaring hiervoor is de trainingsdata en de complexiteit van het taalmodel, mede door het aantal parameters dat ze bevatten. Er zijn lichte verschillen in lexicale complexiteit tussen de HuggingFace-taalmodellen die getraind waren op wetenschappelijke artikelen en taalmodellen die getraind waren op algemene data. Meer onderzoek is echter nodig om deze verschillen beter te begrijpen binnen de context van wetenschappelijke artikelen. De studie kon geen gebruikmaken van het opvolgende GPT-model, namelijk GPT-4, en hield geen rekening met het hoofdstuk waarin die een zin beoordeelt. Daarom bleven vragen over het verschil na ATS per hoofdstuk in een wetenschappelijk artikel onbeantwoord. Dit vereist verder onderzoek.

LLM's, waaronder GPT-3, kunnen vragen beantwoorden en tekstvereenvoudiging vergemakkelijken. Er is meer onderzoek nodig naar de verschillen op taalgebied in relatie tot de toename van parameters bij grotere taalmodellen, zoals GPT-4 en LLaMa. Verder onderzoek naar de effecten van het meegeven van doelgroepen via prompts is ook nodig. Daarnaast zou toekomstig onderzoek zich kunnen richten op het potentieel van de combinatie van GPT-3 en \textit{full-text-search}-technologieën. Meer onderzoek is nodig naar de verschillen tussen taalmodellen die getraind zijn op wetenschappelijke artikelen en taalmodellen die getraind zijn op algemene data binnen de context van wetenschappelijke artikelen. Er is ook weinig onderzoek gedaan naar het verschil tussen het laten schrijven van prompts en het gebruik van vooraf gedefinieerde prompts, wat verdere aandacht verdient. Er is behoefte aan onderzoek naar het gebruik van nieuwe modellen zoals GPT-4 en Bing AI in het onderwijs. De verschillen tussen FRE en FOG waren onderling miniem. Verder onderzoek is nodig om de bruikbaarheid van deze leesgraadscores te bepalen en te begrijpen hoe ze zich verhouden tot de kwaliteit van de vereenvoudigde tekst.

\medspace

Verworven kennis en aangeleerde tools uit alle richtingen Toegepaste Informatica aan Vlaamse Hogescholen, lieten toe om het prototype voor de webtool te ontwikkelen. Dit prototype dient slechts als een haalbaarheidsmeting voor ontwikkelaars bij het ontwikkelen van dergelijke toepassing. Het is belangrijk dat de lezer zich bewust is van het feit dat de webtool zich baseert op eerder onderzochte kenmerken en technieken die de impact van tekstvereenvoudiging met MTS hebben aangetoond bij scholieren met dyslexie. 

Daarnaast gebeurde de ontwikkeling van het prototype met het oog op een snelle en eenduidige implementatie van technieken die voordien enkel beschikbaar waren via commandline. Tijdens de ontwikkeling van het prototype is gebleken dat ontwikkelaars met vrij beschikbare middelen en API's in staat zijn om gepersonaliseerde ATS-toepassingen te bieden aan scholieren met dyslexie in de derde graad van het middelbaar onderwijs. Zo kunnen ontwikkelaars het stappenplan volgen om een vergelijkbaar resultaat te behalen, alsook de taken binnenin een projectteam parallel te laten uitvoeren volgens de flowchart.

Dit resultaat komt overeen met de verwachting dat ontwikkelaars over de benodigde tools beschikken om een dergelijk prototype voor gepersonaliseerde ATS te maken. Een mogelijke verklaring hiervoor is de beschikbaarheid van \textit{open-source} tools en Python-bibliotheken die ontwikkelaars in staat stellen complexe taken eenvoudig uit te voeren. Toch moet de lezer zich ervan bewust zijn dat het prototype niet getest is bij het doelpubliek tijdens het begrijpend lezen van een wetenschappelijk artikel. Daarom kan het alleen dienen als een meting van de haalbaarheid voor ontwikkelaars. Er is momenteel een gebrek aan wetenschappelijke vakliteratuur met betrekking tot tekstvereenvoudiging met ATS voor deze specifieke doelgroep.

Logopedisten of studenten in deze richting kunnen dit prototype gebruiken om onderzoek uit te voeren naar het effect op leesbegrip bij scholieren met dyslexie in de derde graad van het middelbaar onderwijs. Onderzoekers binnen het vakdomein secundair onderwijs kunnen de effecten van deze tool observeren bij leerlingen en leerkrachten in het middelbaar onderwijs. Er is echter meer onderzoek nodig om de inzet van gepersonaliseerde ATS-toepassingen en browserextensies voor tekstvereenvoudiging in het onderwijs te verbeteren. Er is behoefte aan een toepassing die alle functionaliteiten kan combineren. Bovendien is er behoefte aan meer onderzoek naar tekstvereenvoudiging met ATS voor de specifieke doelgroep van scholieren met dyslexie. Ten slotte is er onderzoek nodig naar de verbetering van de inzet van gepersonaliseerde ATS-toepassingen en browserextensies voor tekstvereenvoudiging in het onderwijs. Ook moet er onderzoek worden gedaan naar het gebruik van gepersonaliseerde ATS-toepassingen en browserextensies voor tekstvereenvoudiging bij scholieren met dyslexie in de derde graad van het middelbaar onderwijs.