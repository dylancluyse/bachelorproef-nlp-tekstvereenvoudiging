%%=============================================================================
%% Discussie
%%=============================================================================

\chapter{\IfLanguageName{dutch}{Discussie}{Discussie}}%
\label{ch:discussie}

Dit onderzoek gebruikt drie onderzoeksmethoden om te bepalen hoe ontwikkelaars een optimale vorm van gepersonaliseerde ATS kunnen bieden aan scholieren met dyslexie in de derde graad van het middelbaar onderwijs.

\medspace

Uit de resultaten van de requirementsanalyse blijkt dat zowel erkende toepassingen als online tools onvoldoende functionaliteiten bieden voor gepersonaliseerde ATS. Daarnaast bieden deze tools onvoldoende gepersonaliseerde opmaakopties. Dit resultaat komt overeen met de verwachting dat bestaande tools niet specifiek gericht zijn op gepersonaliseerde ATS voor scholieren met dyslexie in de derde graad van het middelbaar onderwijs. Mogelijke verklaringen hiervoor zijn de complexiteit die gepaard gaat met de ontwikkeling van dergelijk toepassing, het gebrek aan iniatief binnen dit vakgebied en de populariteit van pure samenvattingstools, zoals in de literatuurstudie aangegeven door \textcite{Gooding2022}.

\medspace

Hoewel ChatGPT en Bing Chatbot functionaliteiten bieden voor gepersonaliseerde ATS, ontbreken eenduidige handelingen waardoor gebruikers moeite kunnen hebben met het vereenvoudigen van wetenschappelijke artikelen. Daarnaast houdt het model van ChatGPT geen rekening met verwijzingen of artikelen buiten de getrainde data, wat problemen kan veroorzaken voor data-integriteit. Hiertegenover staat Bing Chat, dat wel rekening houdt met externe referenties en daarmee een goede basis vormt voor ontwikkelaars om referentiemateriaal aan te bieden in ondersteunende onderwijstoepassingen. Verder onderzoek naar de toepassing van deze AI via een API is noodzakelijk en kan baanbrekend zijn voor de onderwijssector, ondersteund door \textcite{Roose2023, Garg2022}. Anderzijds is er de mogelijkheid om bestaande toepassingen, zoals Kurzweil, uit te breiden met functionaliteiten die gepersonaliseerde ATS aanbieden aan scholieren met dyslexie in de derde graad van het middelbaar onderwijs.

\medspace

Vervolgens wijzen de resultaten van de vergelijkende studie uit dat het GPT-3 model geschikter is voor gepersonaliseerde ATS. Het geteste GPT-3-model gebruikt de davinci-engine en finetuned alleen API-parameters. Zo bevat het ook geen extra \textit{pre-trained} data van wetenschappelijke artikelen. De vergelijkende studie wijst verder uit dat de drie geteste HF-modellen via API en het geteste GPT-3-model via API beschikken over CWI-functionaliteiten en substitution generation. Hoewel de vrij beschikbare HF-taalmodellen LS mogelijk kunnen maken, staan ze in de schaduw van GPT-3, dat als API vrij beschikbaar is voor ontwikkelaars. Het GPT-3-model kan een baanbrekende oplossing bieden voor gepersonaliseerde ATS van wetenschappelijke artikelen, want het taalmodel kan snel en efficiënt moeilijke woorden herkennen in doorlopende tekst en structurele aanpassingen maken aan de oorspronkelijke tekst.

\medspace

Dit resultaat bevestigt de verwachting dat GPT-3 beter in staat is om gepersonaliseerde ATS aan te bieden in vergelijking met vrij beschikbare HF-taalmodellen. Een verklaring hiervoor is de complexiteit van het taalmodel. De geteste taalmodellen zijn getraind op data van wetenschappelijke artikelen. Meer onderzoek is echter nodig om deze verschillen beter te begrijpen binnen de context van wetenschappelijke artikelen. LLM's, waaronder GPT-3, kunnen vragen beantwoorden en een eenduidige oplossing voor gepersonaliseerde ATS aan ontwikkelaars aanbieden. Er is onderzoek nodig naar de verschillen op taalgebied in relatie tot de toename van parameters bij grotere taalmodellen, zoals aangewezen in \textcite{Simon2021}. Er is behoefte aan onderzoek naar het gebruik van nieuwe modellen zoals GPT-4 en Bing Chat in het onderwijs. De scriptie kon geen gebruikmaken van GPT-4. Een opvolgend onderzoek met dit taalmodel is vereist om te testen of dit taalmodel over voldoende data beschikt om wetenschappelijke artikelen te vereenvoudigen op maat van scholieren met dyslexie in de derde graad van het middelbaar onderwijs. Verder onderzoek naar doelgroepinschattingen via prompts is ook nodig. Daarnaast zou toekomstig onderzoek zich kunnen richten op het potentieel van de combinatie van GPT-3 en textit{full-text-search}-technologieën. Onderzoek is nodig naar de verschillen tussen taalmodellen, die getraind zijn op wetenschappelijke artikelen, en taalmodellen, die getraind zijn op algemene data. 

\medspace

De vergelijkende studie bevatte minieme verschillen tussen de taalmodellen bij de leesgraadscores FRE en FOG. Verder wijst de vergelijkende studie uit dat de \textit{readability}-library geen directe manier heeft om de actieve stem van een zin te achterhalen. Zo kan het onderzoek geen vaststelling maken of de uitgeteste taalmodellen in staat zijn om passief naar actief te schrijven. Spacy textit{word embeddings} kunnen een alternatieve manier aanreiken om hulpwerkwoorden en vervoegingen van het werkwoord zijn te achterhalen. Verder onderzoek is nodig om de bruikbaarheid van leesgraadscores te bepalen en te begrijpen hoe ze zich verhouden tot de kwaliteit van de vereenvoudigde tekst. Toepassingen zoals TextInspector meer metrieken dan de uitgeteste leesgraadscores aan. Daarom is er meer onderzoek nodig naar een optimale om geautomatiseerde tekstanalyse uit te kunnen voeren.

\medspace

Verworven kennis en aangeleerde tools uit alle richtingen Toegepaste Informatica aan Vlaamse Hogescholen, lieten toe om het prototype voor de webtool te ontwikkelen. Dit prototype dient slechts als een haalbaarheidstoetsing voor ontwikkelaars bij het ontwikkelen van dergelijke toepassing. Het is belangrijk dat de lezer zich bewust is van het feit dat de webtool zich baseert op onderzochte kenmerken en technieken die de impact van tekstvereenvoudiging met MTS hebben aangetoond bij scholieren met dyslexie. Daarnaast gebeurde de ontwikkeling van het prototype met het oog op een snelle en eenduidige implementatie van technieken die voordien enkel beschikbaar waren via CLI. Tijdens de ontwikkeling van het prototype is gebleken dat ontwikkelaars met vrij beschikbare middelen en API's in staat zijn om gepersonaliseerde ATS-toepassingen te bieden aan scholieren met dyslexie in de derde graad van het middelbaar onderwijs. Zo kunnen ontwikkelaars het stappenplan volgen om een vergelijkbaar resultaat te behalen en optioneel de taken in een projectteam parallel laten uitvoeren volgens de flowchart.

\medspace

Dit resultaat komt overeen met de verwachting dat ontwikkelaars over de benodigde tools beschikken om een dergelijk prototype voor gepersonaliseerde ATS te maken. Een verklaring hiervoor is de beschikbaarheid van textit{open-source} tools en python-bibliotheken die ontwikkelaars in staat stellen complexe taken eenvoudig uit te voeren. Toch moet de lezer zich ervan bewust zijn dat het prototype niet getest is bij het doelpubliek tijdens het begrijpend lezen van een wetenschappelijk artikel. Daarom kan het alleen dienen als een meting van de haalbaarheid voor ontwikkelaars. Er is een gebrek aan wetenschappelijke vakliteratuur over tekstvereenvoudiging met ATS voor deze doelgroep.

\medspace

Tot slot kunnen logopedisten of studenten in een logopedische studierichting dit prototype gebruiken om onderzoek uit te voeren naar het effect van dit prototype op leesbegrip bij scholieren met dyslexie in de derde graad van het middelbaar onderwijs. Dit stemt overeen met de implicaties waar \textcite{Gooding2022} op wijst. Onderzoekers binnen het vakdomein secundair onderwijs kunnen de effecten en voor -of nadelen van deze tool observeren bij leerlingen en leerkrachten in het middelbaar onderwijs. Er is echter meer onderzoek nodig om de inzet van gepersonaliseerde ATS-toepassingen en browserextensies voor tekstvereenvoudiging in het onderwijs te verbeteren. Zo is er behoefte aan een toepassing die alle functionaliteiten kan combineren. Bovendien is er behoefte aan meer onderzoek naar tekstvereenvoudiging met ATS voor de specifieke doelgroep van scholieren met dyslexie.