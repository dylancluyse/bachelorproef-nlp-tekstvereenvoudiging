%%=============================================================================
%% Inleiding
%%=============================================================================

\chapter{\IfLanguageName{dutch}{Inleiding}{Introduction}}%
\label{ch:inleiding}

Iedereen wordt dagelijks geconfronteerd met lezen. Deze vaardigheid strekt zich uit tot elk aspect van ons dagelijks leven. Dit geldt des te meer in het onderwijs, waar leraren worden aangemoedigd om diverse leesmaterialen te gebruiken om lesinhouden op een authentieke manier over te brengen. Wetenschappelijke artikelen kunnen ingezet worden als leesvoer voor scholieren in de derde graad van het middelbaar onderwijs, maar de leesgraad van deze artikelen brengt een nieuwe uitdaging voor zowel scholieren als leerkrachten met zich mee. 

\medspace

Zo stampte het Amerikaanse onderwijs C.R.E.A.T.E.\footnote{https://teachcreate.org/} uit de grond. Dit initiatief zet scholieren tussen 12 en 18 jaar aan om wetenschappelijke artikelen te lezen in plaats van enkel boeken. Ze begrijpen hoe wetenschappers experimenten uitvoeren, plannen en resultaten analyseren en interpreteren. Vlaamse gelijkaardige initiatieven bestaan niet, maar Vlaamse lerarenopleidingen benadrukken het gebruik van divers didactisch leesmateriaal in de klas. Volgens het M-decreet en de leerplannen van het katholiek\footnote{https://pro.katholiekonderwijs.vlaanderen/basisoptie-stem/ondersteunend-materiaal} en het gemeenschapsonderwijs\footnote{https://g-o.be/stem/} worden Vlaamse leerkrachten geadviseerd om hun lessen op een toegankelijke manier aanbieden, zodat alle scholieren ongeacht leesachterstand worden meegenomen in het verhaal. 

\medspace

Vlaanderen is met een jaarlijks budget van 32 miljoen een pionier in Europa op het gebied van artificiële intelligentie (AI) op de werkvloer \autocite{Crevits2022}. Zo stampte de Vlaamse overheid verschillende AI- projecten uit de grond, om Vlaamse AI ontwikkelingen te ondersteunen en om AI- softwarebedrijven te inspireren. Het amai!-project\footnote{https://amai.vlaanderen/} brengt AI softwarebedrijven uit diverse domeinen samen, waaronder het onderwijs. Zij doelen op een automatisering van processen om de werkdruk bij leerkrachten te verminderen, door middel van real-time ondertiteling in de klas en een taalassistent voor leerkrachten in meertalige klasgroepen.


\section{\IfLanguageName{dutch}{Probleemstelling}{Problem Statement}}%
\label{sec:probleemstelling}

In 79 geïndustrialiseerde landen wordt de driejaarlijkse PISA-test afgenomen om de leesvaardigheid en wetenschappelijke geletterdheid van 15-jarige scholieren te meten. Uit de PISA-test van 2018 blijkt dat deze doelgroep in Vlaanderen zich echter negatief uit over leesplezier en daarmee de slechtst scorende doelgroep is van alle bevraagde landen. Zoals aangegeven in Figuur \ref{img:oeso-leesplezier} beschouwt bijna de helft van de bevraagden intensief lezen als tijdverspilling en slechts 17\% beschouwt lezen als een hobby. Dit is een dalende trend, want voordien lag deze trend hoger dan 20\%. Boeiende topics uit vakliteratuur zoals Humo, Trends of EOS magazines kunnen belemmerd worden door deze vorm van media bij deze doelgroep.

\begin{figure}[H]
	\begin{center}
		\includegraphics[width=\linewidth]{img/oeso-graphic-leesplezier.png}
	\end{center}
	\caption{Het leesplezier bij 15-jarigen volgens de PISA-test \autocite{DeMeyer2019}.}
	\label{img:oeso-leesplezier}
\end{figure}

% DEEL ROND INTENSIEF LEZEN
% TODO https://www.advragen.nl/wat-betekent-intensief-lezen/
% EINDE DEEL INTENSIEF LEZEN

\medspace

% TODO dit toevoegen na einde van eerste zin: 

Intensief leesbegrip valt niet te omzeilen in onze huidige samenleving, maar dit leesbegrip verschilt sterk in het middelbaar onderwijs. Zo benadrukt de inspectie van \textcite{Vlaanderen2020} dat intensief lezen een essentiële vaardigheid is en een direct effect kan hebben op vakken buiten Nederlands, zoals bij het lezen van vraagstukken bij wiskundige vakken, of het begrijpen van vakjargon bij STEM-vakken. 

\medspace 

Een doelgroep die extra met intensief leesbegrip geconfronteerd wordt, zijn scholieren met dyslexie. Onderzoeken van \textcite{Bonte2020, VanDerMeer2022} schatten dat ongeveer 15\% van de Vlaamse scholieren in het middelbaar onderwijs een vorm van dyslexie heeft. Zo kunnen scholieren met dyslexie bij het intensief lezen geconfronteerd worden met een moeizame en stroeve automatisering bij het lezen en spellen. Hoewel scholieren met dyslexie ondersteuning kunnen krijgen, mag de impact van leesstoornissen niet onderschat worden. De gevolgen hiervan kunnen zich namelijk doorzetten na het middelbaar onderwijs \autocite{Lissens2020}. Leesvaardigheid blijft daarmee cruciaal voor succes op school en in het werkveld. Scholieren met dyslexie kunnen problemen hebben met spelling, wat kan leiden tot onzekerheid en stress. Daarnaast zijn vooroordelen nog steeds een probleem en kunnen ze leiden tot stigmatisering. Echter toont onderzoek aan dat scholieren met dyslexie doorzettingsvermogen hebben en goede probleemoplossers zijn \autocite{Ghesquiere2018, Lissens2020, Bonte2020}. 

\medspace

Het leerplan voor STEM-vakken stimuleert het gebruik van wetenschappelijke artikelen, maar houdt niet altijd rekening met de bijhorende complexe leesgraad. De ingewikkelde woordenschat en syntax in wetenschappelijke artikelen kunnen een hindernis vormen voor de begrijpelijkheid van een tekst, waardoor scholieren met dyslexie de kerninhoud moeilijk kunnen doorgronden. Het handmatig vereenvoudigen van wetenschappelijke artikelen kan planning, tijd en energie van leerkrachten in de derde graad middelbaar onderwijs opslorpen. Het Vlaamse middelbaar onderwijs staat onder druk en docenten hebben moeite om met deze werkdruk boven water te blijven. 

\medspace

Nu is AI technologisch voldoende hoogstaand om tekstvereenvoudiging te automatiseren en om een baanbrekende oplossing aan te bieden aan het middelbaar onderwijs. Soortgelijke technologieën worden echter amper toegepast in het onderwijs. Er is terughoudendheid door enerzijds ouders van leerlingen \autocite{Martens2021a}, anderzijds door de weinige ontwikkelingen in schoolgerelateerde AI-software. Dit terwijl AI-ondersteuning in het onderwijs wel degelijk een positief effect heeft \autocite{Belpaeme2018, Kraft2020}. 

\medspace

% TODO BRON VOOR COMMANDLINE

Recente technologieën bieden reeds mogelijkheden tot tekstvereenvoudiging aan, maar zijn voorlopig enkel in commandline of met scripts ter beschikking. Voor het gebruik van taalmodellen of API's is uitgebreide informaticakennis nodig, waarover de meeste scholieren en leraren niet beschikken. Anderzijds zijn de huidige online tools te beperkt en eerder gericht op samenvatten; wat niet noodzakelijk bijdraagt tot een eenvoudigere tekst. Er is nood aan een intuïtieve en gebruikersvriendelijke toepassing die taalmodellen of API's kan integreren en aanpassen naargelang de specifieke behoeften van een student met dyslexie. Zo kan dit enerzijds ook de werkdruk bij leerkrachten verminderen, anderzijds scholieren in de derde graad ondersteunen bij het lezen van complexe wetenschappelijke artikelen. 

\section{\IfLanguageName{dutch}{Onderzoeksvraag}{Research question}}%
\label{sec:onderzoeksvraag}

Dit onderzoek beschrijft het gebruik van artificiële intelligentie in de vorm van tekstvereenvoudiging, als advies voor implementatie in het onderwijs. Specifiek om scholieren met dyslexie in de derde graad van het middelbaar onderwijs te ondersteunen bij het intensief lezen van wetenschappelijke artikelen. Hiervoor is de volgende onderzoeksvraag opgesteld: 

\begin{itemize}
	\item Hoe kan een wetenschappelijke artikel automatisch vereenvoudigd worden, gericht op de unieke noden van scholieren met dyslexie in de derde graad middelbaar onderwijs?
\end{itemize}

Om deze onderzoeksvraag te kunnen beantwoorden, moet een antwoord gezocht worden op de volgende deelvragen:

\begin{enumerate}
	% 1
	\item Welke specifieke noden hebben scholieren met dyslexie van de derde graad middelbaar onderwijs bij het begrijpen van complexere teksten? Aanvullend hierop: 
	\begin{itemize}
		\item Wat zijn de specifieke kenmerken van wetenschappelijke artikelen?
	\end{itemize} 
	% 2
	\item Welke aanpakken zijn er voor tekstvereenvoudiging?
	\begin{itemize}
		\item Hoe worden teksten handmatig vereenvoudigd voor scholieren met dyslexie?
		\item Welke toepassingen, tools en modellen zijn er beschikbaar om Nederlandse geautomatiseerde tekstvereenvoudiging met AI mogelijk te maken?
		\item Hoe kunnen geautomatiseerde tekstvereenvoudiging en gepersonaliseerde tekstvereenvoudiging gecombineerd worden?
	\end{itemize}
	%4 
	\item Welke functies ontbreken AI-toepassingen om geautomatiseerde tekstvereenvoudiging mogelijk te maken voor scholieren met dyslexie in de derde graad middelbaar onderwijs? 
	\begin{itemize}
		\item Welke manuele methoden voor tekstvereenvoudiging ontbreken in deze tools?
	\end{itemize}
	%3 
	\item Met welke valkuilen bij taalverwerking met AI moeten ontwikkelaars rekening houden?
	% 5
	\item Welk taalmodel of LLM is geschikt voor de ATS van wetenschappelijke artikelen voor scholieren met dyslexie in de derde graad van het middelbaar onderwijs, met dezelfde of gelijkaardige kwaliteiten als gepersonaliseerde MTS?
	% 6
	\item Hoe kan een intuïtieve lokale webtoepassing worden ontwikkeld die zowel scholieren met dyslexie als leerkrachten helpt bij het vereenvoudigen van wetenschappelijke artikelen met behoud van semantiek, jargon en zinsstructuren?
\end{enumerate}


\section{\IfLanguageName{dutch}{Onderzoeksdoelstelling}{Research objective}}%
\label{sec:onderzoeksdoelstelling}

Het onderzoek heeft als doel om de technologische en logopedische aspecten te identificeren die AI-ontwikkelaars in overweging moeten nemen bij het creëren van een op maat gemaakte AI-toepassing voor geautomatiseerde tekstvereenvoudiging, specifiek ontwikkeld voor scholieren in de derde graad. Het resultaat van dit onderzoek is een prototype voor een AI-toepassing voor tekstvereenvoudiging in de vorm van een webtool. De webtool heeft twee functies. Enerzijds kan de tool de inhoud van wetenschappelijke artikelen vereenvoudigen op basis van de specifieke behoeften van scholieren met dyslexie in de derde graad van het middelbaar onderwijs. Anderzijds biedt de tool een geautomatiseerde benadering voor lectoren om wetenschappelijke artikelen te vereenvoudigen op basis van geselecteerde parameters en deze vervolgens in een bruikbaar formaat (pdf of word) terug te geven. De invoer bij dit prototype is een wetenschappelijk artikel in tekst- of PDF-formaat.


\section{\IfLanguageName{dutch}{Opzet van deze bachelorproef}{Structure of this bachelor thesis}}%
\label{sec:opzet-bachelorproef}

De rest van deze bachelorproef is als volgt opgebouwd:

In Hoofdstuk~\ref{ch:stand-van-zaken} wordt een overzicht gegeven van de stand van zaken binnen het onderzoeksdomein, op basis van een literatuurstudie.

In Hoofdstuk~\ref{ch:methodologie} wordt de methodologie toegelicht en worden de gebruikte onderzoekstechnieken besproken om een antwoord te kunnen formuleren op de onderzoeksvragen. Eerst wordt er een requirementsanalyse uitgevoerd, gevolgd door de ontwikkeling van een prototype voor tekstvereenvoudiging. In Hoofdstuk~\ref{ch:resultaten} worden de resultaten gegeven op dit onderzoek. 

In Hoofdstuk~\ref{ch:conclusie}, tenslotte, wordt de conclusie gegeven en een antwoord geformuleerd op de onderzoeksvragen. Ten slotte geeft Hoofdstuk~\ref{ch:discussie} verdere aanbevelingen en aanzet voor toekomstig onderzoek binnen de bestudeerde domeinen. 