%%=============================================================================
%% Inleiding
%%=============================================================================

\chapter{\IfLanguageName{dutch}{Inleiding}{Introduction}}%
\label{ch:inleiding}

% De inleiding moet de lezer net genoeg informatie verschaffen om het onderwerp te begrijpen en in te zien waarom de onderzoeksvraag de moeite waard is om te onderzoeken. In de inleiding ga je literatuurverwijzingen beperken, zodat de tekst vlot leesbaar blijft. Je kan de inleiding verder onderverdelen in secties als dit de tekst verduidelijkt. Zaken die aan bod kunnen komen in de inleiding~\autocite{Pollefliet2011}:

%\begin{itemize}
%  \item context, achtergrond
%  \item afbakenen van het onderwerp
%  \item verantwoording van het onderwerp, methodologie
%  \item probleemstelling
%  \item onderzoeksdoelstelling
%  \item onderzoeksvraag
%  \item \ldots
%\end{itemize}

Ontsnappen aan taal is bijna onmogelijk. Dagelijks komen mensen in aanraking met talen, van nieuwsartikelen tot de ondertiteling van Netflix-series, ongeacht de doelgroep. Het onderwijs richt zich de afgelopen tien jaar sterk op het gebruik van gevarieerde bronnen in lessen. Echter, de moeilijkheidsgraad van deze bronnen verandert niet, omdat de noodzaak aan verscheidenheid ook de noodzaak aan uitdagingen met zich meebrengt. STEM-docenten in een derde graad middelbaar onderwijs moeten volgens het leerplan hun theorielessen op een toegankelijke manier aanbieden, zodat iedereen betrokken is bij het verhaal.

\section{\IfLanguageName{dutch}{Probleemstelling}{Problem Statement}}%
\label{sec:probleemstelling}

% Uit je probleemstelling moet duidelijk zijn dat je onderzoek een meerwaarde heeft voor een concrete doelgroep. De doelgroep moet goed gedefinieerd en afgelijnd zijn. Doelgroepen als ``bedrijven,'' ``KMO's'', systeembeheerders, enz.~zijn nog te vaag. Als je een lijstje kan maken van de personen/organisaties die een meerwaarde zullen vinden in deze bachelorproef (dit is eigenlijk je steekproefkader), dan is dat een indicatie dat de doelgroep goed gedefinieerd is. Dit kan een enkel bedrijf zijn of zelfs één persoon (je co-promotor/opdrachtgever).

% bron leerplan

Scholieren met dyslexie in het middelbaar onderwijs hebben meerdere horden waar ze over moeten gaan. Ze worden echter niet aan hun lot over gelaten, want zij kunnen rekenen op de steun van coaches en vrij beschikbare middelen om hun achterstand zo kort mogelijk te houden. Het leerplan bij STEM-vakken moedigt het gebruik van wetenschappelijke artikelen aan, al wordt er minder rekening gehouden met de moeilijkheidsgraad die ze met zich meebrengen. Complexe woordenschat en zinsopbouw vormen een mistlaag van onduidelijkheid die zich uitstrijkt over de artikelen en zo raken scholieren niet tot de kerninhoud.

% bronnen

STEM-docenten van het derde graad middelbaar onderwijs zijn in staat om wetenschappelijke artikelen te vereenvoudigen of samen te vatten, maar dit vergt tijd en energie. Het onderwijs staat onder druk en docenten kunnen amper de werkdruk aan. Er is nood aan software die de tekstinhoud van wetenschappelijke artikelen automatisch kan vereenvoudigen, met voorkeur op maat van scholieren met dyslexie. Een toepassing zoals deze verlicht routinematige arbeid van STEM-docenten en biedt scholieren met dyslexie in de derde graad middelbaar onderwijs een kans om de kern van een tekst mee te hebben.


\section{\IfLanguageName{dutch}{Onderzoeksvraag}{Research question}}%
\label{sec:onderzoeksvraag}

% Wees zo concreet mogelijk bij het formuleren van je onderzoeksvraag. Een onderzoeksvraag is trouwens iets waar nog niemand op dit moment een antwoord heeft (voor zover je kan nagaan). Het opzoeken van bestaande informatie (bv. ``welke tools bestaan er voor deze toepassing?'') is dus geen onderzoeksvraag. Je kan de onderzoeksvraag verder specifiëren in deelvragen. Bv.~als je onderzoek gaat over performantiemetingen, dan 

% Dit onderzoek wijst aan welke transformaties er nodig zijn om de tekstinhoud van een wetenschappelijk artikel automatisch met kunstmatige intelligentie te vereenvoudigen, specifiek gericht op de noden van een scholier met dyslexie in het derde graad middelbaar onderwijs. 

Dit onderzoek toont aan hoe de inhoud van wetenschappelijke artikelen door middel van kunstmatige intelligentie automatisch vereenvoudigd kan worden, specifiek gericht op de noden van een scholier met dyslexie in het derde graad middelbaar onderwijs.

Om een antwoord op deze onderzoeksvraag te vinden, moet het onderzoek eerst zeven fasen doorlopen. (temp)
\begin{itemize}
	\item Wat is tekstvereenvoudiging? Allereerst moeten er een definitie worden gevormd wat tekstvereenvoudiging inhoudt en in welke soorten een tekst kan worden vereenvoudigd. 
	\item Welke soorten van tekstvereenvoudiging zijn er?
	\item Wat zijn de voordelen van wetenschappelijke artikelen te vereenvoudigen bij scholieren met dyslexie in de derde graad middelbaar onderwijs? Waarom speelt tekstvereenvoudiging een rol bij wetenschappelijke artikelen?
	\item Wat zijn de struikelblokken bij het vereenvoudigen van wetenschappelijke artikelen?
	\item Welke fasen heeft een pipeline voor tekstvereenvoudiging bij wetenschappelijke artikelen? Welke modellen zijn er white-box of black-box?
	\item Aan welke metrieken moet een vereenvoudigd wetenschappelijk artikel voldoen? In welke mate kan de eindgebruiker hiervan op de hoogte gesteld worden? 
\end{itemize}


\section{\IfLanguageName{dutch}{Onderzoeksdoelstelling}{Research objective}}%
\label{sec:onderzoeksdoelstelling}

% Wat is het beoogde resultaat van je bachelorproef? Wat zijn de criteria voor succes? Beschrijf die zo concreet mogelijk. Gaat het bv.\ om een proof-of-concept, een prototype, een verslag met aanbevelingen, een vergelijkende studie, enz.

Het resultaat van dit onderzoek is een vergelijkende studie en een prototype voor een toepassing die de tekstinhoud van een wetenschappelijke paper zal omzetten.

De vergelijkende studie zal vereenvoudigde of samengevatte teksten van drie verschillende soorten programma's vergelijken:
\begin{itemize}
	\item Toepassingen die momenteel in het onderwijs worden ingezet en waarvan licenties aan te vragen zijn voor scholieren in het derde graad van het middelbaar.
	\item Toepassingen die online terug te vinden zijn.
	\item Een zelfgemaakte prototype dat de inhoud van een wetenschappelijke paper automatisch zal vereenvoudigen met kunstmatige intelligentie.
\end{itemize}

Als tweede onderdeel is er het prototype om wetenschappelijke artikelen automatisch te vereenvoudigen naargelang de noden van een scholier in de derde graad van het middelbaar onderwijs. Het prototype zal geen rekening houden met de transformatie van het bronbestand naar de tekstinhoud. Dergelijke AI-toepassingen of AI-modellen die tekst uit afbeeldingen of PDF-bestanden halen, bestaan al. De invoer van dit prototype is een wetenschappelijk artikel van 300 tot 500 woorden lang. De uitvoer van dit prototype is een vereenvoudigde versie van ditzelfde wetenschappelijk artikel. Metrieken, indien mogelijk per zin, worden weergegeven. Verdere concretisering volgt...

\section{\IfLanguageName{dutch}{Opzet van deze bachelorproef}{Structure of this bachelor thesis}}%
\label{sec:opzet-bachelorproef}

% Het is gebruikelijk aan het einde van de inleiding een overzicht te
% geven van de opbouw van de rest van de tekst. Deze sectie bevat al een aanzet die je kan aanvullen/aanpassen in functie van je eigen tekst.

De rest van deze bachelorproef is als volgt opgebouwd:

In Hoofdstuk~\ref{ch:stand-van-zaken} wordt een overzicht gegeven van de stand van zaken binnen het onderzoeksdomein, op basis van een literatuurstudie.

In Hoofdstuk~\ref{ch:methodologie} wordt de methodologie toegelicht en worden de gebruikte onderzoekstechnieken besproken om een antwoord te kunnen formuleren op de onderzoeksvragen.

% TODO: Vul hier aan voor je eigen hoofstukken, één of twee zinnen per hoofdstuk

TODO
\begin{itemize}
	\item Wat is tekstvereenvoudiging? Allereerst moeten er een definitie worden gevormd wat tekstvereenvoudiging inhoudt en in welke soorten een tekst kan worden vereenvoudigd. 
	\item Welke soorten van tekstvereenvoudiging zijn er?
	\item Wat zijn de voordelen van wetenschappelijke artikelen te vereenvoudigen bij scholieren met dyslexie in de derde graad middelbaar onderwijs? Waarom speelt tekstvereenvoudiging een rol bij wetenschappelijke artikelen?
	\item Wat zijn de struikelblokken bij het vereenvoudigen van wetenschappelijke artikelen?
	\item Welke fasen heeft een pipeline voor tekstvereenvoudiging bij wetenschappelijke artikelen? Welke modellen zijn er white-box of black-box?
	\item Aan welke metrieken moet een vereenvoudigd wetenschappelijk artikel voldoen? In welke mate kan de eindgebruiker hiervan op de hoogte gesteld worden? 
\end{itemize}

In Hoofdstuk~\ref{ch:conclusie}, tenslotte, wordt de conclusie gegeven en een antwoord geformuleerd op de onderzoeksvragen. Daarbij wordt ook een aanzet gegeven voor toekomstig onderzoek binnen dit domein.