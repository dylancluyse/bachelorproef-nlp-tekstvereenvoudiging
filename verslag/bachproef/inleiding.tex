%%=============================================================================
%% Inleiding
%%=============================================================================

\chapter{\IfLanguageName{dutch}{Inleiding}{Introduction}}%
\label{ch:inleiding}

% De inleiding moet de lezer net genoeg informatie verschaffen om het onderwerp te begrijpen en in te zien waarom de onderzoeksvraag de moeite waard is om te onderzoeken. In de inleiding ga je literatuurverwijzingen beperken, zodat de tekst vlot leesbaar blijft. Je kan de inleiding verder onderverdelen in secties als dit de tekst verduidelijkt. Zaken die aan bod kunnen komen in de inleiding~\autocite{Pollefliet2011}:

%\begin{itemize}
%  \item context, achtergrond
%  \item afbakenen van het onderwerp
%  \item verantwoording van het onderwerp, methodologie
%  \item probleemstelling
%  \item onderzoeksdoelstelling
%  \item onderzoeksvraag
%  \item \ldots
%\end{itemize}

Het middelbaar onderwijs staat op springen. Dagelijks sneuvelen leerkrachten en leerlingen van het middelbaar onderwijs onder de te harde werkdruk. Daarnaast is taal vrijwel onmogelijk om aan te ontsnappen. Dagelijks komen mensen in aanraking met taal, van Nederlandse nieuwsartikelen tot de ondertiteling van Koreaanse Netflix-series, ongeacht de doelgroep. Lerarenopleidingen richt zich de afgelopen tien jaar sterk op het gebruik van gevarieerde bronnen in lessen. De moeilijkheidsgraad van deze bronnen verandert echter niet, want de noodzaak aan verscheidenheid brengt ook de noodzaak aan uitdagingen met zich mee. STEM-leerkrachten in een derde graad middelbaar onderwijs moeten volgens het leerplan van zowel het katholiek\footnote{https://pro.katholiekonderwijs.vlaanderen/basisoptie-stem/ondersteunend-materiaal} als het gemeenschapsonderwijs\footnote{https://g-o.be/stem/} hun theorielessen op een toegankelijke manier aanbieden, zodat iedereen betrokken is bij het verhaal.

\newline

Met een jaarlijks budget van 32 miljoen is België een pionier \autocite{Crevits2022} in het vakgebied kunstmatige intelligentie (AI) op de werkvloer.  Zo stampte de Vlaamse overheid verschillende AI-projecten uit de grond, om Vlaamse AI-ontwikkelingen te ondersteunen en inspireren. Het amai!-project\footnote{https://amai.vlaanderen/}  brengt AI-softwarebedrijven samen uit verschillende domeinen. Dit project leidt tot het ontstaan van AI-toepassingen die processen automatiseren om de werkdruk te verminderen, zoals binnen het onderwijs \textit{real-time} ondertiteling en een taalassistent voor leerkrachten in meertalige klasgroepen.

\section{\IfLanguageName{dutch}{Probleemstelling}{Problem Statement}}%
\label{sec:probleemstelling}

% Uit je probleemstelling moet duidelijk zijn dat je onderzoek een meerwaarde heeft voor een concrete doelgroep. De doelgroep moet goed gedefinieerd en afgelijnd zijn. Doelgroepen als ``bedrijven,'' ``KMO's'', systeembeheerders, enz.~zijn nog te vaag. Als je een lijstje kan maken van de personen/organisaties die een meerwaarde zullen vinden in deze bachelorproef (dit is eigenlijk je steekproefkader), dan is dat een indicatie dat de doelgroep goed gedefinieerd is. Dit kan een enkel bedrijf zijn of zelfs één persoon (je co-promotor/opdrachtgever).

% Uitleg over dyslexie.

Volgens \textcite{Ghesquiere2018} heeft ongeveer 5 tot 9 \% van de Nederlandstalige bevolking\footnote{Deze schatting is gebaseerd op de Vlaamse en Nederlandse bevolking.} te maken met dyslexie. Verder benadrukt de studie van \textcite{Lissens2020} dat de impact van leerstoornissen niet stopt na het middelbaar onderwijs. Scholieren met dyslexie in het middelbaar onderwijs krijgen te maken met unieke uitdagingen. Gelukkig worden ze niet aan hun lot overgelaten en kunnen ze rekenen op ondersteuning van coaches en beschikbare hulpmiddelen om hun achterstand te beperken. Het leerplan voor STEM-vakken stimuleert het gebruik van wetenschappelijke artikelen, maar houdt niet altijd rekening met de moeilijkheidsgraad ervan. De complexe woordenschat en zinsopbouw in deze artikelen vormen een barrière voor de begrijpelijkheid van een tekst, waardoor de scholieren de kerninhoud moeilijk kunnen doorgronden. Een oplossing hiervoor is om de tekst te vereenvoudigen, waardoor de kerninhoud wordt behouden.

\newline

\begin{quote}
	“Ik vind het zeker de moeite waard om te onderzoeken. Leerlingen met dyslexie hebben naast problemen met technisch lezen (maken meer leesfouten en lezen vaak trager) ook problemen met begrijpend lezen. Samengestelde zinnen en complexere zinnen worden niet altijd even makkelijk begrepen (belangrijke signaalwoorden niet herkend, woordbetekenissen worden moeizamer uit de tekstcontext gehaald, enz...) Een toegankelijke tekststructuur kan voor veel doelgroepen soelaas brengen (dat is trouwens wat organisaties zoals 'Wablieft' voor ogen hebben.) “
\end{quote}

\newline

Wetenschappelijke artikelen vereenvoudigen vraagt tijd en energie van docenten in de derde graad middelbaar onderwijs. Het middelbaar onderwijs staat onder druk en docenten hebben moeite om met deze werkdruk boven water te blijven. Daarom is er nood aan software die wetenschappelijke artikelen automatisch kan vereenvoudigen, specifiek gericht op de noden van scholieren met dyslexie. Een dergelijke toepassing vermindert het routinematige werk van STEM-docenten en biedt scholieren met dyslexie in de derde graad middelbare onderwijs de mogelijkheid om de kern van een tekst sneller te begrijpen.

\section{\IfLanguageName{dutch}{Onderzoeksvraag}{Research question}}%
\label{sec:onderzoeksvraag}

% Wees zo concreet mogelijk bij het formuleren van je onderzoeksvraag. Een onderzoeksvraag is trouwens iets waar nog niemand op dit moment een antwoord heeft (voor zover je kan nagaan). Het opzoeken van bestaande informatie (bv. ``welke tools bestaan er voor deze toepassing?'') is dus geen onderzoeksvraag. Je kan de onderzoeksvraag verder specifiëren in deelvragen. Bv.~als je onderzoek gaat over performantiemetingen, dan 

% Dit onderzoek toont aan hoe de inhoud van wetenschappelijke artikelen met kunstmatige intelligentie automatisch vereenvoudigd kan worden, specifiek gericht op de noden van een scholier met dyslexie in het derde graad middelbaar onderwijs. Om een antwoord op deze onderzoeksvraag te vinden, moet het onderzoek eerst zeven fasen doorlopen. Het doel van dit onderzoek is om te achterhalen met welke technologische en logopedische aspecten AI ontwikkelaars rekening moeten houden bij het een de ontwikkeling van een adaptieve AI toepassing voor geautomatiseerde tekstvereenvoudiging. 

De volgende onderzoeksvraag is opgesteld: ”Hoe kan een wetenschappelijke artikel automatisch vereenvoudigd worden, gericht op de unieken noden van scholieren met dyslexie in de derde graad middelbaar onderwijs?”. Daarnaast worden de volgende deelvragen beantwoord.

\begin{itemize}
\item Welke aanpakken zijn er voor geautomatiseerde tekstvereenvoudiging? Aansluitende vraag: "Hoe worden teksten handmatig vereenvoudigd voor scholieren met dyslexie?"
\item Welke specifieke noden hebben scholieren van de derde graad middelbaar onderwijs bij het begrijpen van complexere teksten?
\item Wat zijn de specifieke kenmerken van wetenschappelijke artikelen?
\item Met welke valkuilen bij taalverwerking met AI moeten ontwikkelaars rekening houden?
\item Welke toepassingen, tools en modellen zijn er beschikbaar om Nederlandse geautomatiseerde tekstvereenvoudiging met AI mogelijk te maken?
\item Welke functies ontbreken AI-toepassingen om geautomatiseerde tekstvereenvoudiging mogelijk te maken voor scholieren met dyslexie in de derde graad middelbaar onderwijs? Aansluitende vraag: ”Welke manuele methoden voor tekstverereenvoudiging komen niet in deze tools voor?"
\end{itemize}


\section{\IfLanguageName{dutch}{Onderzoeksdoelstelling}{Research objective}}%
\label{sec:onderzoeksdoelstelling}

% Wat is het beoogde resultaat van je bachelorproef? Wat zijn de criteria voor succes? Beschrijf die zo concreet mogelijk. Gaat het bv.\ om een proof-of-concept, een prototype, een verslag met aanbevelingen, een vergelijkende studie, enz.

Het doel van dit onderzoek is om te achterhalen met welke technologische en logopedische aspecten AI ontwikkelaars rekening moeten houden bij het een de ontwikkeling van een adaptieve AI toepassing voor geautomatiseerde tekstvereenvoudiging. Het resultaat van dit onderzoek is een prototype voor een toepassing die de tekstinhoud van een wetenschappelijke paper zal vereenvoudigen, naargelang de specifieke noden van een scholier met dyslexie in de derde graad middelbaar onderwijs. Het prototype houdt rekening met de transformatie van het bronbestand, bijvoorbeeld een PDF of een afbeelding, naar de tekstinhoud. Hiervoor bestaan er kant-en-klare pakketten die het omzettingswerk al voor de ontwikkelaar doen. De invoer van dit prototype is een wetenschappelijk artikel van minstens 500 woorden lang.

% Als tweede onderdeel wordt er een prototype ontwikkeld om wetenschappelijke artikelen automatisch te vereenvoudigen, specifiek gericht op de noden van een scholier in de derde graad middelbaar onderwijs. Het prototype houdt geen rekening met de transformatie van het bronbestand, bijvoorbeeld een PDF of een afbeelding, naar de tekstinhoud. Dergelijke AI-toepassingen of AI-modellen die tekst uit afbeeldingen of PDF-bestanden halen, bestaan al. De invoer van dit prototype is een wetenschappelijk artikel van 300 tot 500 woorden lang. De uitvoer van dit prototype is een vereenvoudigde versie van ditzelfde wetenschappelijk artikel. Metrieken, indien mogelijk per zin, worden weergegeven. Verdere concretisering volgt...

\section{\IfLanguageName{dutch}{Opzet van deze bachelorproef}{Structure of this bachelor thesis}}%
\label{sec:opzet-bachelorproef}

% Het is gebruikelijk aan het einde van de inleiding een overzicht te
% geven van de opbouw van de rest van de tekst. Deze sectie bevat al een aanzet die je kan aanvullen/aanpassen in functie van je eigen tekst.

De rest van deze bachelorproef is als volgt opgebouwd:

In Hoofdstuk~\ref{ch:stand-van-zaken} wordt een overzicht gegeven van de stand van zaken binnen het onderzoeksdomein, op basis van een literatuurstudie.

In Hoofdstuk~\ref{ch:methodologie} wordt de methodologie toegelicht en worden de gebruikte onderzoekstechnieken besproken om een antwoord te kunnen formuleren op de onderzoeksvragen.

% TODO: Vul hier aan voor je eigen hoofstukken, één of twee zinnen per hoofdstuk

\begin{itemize}
	\item Welke aanpakken zijn er voor geautomatiseerde tekstvereenvoudiging? Aansluitende vraag: "Hoe worden teksten handmatig vereenvoudigd voor scholieren met dyslexie?"
	\item Welke specifieke noden hebben scholieren van de derde graad middelbaar onderwijs bij het begrijpen van complexere teksten?
	\item Wat zijn de specifieke kenmerken van wetenschappelijke artikelen? 
	\item Met welke valkuilen bij taalverwerking met AI moeten ontwikkelaars rekening houden?
	\item Welke toepassingen, tools en modellen zijn er beschikbaar om Nederlandstalige geautomatiseerde tekstvereenvoudiging met AI mogelijk te maken?
	\item Welke functies ontbreken AI-toepassingen om geautomatiseerde én adaptieve tekstvereenvoudiging mogelijk te maken voor \newline scholieren met dyslexie in de derde graad \newline middelbaar onderwijs? Aansluitende vraag: "Welke manuele methoden voor tekstvereenvoudiging komen niet in deze tools voor?"
\end{itemize}

In Hoofdstuk~\ref{ch:conclusie}, tenslotte, wordt de conclusie gegeven en een antwoord geformuleerd op de onderzoeksvragen. Daarbij wordt ook een aanzet gegeven voor toekomstig onderzoek binnen dit domein.