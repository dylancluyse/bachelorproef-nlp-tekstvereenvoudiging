%%=============================================================================
%% Conclusie
%%=============================================================================

\chapter{Conclusie}%
\label{ch:conclusie}

Deze scriptie tracht een antwoord te bieden op de volgende onderzoeksvraag:

\begin{itemize}
	\item Hoe kan een wetenschappelijk artikel automatisch vereenvoudigd worden, gericht op de unieke noden van scholieren met dyslexie in de derde graad middelbaar onderwijs?
\end{itemize}

% ontbrekende mts en ontbrekende
Allereerst geeft de requirementsanalyse nieuwe inzichten in de huidige toepassingen voor ATS. Zo beschikken online tools te weinig over gepersonaliseerde ATS-functionaliteiten, zoals gebleken in sectie \ref{sec:requirementsanalyse}. Daarnaast maken geen tools, uitgezonderd van software specifiek voor scholieren met dyslexie, geen gebruik van gepersonaliseerde opmaakopties. Toepassingen die wél wetenschappelijke artikelen kunnen opladen, beschikken over onvoldoende functionaliteiten om gepersonaliseerde ATS mogelijk te maken. Daartegenover ontbreken toepassingen die wél gepersonaliseerde ATS kunnen aanreiken over de nodige middelen om wetenschappelijke artikelen op een eenduidige manier op te laden. Om deze tools te kunnen gebruiken, moeten gebruikers \textit{commandline-interfaces} of chatbots gebruiken. Deze sectie benadrukt de nood aan een eenduidige toepassing voor scholieren en leerkrachten om wetenschappelijke teksten te laten vereenvoudigen.

\medspace

% welk taalmodel gebruiken?
De vergelijkende studie wijst uit dat de geteste taalmodellen in staat zijn om LS mogelijk te maken. SS is enkel beschikbaar bij het GPT-3 model. Dit taalmodel kan doelgroepen in grote lijn inschatten en ook SA-technieken toepassen, zoals het herschrijven van een tekst als opsomming of in tabelvorm. Andere geteste HF-taalmodellen behalen zwakkere resultaten dan het GPT-3 model en vereisen een extra vertaalfase. Een vertaalfase is niet nodig bij het aanspreken van de GPT-3 API. Het prototype moet gebruik maken van specifieke prompts en de technieken aangegeven door \textcite{McFarland2023, White2023}.

\medspace

% hoe een prototype opzetten?
Uit de ontwikkeling van het prototype voor gepersonaliseerde ATS blijkt dat de gebruikte \textit{open-source} AI en NLP-technologieën capabel zijn om tekstvereenvoudigingssoftware ermee te ontwikkelen. Zo kunnen ontwikkelaars gebruikmaken van PDFMiner om tekstinhoud uit wetenschappelijke artikelen te extraheren, van OpenAI's GPT-3 model via de API om gepersonaliseerde ATS mogelijk te maken en ten slotte van Pandoc om dynamische en gepersonaliseerde PDF-documenten automatisch te genereren. Binnen een webapplicatie kunnen eenduidige handelingen, gebouwd in JavaScript en HTML\&CSS, complexe commandlinehandelingen afhandelen. Ontwikkelaars kunnen met eenvoudige en open-source tools een webpagina opbouwen die voldoet aan de noden beschreven in \textcite{Rello2012a}. Hoewel het prototype niet voldoet aan alle vooraf opgestelde functionaliteiten, toch kunnen ontwikkelaars met de gebruikte softwarepakketten een prototype of volledig afgewerkte toepassing ontwikkelen die aan alle criteria kan voldoen. 

\medspace

Ontwikkelaars hebben toegang tot HF-taalmodellen voor LS-taken. Deze taalmodellen zijn echter ontoereikend voor gepersonaliseerde ATS, want ze ontbreken SS-technieken om de tekst op een syntactisch niveau te vereenvoudigen. Daarnaast beschikken de HF-taalmodellen over een ingebakken doelgroep, afhankelijk van de data waarop deze taalmodellen zijn getraind. GPT-3 is een geschikter model voor het vereenvoudigen van wetenschappelijke artikelen op maat van scholieren met dyslexie in de derde graad van het middelbaar onderwijs. Zo presteert GPT-3 goed op gepersonaliseerde LS en SS-technieken, maar het is belangrijk om op te merken dat geen enkel taalmodel de doelgroep altijd nauwkeurig kan inschatten. Extra trainingsdata, zoals aangeraden door \textcite{Gooding2022} in de vorm van leerstof op leesniveau van de doelgroep kan het model helpen bij de doelgroepsinschatting. Het gebruik van Engelstalige prompts met expliciete vermelding van de gewenste uitvoertaal, resulteert in coherentere teksten dan bij een Nederlandstalige prompt.