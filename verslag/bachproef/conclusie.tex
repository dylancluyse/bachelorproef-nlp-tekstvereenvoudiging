%%=============================================================================
%% Conclusie
%%=============================================================================

\chapter{Conclusie}%
\label{ch:conclusie}

\subsubsection{Software ontwikkelen voor scholieren met dyslexie in de derde graad van het middelbaar onderwijs.}

De erkende software uitgeleend aan scholieren met dyslexie in de derde graad van het middelbaar onderwijs voldoet niet aan de noden. De software biedt ondersteunende functionaliteiten aan zoals het aanmaken van een woordenlijst, alsook het markeren van zinnen om deze later om te vormen naar een tekst. Syntactische vereenvoudiging of abstraherende samenvatting zijn niet tot de orde. Online toepassingen staan verder en reiken functionaliteiten aan die zowel hoogstaand als effectief te reproduceren zijn. Echter is er geen manusje-van-alles en er is nood aan een toepassing die alle functionaliteiten kan combineren. Echter kunnen deze softwarepakketten opgeschaald worden door middel van LLM's, zodat deze de functionaliteiten hebben om verbeterde tekstvereenvoudigingstechnieken aan te reiken aan scholieren met dyslexie in de derde graad van het middelbaar onderwijs. 

\medskip

Het prototype wijst uit dat het combineren van functionaliteiten een haalbare zaak is voor zowel klein- als grootschalige softwareondernemingen. Met gebruik van kant-en-klare taalmodellen, API's en gekende programmeertalen zijn ontwikkelaars ertoe in staat om een webtoepassing te ontwikkelen die ondersteuning kan bieden aan scholieren met dyslexie in de derde graad van het middelbaar onderwijs. Dit prototype is gebouwd op eerder onderzochte visuele kenmerken en de impact van een vereenvoudigde tekst op de leessnelheid en -begrip bij een scholier met dyslexie tijdens het intensief lezen van een tekst. Het prototype werd niet uitgetest bij het doelpubliek en vereist verder uittesten. 

\medskip

Onderzoek naar het verschil tussen het laten schrijven van prompts en vooraf gedefinieerde prompts is schaars, maar deze keuze kan een effect hebben op het gedrag of ervaring van de eindgebruiker. De doelgroep wordt expliciet aangeduid in de prompts en is daarmee parameteriseerbaar. Er is verder onderzoek nodig naar de effecten op het meegeven van doelgroepen via prompts en of deze al dan niet rekening houden met de doelgroep.

\subsubsection{Verschillen tussen de handmatige tekst, het origineel, uitvoer van beschikbare tools en het prototype.}

Uit een vergelijkende studie blijkt dat er lichte verschillen zijn tussen de verschillende tools wat betreft de scores op het gebied van FRE, SMOG en Kincaid. Het prototype en het GPT-3-model dat wordt gebruikt voor gepersonaliseerde samenvattingen scoren beter op leesbaarheid. Een niet-gepersonaliseerde samenvatting scoort echter lager en heeft problemen in de doorlopende tekst, waarbij het taalmodel soms woorden plaatst die niet nuttig zijn in de context. Resoomer en Scispace verkorten effeHierdoor wordt de bruikbaarheid van deze scores en het gebruik van de Python-bibliotheek in twijfel getrokken. Hoewel deze scores een goed alternatief zijn om de leesbaarheid te meten, houden ze geen rekening met verkeerd geïnterpreteerde resultaten, zoals letterlijk overgenomen bronvermeldingen of verkeerd gegenereerde woordenschat door het taalmodel. Deze fouten zijn minder vaak aanwezig bij het vereenvoudigen van teksten met GPT-3. Er is meer onderzoek nodig om de bruikbaarheid van deze scores te bepalen en om te begrijpen hoe de scores zich verhouden tot de kwaliteit van de vereenvoudigde tekst. 

\medskip

Het vergelijken met een referentietekst blijft nog steeds een handmatige vergelijking en biedt een inkijk in hoe lectoren teksten kunnen vereenvoudigen. De vergelijkende studie hield geen rekening met het hoofdstuk waarin een zin werd beoordeeld. Vragen naar het verschil na een tekstvereenvoudiging per hoofdstuk in een wetenschappelijk artikel kan daarmee niet beantwoord worden en moet opgevolgd worden in een verder onderzoek.

\subsubsection{Het gebruik van HuggingFace-taalmodellen en LLM's voor een toepassing van tekstvereenvoudiging.}

Bestaande taalmodellen maken de ontwikkeling van toepassingen op het gebied van semantische analyse, kernwoordenidentificatie en het extraheren van samenvattingen veel eenvoudiger. Een prototype heeft aangetoond dat vrij beschikbare taalmodellen, zoals BERT en gerelateerde BERT-varianten, een oplossing bieden voor deze taken. Voor meer complexe vereenvoudigingstechnieken, zoals lexicale en syntactische vereenvoudiging en abstraherende samenvatting, zijn complexere taalmodellen zoals GPT-3 geschikt. Ontwikkelaars moeten echter rekening houden met de schaal van de modellen bij het maken van deze keuze. Hoewel vrij beschikbare modellen op HuggingFace in staat zijn om abstraherende samenvattingen of lexicale vereenvoudiging mogelijk te maken, staan ze in de schaduw van GPT-3, dat voor ontwikkelaars vrij beschikbaar is. 

\medskip

Omgekeerd is GPT-3 echter een overkill voor taken zoals het identificeren van kernwoorden of het aanduiden van belangrijke zinnen, die kosteneffectief kunnen worden aangepakt zonder het gebruik van GPT-3. GPT-3 moet niet voor iedere functionaliteit worden ingezet om zo kostenbesparend te werken. Het GPT-3 model maakt complexe en granulaire NLP-transformaties op lexicaal en syntactisch niveau mogelijk voor geautomatiseerde tekstvereenvoudiging. Echter houdt het model geen rekening met referenties buiten de getrainde data, wat tot problemen bij de credibiliteit van data kan leiden. Bing AI daarentegen doet dit wel en vormt een goede fundering voor ontwikkelaars om meer referentiemateriaal aan te bieden in ondersteunende software binnen het onderwijs. Verder onderzoek op de toepassing van deze AI via een mogelijke API is zeker nodig en kan baanbrekend zijn voor de onderwijssector. 

\subsubsection{Verdere finetuning en pre-training van taalmodellen.}

Het huidige prototype maakt geen gebruik van verdere finetuning en het GPT-3 model dat wordt gebruikt, is niet vooraf getraind op extra data van wetenschappelijke papers. Echter, er is wetenschappelijke data beschikbaar die vrij beschikbaar is en kan worden gebruikt om het GPT-3 taalmodel accurater te maken op interpretatie van complexiteit bij wetenschappelijke artikelen. Ondanks dit, is er wel een licht effect waargenomen op de verschillen in lexicale complexiteit tussen de HuggingFace-taalmodellen die wel getraind zijn op wetenschappelijke papers in vergelijking met taalmodellen die getraind zijn op algemene data. 

\medskip

Om de verschillen tussen deze taalmodellen binnen de context van wetenschappelijke papers beter te begrijpen, is meer onderzoek nodig. Het is belangrijk om op te merken dat de taalmodellen van OpenAI continu evolueren. Het verschil tussen GPT-2 en GPT-3 via de API is merkbaar en daarom wordt er overwogen om GPT-2 achterwege te laten, met oog op verdere edities van OpenAI's GPT-modellen. Op dit moment wordt GPT-4 uitgerold en Bing AI is beschikbaar volgens een wachtlijst. Echter, op dit moment zijn deze twee taalmodellen nog niet in staat om in productie te worden gebruikt. Daarom is verder onderzoek nodig naar het gebruik van deze modellen binnen het onderwijs.

\subsubsection{Deployment van het prototype en andere toepassingen voor tekstvereenvoudiging.}

Dit onderzoek heeft zich beperkt tot de tekstinhoud van een vereenvoudigd artikel. Er is echter meer onderzoek nodig naar hoe de inzet van webtoepassingen en browserextensies voor tekstvereenvoudiging in het onderwijs kan worden verbeterd. Met alsmaar grotere taalmodellen, zoals het opkomende GPT-4 en LLaMa, is er ook meer onderzoek nodig naar de verschillen op taalvlak ten opzichte van de toename in parameters. Het GPT-3 model dat in dit onderzoek werd gebruikt, maakte enkel gebruik van aangepaste parameters zoals de \textit{temperature} en \textit{top\_p}. Hoewel de overstap naar andere taalmodellen kostelijk kan zijn voor ontwikkelaars, is het belangrijk om te onderzoeken of en hoe deze nieuwe modellen kunnen bijdragen aan betere resultaten. 

\medskip

Het is echter eerder uitgewezen dat de grootte van taalmodellen alsmaar minder relevant wordt. Een andere uitdaging die aan het licht kwam tijdens het onderzoek is de kwaliteit van de geëxtraheerde tekstinhoud uit een PDF- of Docx-bestand door de gebruikte Python-libraries. Hoewel LLM's rekening houden met mogelijke noise, kunnen woorden tot zinnen ontbreken bij het ophalen van tekst uit een PDF afhankelijk van het formaat waarin het is opgeslagen. Om deze uitdaging aan te pakken, is onderzoek nodig naar robuuste oplossingen om PDF-inhoud uit een tekst te extraheren, met minimale invloed van het gebruikte softwarepakket.