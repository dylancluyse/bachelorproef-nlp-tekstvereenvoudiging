%%=============================================================================
%% Conclusie
%%=============================================================================

\chapter{Conclusie}%
\label{ch:conclusie}

\subsubsection{Welke specifieke noden hebben scholieren met dyslexie van de derde graad middelbaar onderwijs bij het begrijpen van complexere teksten?}


\subsubsection{Welke aanpakken zijn er voor tekstvereenvoudiging?}

% Hoe worden teksten handmatig vereenvoudigd voor scholieren met dyslexie?
% Welke toepassingen, tools en modellen zijn er beschikbaar om Nederlandse geautomatiseerde tekstvereenvoudiging met AI mogelijk te maken?
% Hoe kunnen geautomatiseerde tekstvereenvoudiging en gepersonaliseerde tekstvereenvoudiging gecombineerd worden?


\subsubsection{Welke functies ontbreken AI-toepassingen om geautomatiseerde tekstvereenvoudiging mogelijk te maken voor scholieren met dyslexie in de derde graad middelbaar onderwijs?}
% Welke manuele methoden voor tekstvereenvoudiging ontbreken in deze tools?

 
\subsubsection{Met welke valkuilen bij taalverwerking met AI moeten ontwikkelaars rekening houden?}


\subsubsection{Welke taalmodellen of LLM's zijn geschikt voor automatische tekstvereenvoudiging voor vereenvoudigde wetenschappelijke artikelen voor scholieren met dyslexie in de derde graad van het middelbaar onderwijs met dezelfde of gelijkaardige kwaliteiten als manuele tekstvereenvoudiging?}


\subsubsection{Hoe kan een intuïtieve lokale webtoepassing worden ontwikkeld die zowel scholieren met dyslexie als leerkrachten helpt bij het vereenvoudigen van wetenschappelijke artikelen met behoud van semantiek, jargon en zinsstructuren?}