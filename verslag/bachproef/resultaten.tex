\chapter{\IfLanguageName{dutch}{Resultaten}{Results}}%
\label{ch:resultaten}

In dit hoofdstuk worden de resultaten uit de requirementsanalyse, vergelijkende studie en de ontwikkeling van het prototype besproken. 

\section{Requirementsanalyse}

% Onderzoeksvraag 1 Welke functies ontbreken AI-toepassingen om geautomatiseerde tekstvereenvoudiging mogelijk te maken voor scholieren met dyslexie in de derde graad middelbaar onderwijs?

Huidige door de overheid erkende software kunnen woorden- en synoniemenlijsten genereren na een handmatige selectie van de woorden. Toch kunnen de geteste softwarepakketten geen syntactische vereenvoudiging toepassen op de oorspronkelijke tekst. Daarnaast ontbreken deze erkende software de nodige opmaakopties om een gepersonaliseerde leeservaring aan te bieden. Online beschikbare tools zijn in staat om ATS op zowel Nederlandstalige als Engelstalige wetenschappelijke artikelen toe te passen. Echter houdt geen enkele tool (bewust) rekening met de doelgroep. Daardoor worden alle moeilijke woorden indien mogelijk vervangen door een eenvoudiger synoniem. Indien er geen eenvoudiger synoniem is, dan wordt het woord behouden in de vereenvoudigde tekst en het taalmodel voegt geen ondersteunende definitie toe. Daarnaast maken deze geteste tools geen gebruik van gepersonaliseerde opmaakopties, noch van de webtoepassing noch van het uitvoerbestand. Alle prompts bij ChatGPT en Bing Chat kunnen de benoemde technieken voor lexicale vereenvoudiging toepassen. Enkel schrijven naar de actieve verloopt stroef en alle prompts resulteren in een zin met identieke semantiek, maar geschreven in de passieve stem. De twee chatbots voldoen niet aan de gebruiken een statische weergave en zijn daarmee niet personaliseerbaar. De twee chatbots zijn als enige in staat om structurele aanpassingen in de vorm van opsommingen of tabelstructuren toe te passen op de oorspronkelijke tekst.

\medspace

% Onderzoeksvraag 2
ChatGPT en Bing Chat bewijzen dat gepersonaliseerde ATS met evenwaardige kwaliteiten als gepersonaliseerde MTS mogelijk is, maar de online toepassingen ontbreken de nodige opmaakopties om een gepersonaliseerde leeservaring aan te kunnen bieden. Alsook kunnen deze tools geen wetenschappelijke artikelen op een intuïtieve manier inlezen, maar vereisen deze het manueel extraheren van de tekstinhoud om vervolgens deze in de chatbot chunk-by-chunk in te voeren. Daarnaast zijn deze toepassingen niet in staat om teksten naar de actieve stem te schrijven bij geen van de uitgeteste prompts. Passieve zinnen bij de verschillende prompts resulteren in een zin met dezelfde semantiek, maar opnieuw in de passieve vorm. Enkel indien expliciet aangegeven, houdt GPT-3 rekening met de doelgroep waar andere toepassingen niet ertoe in staat zijn.

\section{Vergelijkende studie}


De vergelijkende studie beoordeelt de uitvoer van de uitgeteste taallmodellen, beschreven in \ref{table:vergelijkende-studie-taalmodellen}, met een subjectieve en een objectieve benadering. Zo achterhaalt deze onderzoeksmethode welk taalmodel of LLM beter aansluit bij het aanbieden van gepersonaliseerde ATS voor scholieren met dyslexie in de derde graad van het middelbaar onderwijs. 

\medspace

Na de geteste taalmodellen bestaat de vereenvoudigde tekst bij T1, T2 en T3 uit meer zinnen dan in de oorspronkelijke tekst, zoals weergegeven in tabel \ref{table:resultaten-aantal-zinnen}. Daarnaast hebben de vereenvoudigde versies van T1, T2 en T3 gemiddeld minder woorden per zin dan het oorspronkelijke artikel. Alleen P3 van T4 slaagt erin om gemiddeld minder woorden per zin te gebruiken dan de oorspronkelijke en de referentieteksten van zowel leerlingen en de referentieteksten van leerkrachten, vergeleken met P1 en P2 die elk gemiddeld meer dan 19 woorden per zin gebruiken, zoals te zien is in figuren \ref{img:boxplot-min-max-avg-words-a1} en \ref{img:boxplot-min-max-avg-words-a2}. De FRE-scores van alle geteste taalmodellen en MTS-referentieteksten zijn niet significant hoger of lager dan die van het oorspronkelijke wetenschappelijke artikel, zoals weergegeven in figuren \ref{img:boxplot-fre-a1} en \ref{img:boxplot-fre-a2}. Evenzo zijn de FOG-scores ook niet significant hoger of lager bij de vereenvoudigde wetenschappelijke artikelen, zoals aangegeven in figuren \ref{img:boxplot-fog-a1} en \ref{img:boxplot-fog-a2}. T1, T2 en T3 gebruiken bij zowel A1 als A2 langere woorden vergeleken met de oorspronkelijke tekst, in tegenstelling tot alle T4 prompts die wel langere woorden wegwerken en zo een gelijk resultaat bekomen als de MTS referentieteksten. Deze verhouding wordt aangewezen in figuur \ref{img:violinplot-long-a1} en figuur \ref{img:violinplot-long-a2}. Daarnaast vervangen T1, T2 en T3 ook minder complexe woorden vergeleken met T4 en de MTS-referentietekst. Dit verschil wordt geïllustreerd in figuur \ref{img:violinplot-complex-a1} en figuur \ref{img:violinplot-complex-a2}.

\medspace

Taalmodellen T1, T2 en T3 zijn niet in staat om syntactische vereenvoudiging op een tekst toe te passen. Alleen T4 kan via P1, P2 en P3 de zinsyntax verlagen. Hoewel alle geteste taalmodellen in staat zijn om lexicale vereenvoudiging te realiseren, wordt de nauwkeurigheid van de doelgroepsinschatting in twijfel getrokken. De referentieteksten schatten de doelgroep correct in door bekend jargon niet aan te passen, maar wel nieuwe jargon aan te passen als er een beschikbaar synoniem is.  Daarnaast kan P1 van T4 ook de coherentie van een meegegeven paragraaf bevorderen, door onder meer omslachtige zinsstructuren aan te passen naar signaalwoorden. In tegenstelling tot T4 zijn T1, T2 en T3 niet in staat om het formaat van de uitvoer aan te passen. De uitvoer blijft een doorlopende tekst. In de referentietekst past één van de auteurs het formaat aan naar tabelvorm voor enkele paragrafen, waar de inhoud beter in tabelvorm kan gestructureerd worden. Enkel prompts P5 en P6 wordt er expliciet gevraagd om een formaatwijziging, anders geeft T4 vrijwel altijd een doorlopende tekst terug. Zonder de expliciete aanduiding, zoals uitgetest in sectie \ref{sec:requirementsanalyse}, is T4 niet in staat om deze structurele aanpassing bewust uit te voeren. 

\medspace

T4 heeft verschillende voordelen ten opzichte van T1, T2 en T3 bij gepersonaliseerde ATS voor wetenschappelijke artikelen. Ten eerste kan T4 het aantal zinnen verminderen, terwijl T1, T2 en T3 juist meer zinnen genereren in de vereenvoudigde tekst. T4 slaagt er ook in om gemiddeld minder woorden per zin te gebruiken dan de oorspronkelijke tekst, terwijl de andere modellen dit niet kunnen bereiken. Daarnaast weet T4 langere woorden weg te laten en complexe woorden beter te vervangen dan T1, T2 en T3. Dit resulteert in vereenvoudigde wetenschappelijke artikelen die vergelijkbaar zijn met de referentieteksten. T4 is ook uniek in zijn vermogen om de zinsyntax te verlagen en de coherentie van een paragraaf te verbeteren door signaalwoorden toe te passen. Een ander voordeel van T4 is dat het flexibeler is in het aanpassen van het formaat van de uitvoer. Terwijl T1, T2 en T3 alleen doorlopende tekst produceren, kan T4 ook het formaat wijzigen naar een tabelvorm of opsomming indien expliciet gevraagd. Deze voordelen maken T4 een aantrekkelijke keuze ten opzichte van T1, T2 en T3 bij gepersonaliseerde ATS van wetenschappelijke artikelen voor scholieren met dyslexie in de derde graad van het middelbaar onderwijs.

\section{Opbouw van het prototype}

% deelvraag: Hoe kan een intuïtieve en lokale webtoepassing worden ontwikkeld die zowel scholieren met dyslexie als docenten helpt bij het vereenvoudigen van wetenschappelijke artikelen met behoud van semantiek, jargon en zinsstructuren?

De ontwikkeling van een prototype voor een gepersonaliseerde ATS van wetenschappelijke artikelen heeft tot doel om ontwikkelaars inzicht te verschaffen in de mogelijkheden om een coherente en lokale webtoepassing te ontwikkelen. Dit prototype is bedoeld om zowel scholieren met dyslexie in de derde graad van het middelbaar onderwijs als docenten te ondersteunen bij het vereenvoudigen van wetenschappelijke artikelen, met behoud van de semantiek, jargon en zinsstructuren. Het prototype is uitgerust met functionaliteiten waarmee wetenschappelijke artikelen op consistente wijze kunnen worden ingeladen, waarna de gebruikers gepersonaliseerde ATS kunnen toepassen. Deze functionaliteiten zijn vergelijkbaar met andere geteste tools in de requirementsanalyse, maar gaan verder dan wat momenteel beschikbaar is in chatbots die geavanceerde logica gebruiken voor gepersonaliseerde ATS. Na evaluatie blijkt het prototype te voldoen aan de gespecificeerde functionaliteiten zoals vastgesteld in het Moscow-schema of tabel \ref{img:moscow-table}. Hiermee overtreft het prototype alle andere geteste tools in tabel \ref{table:overview-tools} op alle fronten. Eerdere belemmeringen, zoals het wijzigen van het formaat, waren enkel mogelijk via de commandline of via een chatbot. Het huidige prototype heeft echter intuïtieve handelingen geïmplementeerd om deze obstakels te overwinnen. Bovendien stelt het prototype gebruikers in staat om op basis van gegeven parameters automatisch personaliseerbare PDF- en Word-documenten te genereren. Zowel scholieren als docenten kunnen ook de opmaakopties van de toepassing aanpassen aan hun persoonlijke voorkeuren bij het lezen van wetenschappelijke artikelen. Een opmerkelijke eigenschap van het prototype is het vermogen om de semantiek van de oorspronkelijke wetenschappelijke artikelen te behouden. Dit wordt gerealiseerd door woordenlijsten te reproduceren na een zorgvuldige handmatige selectie van moeilijke woorden. Het prototype is lokaal op te zetten met behulp van Docker, hoewel voor het gebruik ervan een internetverbinding vereist is, net als bij andere tools in tabel \ref{table:overview-tools}.
