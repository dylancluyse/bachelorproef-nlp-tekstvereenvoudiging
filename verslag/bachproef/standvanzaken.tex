\chapter{\IfLanguageName{dutch}{Stand van zaken}{State of the art}}%
\label{ch:stand-van-zaken}

\section{Tekstvereenvoudiging}

Tekstvereenvoudiging is het proces waarin het technisch leesniveau en/of woordgebruik van een geschreven tekst wordt verminderd. Belangrijk hierbij is dat de vereenvoudiging geen effect mag hebben op de kerninhoud. Een complete vereenvoudiging van een tekst bestaat uit minstens drie transformaties \autocite{Siddharthan2014}. Daarnaast is tekstvereenvoudiging een taalbewerking dat geautomatiseerd kan worden. Tekstvereenvoudiging is namelijk een zijtak van natuurlijke taalverwerking.

\subsection{Natural Language Processing}

Natuurlijke taalverwerking of NLP is een brede term die zich richt op het verwerken en analyseren van menselijke taal door computers en andere technologieën. Het omvat verschillende technieken, zoals tekstanalyse, taalherkenning en -generatie, spraakherkenning en -synthese, en semantische analyse. Computers zijn ertoe in staat om op een menselijke manier te communiceren en begrijpen wat er wordt gezegd. Vooraleer het onderzoek zich verdiept in hoe teksten worden vereenvoudigd, moeten er eerst begrippen worden aangehaald die noodzakelijk zijn om de volgende fasen te kunnen uitleggen. \textcite{Sohom2019} haalt de volgende begrippen aan.

\begin{itemize}
	\item \textbf{Tokenisatie} splitst de stam of basisvorm van woorden in een tekst. Gebruikelijk zetten ontwikkelaars deze stap in om een woordenschat voor een taalmodel op te bouwen. Bij tokenisatie wordt er geen rekening gehouden met de betekenis achter ieder woord.
	\item \textbf{Lemmatiseren} in NLP bouwt verder op \textit{stemming}, maar de betekenis van ieder woord wordt in acht genomen. Voor het lemmatiseren bestaan er Nederlandstalige modellen, waaronder JohnSnow\footnote{https://nlp.johnsnowlabs.com/2020/05/03/lemma\_nl.html}. Bij \textbf{omgekeerd lemmatiseren} wordt er een afgeleide achterhaald vanuit de stam. Bijvoorbeeld voor het werkwoord 'zijn' zou dit 'is', 'was' of 'ben' zijn. Voor zelfstandige naamwoorden, zoals 'hond', is dit dan enkelvoud of meervoud.
	\item Bij een \textbf{parsing}-fase wordt er een label aan ieder woord of zinsdeel toegekend. Voorbeelden van labels zijn zelfstandig naamwoord, bijwoord, werkwoord, bijzin of stopwoord. Het herkennen van zinsdelen wordt \textit{chunking} genoemd. Parsing heeft een dubbelzinnigheidsprobleem, want een 'plant' staat niet gelijk aan de vervoeging van werkwoord 'planten'.
\end{itemize}

\section{De verschillende soorten tekstvereenvoudiging}

Inleiding op de soorten tekstvereenvoudiging. Tekstvereenvoudiging bestaat uit vier soorten transformaties: lexicale, syntactische en semantische vereenvoudiging en samenvatten.

\subsection{Lexicale vereenvoudiging}

% todo bron
Bij lexicale vereenvoudiging worden complexe woorden vervangen door eenvoudigere synoniemen. Bijvoorbeeld, het woord 'adhesief' kan worden vervangen door 'klevend'. De zinsstructuur verandert niet en er is garantie dat de kerninhoud en benadrukking hetzelfde blijft. Het doel van lexicale vereenvoudiging is om de moeilijkheidsgraad van de woordenschat in een zin of tekst te verlagen.

Een onderzoek van \textcite{Bulte2018} ging met dit concept aan de slag. Het resultaat van hun onderzoek was een \textit{pipeline} ontworpen om moeilijke woordenschat naar simpele synoniemen te vervangen. Eerst ging de tekstinhoud door een \textit{preprocessing}-fase, samen met het uitvoeren van WSE. Daarna werd de moeilijkheidsgraad van ieder token overlopen. De moeilijkheidsgraad is gebaseerd op hoe vaak een woord voorkomt in SONAR500\footnote{https://taalmaterialen.ivdnt.org/download/tstc-sonar-corpus/} een corpus met eenvoudige Nederlandstalige woorden. Synoniemen werden teruggevonden met Cornetto\footnote{https://github.com/emsrc/pycornetto}, een lexicale databank met Nederlandstalige woorden. Hiervoor gebruikten de onderzoekers een \textit{reverse lemmatization} fase. Lexicale vereenvoudiging is ingewikkeld wanneer er geen eenvoudigere synoniemen zijn. In dat geval blijft een moeilijk woord voor wat het is.

\subsection{Syntactische vereenvoudiging}

Syntactische vereenvoudiging transformeert de grammatica en zinsstructuur van een tekst om de complexiteit van een zin te verlagen. Bijvoorbeeld, twee afzonderlijke zinnen kunnen worden samengevoegd tot één eenvoudigere zin. Syntactische vereenvoudiging richt zich op het verminderen van complexe of onduidelijke zinsconstructies, terwijl de inhoud en betekenis van de tekst behouden blijft. Dergelijke transformaties zijn het vereenvoudigen van de syntax of door de zinnen korter te maken. Zinnen worden toegankelijker, zonder de kerninhoud of relevante inhoud te verliezen.

\subsection{Conceptuele vereenvoudiging}

Conceptuele vereenvoudiging lost dit probleem op. Theoretische kennis hierover is schaars, maar \textcite{Siddharthan2006} bestudeerde dit concept verder. Dit type vereenvoudiging betreft het opdelen van complexe concepten in eenvoudigere delen, het gebruik van duidelijke en bondige taal en het vermijden van technische jargon en abstracte uitdrukkingen. Het doel is om de inhoud begrijpelijker te maken, zonder dat hierbij de betekenis of nauwkeurigheid wordt aangetast. \textcite{Siddharthan2006} noemt deze transformatie een vorm van elaboratie of het uiteenzetten van een begrip.


\subsection{Tekstvereenvoudiging automatiseren}

Geautomatiseerde tekstvereenvoudiging is geen nieuwe zaak. Volgens het onderzoek van ... en ... waren de eerste aanpakken op geautomatiseerde tekstvereenvoudiging gebouwd op rule-based modellen. Deze modellen waren gericht op het bewerken van de syntax door onder meer zinnen te splitsen, te verwijderen of in een andere volgorde te plaatsen. Pas met recentere onderzoeken van ... en ... werd het duidelijk hoe lexicale en syntactische vereenvoudiging gecombineerd kon worden.

\subsection{Discourse edits}

\textit{Discourse}

\subsection{Combineren tot het geheel van tekstvereenvoudiging}

% Tekstvereenvoudiging omvat drie transformaties. 

Het onderzoek van \textcite{DeBelder2010} richt zich op tekstvereenvoudiging voor kinderen. De doelgroep ligt echter jonger dan deze casus, maar het onderzoek haalt aan hoe de onderzoekers een methode opzetten voor lexicale en syntactische vereenvoudiging.

\subsection{Samenvatten}

Lexicale, conceptuele en syntactische vereenvoudiging is er geen garantie dat de tekstinhoud korter zal worden. Eenvoudigere woordenschat 

\textcite{Wafaa2021} deed verder onderzoek op geautomatiseerd samenvatten.


\section{Voordelen van tekstvereenvoudiging}

\section{Struikelblokken}

\subsection{Evaluatie van de toepassing}

\subsection{Datasets}

\subsection{Meaning distortion}



\subsection{Paternalisme}
De doelstelling van assisterende software is om gelijke kansen te bieden aan iedereen. Zoals eerder vermeld, zorgt tekstvereenvoudiging voor een simpelere syntax en woordenschat in een tekst. Volgens \textcite{Niemeijer2010} zijn de ethische overwegingen die samenhangen met tekstvereenvoudiging via implicaties voor assistieve technologie niet gemakkelijk te scheiden van de technologie die wordt gebruikt om het resultaat te bereiken. Ontwikkelaars moeten, volgens deze auteur, rekening houden met de doelgroep waarvoor ze een toepassing maken.

% lenker bron
Het onderzoek van \textcite{Gooding2022} richtte zich op dit probleem. Ontwikkelaars moeten zich meer bewust worden van de behoeften en verwachtingen van de eindgebruiker bij het ontwikkelen van een tekstvereenvoudigingstoepassing. Haar onderzoek benadrukt de paternalistische en afhankelijke aard van assisterende technologieën. Tekstvereenvoudiging omvat drie transformaties, maar de moeilijkheidsgraad is niet statisch. Een adaptieve tekstvereenvoudigingstoepassing moet de eindgebruiker de keuze bieden om aan te passen wat vereenvoudigd wordt, afhankelijk van zijn of haar specifieke behoeften.

% Xu bron
Volgens \textcite{Punardeep2020}, maken de meeste AI-toepassingen voor tekstvereenvoudiging gebruik van \textit{black-box} modellen. Een \textit{black-box} model maakt het onmogelijk om transparant te zijn over waarom bepaalde transformaties worden uitgevoerd, bijvoorbeeld het vervangen van een woord door een eenvoudiger synoniem. Het model kan dus niet aangeven waarom het juist dat woord heeft vervangen door dat specifieke synoniem. Deze AI-toepassingen vallen onder de categorie van \textit{supervised learning} en het model leert handelingen uit de data waarop het is getraind. Dit is echter problematisch, aangezien \textcite{Xu2015} benadrukt dat veel toepassingen voor tekstvereenvoudiging geen rekening houden met de doelgroep waarvoor ze zijn ontwikkeld.

Om dit probleem op te lossen, is het belangrijk om de eindgebruiker, in dit geval scholieren met dyslexie in het derde graad middelbaar onderwijs, de keuze te geven. Zoals beschreven in \textcite{Gooding2022}, zijn er verschillende mogelijkheden. Bijvoorbeeld, de eindgebruiker moet de mogelijkheid hebben om te kiezen welke synoniemen de tekst lexicaal zullen aanpassen. Een alternatieve aanpak voor syntactische vereenvoudiging is om de scholier zelf zinnen te laten markeren die moeilijk te begrijpen zijn, zodat het systeem alleen de door de eindgebruiker aangegeven zinnen vereenvoudigt.


\subsection{Problemen bij lexicale vereenvoudiging}
\subsection{Problemen bij syntactische vereenvoudiging}
% Volgende zaken aanhalen:
\begin{itemize}
	\item Acroniemen
	\item Homoniemen
	\item Kerninhoud verliezen
	\item Ethisch aspect
	\item Word Sense Ambiguity
\end{itemize}


\section{Tekstvereenvoudigingssoftware in het onderwijs}

\section{Beschikbare tekstvereenvoudigingssoftware}

\section{Tekstvereenvoudigingspipeline opbouwen}

\section{Metrieken om de transformatie van tekstvereenvoudiging te beoordelen}

% Tip: Begin elk hoofdstuk met een paragraaf inleiding die beschrijft hoe
% dit hoofdstuk past binnen het geheel van de bachelorproef. Geef in het
% bijzonder aan wat de link is met het vorige en volgende hoofdstuk.

% Pas na deze inleidende paragraaf komt de eerste sectiehoofding.

% Dit hoofdstuk bevat je literatuurstudie. De inhoud gaat verder op de inleiding, maar zal het onderwerp van de bachelorproef *diepgaand* uitspitten. De bedoeling is dat de lezer na lezing van dit hoofdstuk helemaal op de hoogte is van de huidige stand van zaken (state-of-the-art) in het onderzoeksdomein. Iemand die niet vertrouwd is met het onderwerp, weet nu voldoende om de rest van het verhaal te kunnen volgen, zonder dat die er nog andere informatie moet over opzoeken \autocite{Pollefliet2011}.

% Je verwijst bij elke bewering die je doet, vakterm die je introduceert, enz.\ naar je bronnen. In \LaTeX{} kan dat met het commando \texttt{$\backslash${textcite\{\}}} of \texttt{$\backslash${autocite\{\}}}. Als argument van het commando geef je de ``sleutel'' van een ``record'' in een bibliografische databank in het Bib\LaTeX{}-formaat (een tekstbestand). Als je expliciet naar de auteur verwijst in de zin, gebruik je \texttt{$\backslash${}textcite\{\}}.
% Soms wil je de auteur niet expliciet vernoemen, dan gebruik je \texttt{$\backslash${}autocite\{\}}. In de volgende paragraaf een voorbeeld van elk.

% \textcite{Knuth1998} schreef een van de standaardwerken over sorteer- en zoekalgoritmen. Experten zijn het erover eens dat cloud computing een interessante opportuniteit vormen, zowel voor gebruikers als voor dienstverleners op vlak van informatietechnologie~\autocite{Creeger2009}.

\lipsum[7-20]
