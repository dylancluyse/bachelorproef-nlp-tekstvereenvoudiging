\chapter{\IfLanguageName{dutch}{Stand van zaken}{State of the art}}%
\label{ch:stand-van-zaken}

\section{Tekstvereenvoudiging}

Aan taal ontsnappen is vrijwel onmogelijk. Ongeacht de doelgroep komen mensen dagelijks in contact van nieuwsartikels, schoolopdrachten tot de ondertiteling van Netflix-series. De voorbije tien jaar zetten onderwijsinstellingen sterk in om gevarieerde bronnen in lessen op te nemen. De moeilijkheidsgraad van deze bronnen is echter niet onveranderlijk, want de nood aan verscheidenheid brengt ook de nood aan uitdagingen met zich mee. Volgens het leerplan moeten STEM-richtingen hun theoretische lessen op een zo toegankelijke manier aanbrengen, zodat iedereen mee is in het verhaal. 

Tekstvereenvoudiging is het proces waarin het technisch leesniveau of woordgebruik van een geschreven tekst wordt verminderd. De vereenvoudiging mag geen effect hebben op de kerninhoud. Als lezer moet de basisinformatie na een tekstvereenvoudiging nog steeds terug te vinden zijn.

\section{De verschillende soorten tekstvereenvoudiging}

\section{Voordelen van tekstvereenvoudiging}

\section{Struikelblokken}

Volgende zaken aanhalen:
\begin{itemize}
	\item Acroniemen
	\item Homoniemen
	\item Kerninhoud verliezen
	\item Ethisch aspect
	\item ...
\end{itemize}

\subsection{Acroniemen}

\subsection{Homoniemen}

\subsection{Kerninhoud verliezen}

\subsection{Ethisch aspect}

\cite{Gooding2022}

\section{Tekstvereenvoudigingssoftware in het onderwijs}

\section{Beschikbare tekstvereenvoudigingssoftware}

\section{Tekstvereenvoudigingspipeline opbouwen}

\section{Metrieken om de transformatie van tekstvereenvoudiging te beoordelen}

% Tip: Begin elk hoofdstuk met een paragraaf inleiding die beschrijft hoe
% dit hoofdstuk past binnen het geheel van de bachelorproef. Geef in het
% bijzonder aan wat de link is met het vorige en volgende hoofdstuk.

% Pas na deze inleidende paragraaf komt de eerste sectiehoofding.

% Dit hoofdstuk bevat je literatuurstudie. De inhoud gaat verder op de inleiding, maar zal het onderwerp van de bachelorproef *diepgaand* uitspitten. De bedoeling is dat de lezer na lezing van dit hoofdstuk helemaal op de hoogte is van de huidige stand van zaken (state-of-the-art) in het onderzoeksdomein. Iemand die niet vertrouwd is met het onderwerp, weet nu voldoende om de rest van het verhaal te kunnen volgen, zonder dat die er nog andere informatie moet over opzoeken \autocite{Pollefliet2011}.

% Je verwijst bij elke bewering die je doet, vakterm die je introduceert, enz.\ naar je bronnen. In \LaTeX{} kan dat met het commando \texttt{$\backslash${textcite\{\}}} of \texttt{$\backslash${autocite\{\}}}. Als argument van het commando geef je de ``sleutel'' van een ``record'' in een bibliografische databank in het Bib\LaTeX{}-formaat (een tekstbestand). Als je expliciet naar de auteur verwijst in de zin, gebruik je \texttt{$\backslash${}textcite\{\}}.
% Soms wil je de auteur niet expliciet vernoemen, dan gebruik je \texttt{$\backslash${}autocite\{\}}. In de volgende paragraaf een voorbeeld van elk.

% \textcite{Knuth1998} schreef een van de standaardwerken over sorteer- en zoekalgoritmen. Experten zijn het erover eens dat cloud computing een interessante opportuniteit vormen, zowel voor gebruikers als voor dienstverleners op vlak van informatietechnologie~\autocite{Creeger2009}.

\lipsum[7-20]
