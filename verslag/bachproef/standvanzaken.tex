\chapter{\IfLanguageName{dutch}{Stand van zaken}{State of the art}}%
\label{ch:stand-van-zaken}

\section{Tekstvereenvoudiging}

% TODO
Inleiding

\subsubsection{Tekstvereenvoudiging}

Tekstvereenvoudiging is het proces waarin het technisch leesniveau of woordgebruik van een geschreven tekst wordt verminderd. De vereenvoudiging mag geen effect hebben op de kerninhoud. Als lezer moet de basisinformatie na een tekstvereenvoudiging nog steeds terug te vinden zijn. ... haalt een definitie van tekstvereenvouding aan, namelijk "het proces dat de syntactische of lexicale vereenvoudiging van een tekst omvat en met een resultaat in de vorm van een samenhangende tekst". 

% todo bron aanvullen

\subsubsection{Gebruiksdomeinen}

Tekstvereenvoudiging draagt bij tot het 

\subsubsection{Natuurlijke taalverwerking}

Natuurlijke taalverwerking is een brede term die zich richt op het verwerken en analyseren van menselijke taal door computers en andere technologieën. Het omvat diverse technieken, zoals tekstanalyse, taalherkenning en -generatie, spraakherkenning en -synthese, en gevoelensanalyse. Het doel is om computers in staat te stellen om op menselijke wijze te communiceren en begrijpen wat er gezegd wordt, en om zo taal te verrijken en te verbeteren.

\section{De verschillende soorten tekstvereenvoudiging}

Inleiding. Tekstvereenvoudiging is in drie vormen terug te vinden: lexicale, syntactische en semantische vereenvoudiging.

\subsection{Lexicale vereenvoudiging}

Bij lexicale vereenvoudiging worden complexe woorden vervangen door eenvoudigere synoniemen. Bijvoorbeeld, het woord "adhesief" kan worden vervangen door "klevend". De zinsstructuur verandert niet en zo is er garantie dat de kerninhoud en nuancering identiek blijft. Het doel van lexicale vereenvoudiging is om de complexiteit van de woordenschat te verlagen.

Antwoord op waar het wordt ingezet?

\subsection{Syntactische vereenvoudiging}

Syntactische vereenvoudiging transformeert de grammatica en zinsstructuur van een tekst om de complexiteit van een zin te verlagen. Bijvoorbeeld, twee afzonderlijke zinnen kunnen worden samengevoegd tot één eenvoudigere zin. Syntactische vereenvoudiging richt zich op het verminderen van complexe of onduidelijke zinsconstructies, terwijl de inhoud en betekenis van de tekst behouden blijft. Kortom worden teksten toegankelijker, zonder de kerninhoud of relevante inhoud te verwerpen.

Antwoord op waar het wordt ingezet?

\subsection{Semantische vereenvoudiging}

\subsection{Samenvatten}



\section{Voordelen van tekstvereenvoudiging}

\section{Struikelblokken}

Volgende zaken aanhalen:
\begin{itemize}
	\item Acroniemen
	\item Homoniemen
	\item Kerninhoud verliezen
	\item Ethisch aspect
	\item ...
\end{itemize}

\subsection{Acroniemen}

\subsection{Homoniemen}

\subsection{Kerninhoud verliezen}

\subsection{Ethisch aspect}

\cite{Gooding2022}

\section{Tekstvereenvoudigingssoftware in het onderwijs}

\section{Beschikbare tekstvereenvoudigingssoftware}

\section{Tekstvereenvoudigingspipeline opbouwen}

\section{Metrieken om de transformatie van tekstvereenvoudiging te beoordelen}

% Tip: Begin elk hoofdstuk met een paragraaf inleiding die beschrijft hoe
% dit hoofdstuk past binnen het geheel van de bachelorproef. Geef in het
% bijzonder aan wat de link is met het vorige en volgende hoofdstuk.

% Pas na deze inleidende paragraaf komt de eerste sectiehoofding.

% Dit hoofdstuk bevat je literatuurstudie. De inhoud gaat verder op de inleiding, maar zal het onderwerp van de bachelorproef *diepgaand* uitspitten. De bedoeling is dat de lezer na lezing van dit hoofdstuk helemaal op de hoogte is van de huidige stand van zaken (state-of-the-art) in het onderzoeksdomein. Iemand die niet vertrouwd is met het onderwerp, weet nu voldoende om de rest van het verhaal te kunnen volgen, zonder dat die er nog andere informatie moet over opzoeken \autocite{Pollefliet2011}.

% Je verwijst bij elke bewering die je doet, vakterm die je introduceert, enz.\ naar je bronnen. In \LaTeX{} kan dat met het commando \texttt{$\backslash${textcite\{\}}} of \texttt{$\backslash${autocite\{\}}}. Als argument van het commando geef je de ``sleutel'' van een ``record'' in een bibliografische databank in het Bib\LaTeX{}-formaat (een tekstbestand). Als je expliciet naar de auteur verwijst in de zin, gebruik je \texttt{$\backslash${}textcite\{\}}.
% Soms wil je de auteur niet expliciet vernoemen, dan gebruik je \texttt{$\backslash${}autocite\{\}}. In de volgende paragraaf een voorbeeld van elk.

% \textcite{Knuth1998} schreef een van de standaardwerken over sorteer- en zoekalgoritmen. Experten zijn het erover eens dat cloud computing een interessante opportuniteit vormen, zowel voor gebruikers als voor dienstverleners op vlak van informatietechnologie~\autocite{Creeger2009}.

\lipsum[7-20]
