\chapter{\IfLanguageName{dutch}{Stand van zaken}{State of the art}}%
\label{ch:stand-van-zaken}

\section{Tekstvereenvoudiging}

% TODO
Inleiding

\subsection{Natural Language Processing}

Natuurlijke taalverwerking of NLP is een brede term die zich richt op het verwerken en analyseren van menselijke taal door computers en andere technologieën. Het omvat verschillende technieken, zoals tekstanalyse, taalherkenning en -generatie, spraakherkenning en -synthese, en semantische analyse. Computers zijn ertoe in staat om op een menselijke manier te communiceren en begrijpen wat er wordt gezegd. Vooraleer het onderzoek zich verdiept in hoe teksten worden vereenvoudigd, moeten er eerst begrippen worden aangehaald die noodzakelijk zijn om de volgende fasen te kunnen uitleggen.

\begin{itemize}
	\item Tokenisatie
	\item Lemmatizatie en omgekeerde lemmatizatie
	\item \textit{Parsing}
\end{itemize}

\subsubsection{Tokenisatie}

% https://dair.ai/notebooks/nlp/2020/03/19/nlp_basics_tokenization_segmentation.html

Om de basisvorm of stam van woorden in een tekst te gaan opsplitsen, wordt tokenisatie ingezet. Gebruikelijk wordt deze stap door ontwikkelaars ingezet om een woordenschat voor een taalmodel op te bouwen. Bij tokenisatie wordt er geen rekening gehouden met de betekenis achter ieder woord.


\subsubsection{Lemmatiseren en parsen}

https://www.flex4medics.nl/5445-2/

Lemmatiseren in NLP bouwt verder op \textit{stemming}, maar de betekenis van ieder woord wordt in acht genomen. Voor het lemmatiseren bestaan er Nederlandstalige modellen, waaronder JohnSnow\footnote{https://nlp.johnsnowlabs.com/2020/05/03/lemma\_nl.html}. Bij omgekeerd lemmatiseren wordt er een afgeleide achterhaald vanuit de stam. Bijvoorbeeld voor het werkwoord 'zijn' zou dit 'is', 'was' of 'ben' zijn. Voor zelfstandige naamwoorden, zoals 'hond', is dit dan enkelvoud of meervoud.

Bij een parsing-fase wordt er een label aan ieder woord of zinsdeel toegekend. Voorbeelden van labels zijn zelfstandig naamwoord, bijwoord, werkwoord, bijzin of stopwoord. Het herkennen van zinsdelen wordt \textit{chunking} genoemd. Parsing heeft een dubbelzinnigheidsprobleem, want een 'plant' staat niet gelijk aan de vervoeging van werkwoord 'planten'. 

\subsubsection{Word Sense Ambiguation}

Book: Word Sense Disambiguation: Algorithms and Applications



\subsubsection{Tekstvereenvoudiging}

Tekstvereenvoudiging is het proces waarin het technisch leesniveau en/of woordgebruik van een geschreven tekst wordt verminderd. 

Belangrijk hierbij is dat de vereenvoudiging geen effect mag hebben op de kerninhoud. 



% Als lezer moet de basisinformatie na een tekstvereenvoudiging nog steeds terug te vinden zijn. ... haalt een definitie van tekstvereenvouding aan, namelijk "het proces dat de syntactische of lexicale vereenvoudiging van een tekst omvat en een samenhangende tekst als resultaat heeft". 

% todo bron aanvullen

\subsubsection{Gebruiksdomeinen}

\section{De verschillende soorten tekstvereenvoudiging}

Inleiding op de soorten tekstvereenvoudiging. Tekstvereenvoudiging is in drie vormen terug te vinden: lexicale, syntactische en semantische vereenvoudiging.

\subsection{Lexicale vereenvoudiging}

Bij lexicale vereenvoudiging worden complexe woorden vervangen door eenvoudigere synoniemen. Bijvoorbeeld, het woord "adhesief" kan worden vervangen door "klevend". De zinsstructuur verandert niet en zo is er garantie dat de kerninhoud en nuancering identiek blijft. Het doel van lexicale vereenvoudiging is om de complexiteit van de woordenschat te verlagen.

Een onderzoek van de KU Leuven (bron) ging met dit concept aan de slag. Het resultaat van hun onderzoek was een pipeline ontworpen om complexe woorden naar simpele synoniemen te vervangen. Eerst ging de tekstinhoud door een \textit{preprocessing}-fase, samen met het uitvoeren van WSE. Daarna werd de moeilijkheidsgraad van ieder token overlopen. De moeilijkheidsgraad is gebaseerd op hoe vaak een woord voorkomt in SONAR500\footnote{https://taalmaterialen.ivdnt.org/download/tstc-sonar-corpus/} een corpus met eenvoudige Nederlandstalige woorden. Synoniemen werden teruggevonden met Cornetto\footnote{https://github.com/emsrc/pycornetto}, een lexicale databank met Nederlandstalige woorden. Hiervoor gebruikten de onderzoekers een \textit{reverse lemmatization} fase.

\subsection{Syntactische vereenvoudiging}

Syntactische vereenvoudiging transformeert de grammatica en zinsstructuur van een tekst om de complexiteit van een zin te verlagen. Bijvoorbeeld, twee afzonderlijke zinnen kunnen worden samengevoegd tot één eenvoudigere zin. Syntactische vereenvoudiging richt zich op het verminderen van complexe of onduidelijke zinsconstructies, terwijl de inhoud en betekenis van de tekst behouden blijft. Kortom worden teksten toegankelijker, zonder de kerninhoud of relevante inhoud te verwerpen.

Antwoord op waar het wordt ingezet?

\subsection{Semantische vereenvoudiging}

\subsection{Samenvatten}



\section{Voordelen van tekstvereenvoudiging}

\section{Struikelblokken}

Volgende zaken aanhalen:
\begin{itemize}
	\item Acroniemen
	\item Homoniemen
	\item Kerninhoud verliezen
	\item Ethisch aspect
	\item ...
\end{itemize}

\subsection{Acroniemen}

\subsection{Homoniemen}

\subsection{Kerninhoud verliezen}

\subsection{Ethisch aspect}

\cite{Gooding2022}

\section{Tekstvereenvoudigingssoftware in het onderwijs}

\section{Beschikbare tekstvereenvoudigingssoftware}

\section{Tekstvereenvoudigingspipeline opbouwen}

\section{Metrieken om de transformatie van tekstvereenvoudiging te beoordelen}

% Tip: Begin elk hoofdstuk met een paragraaf inleiding die beschrijft hoe
% dit hoofdstuk past binnen het geheel van de bachelorproef. Geef in het
% bijzonder aan wat de link is met het vorige en volgende hoofdstuk.

% Pas na deze inleidende paragraaf komt de eerste sectiehoofding.

% Dit hoofdstuk bevat je literatuurstudie. De inhoud gaat verder op de inleiding, maar zal het onderwerp van de bachelorproef *diepgaand* uitspitten. De bedoeling is dat de lezer na lezing van dit hoofdstuk helemaal op de hoogte is van de huidige stand van zaken (state-of-the-art) in het onderzoeksdomein. Iemand die niet vertrouwd is met het onderwerp, weet nu voldoende om de rest van het verhaal te kunnen volgen, zonder dat die er nog andere informatie moet over opzoeken \autocite{Pollefliet2011}.

% Je verwijst bij elke bewering die je doet, vakterm die je introduceert, enz.\ naar je bronnen. In \LaTeX{} kan dat met het commando \texttt{$\backslash${textcite\{\}}} of \texttt{$\backslash${autocite\{\}}}. Als argument van het commando geef je de ``sleutel'' van een ``record'' in een bibliografische databank in het Bib\LaTeX{}-formaat (een tekstbestand). Als je expliciet naar de auteur verwijst in de zin, gebruik je \texttt{$\backslash${}textcite\{\}}.
% Soms wil je de auteur niet expliciet vernoemen, dan gebruik je \texttt{$\backslash${}autocite\{\}}. In de volgende paragraaf een voorbeeld van elk.

% \textcite{Knuth1998} schreef een van de standaardwerken over sorteer- en zoekalgoritmen. Experten zijn het erover eens dat cloud computing een interessante opportuniteit vormen, zowel voor gebruikers als voor dienstverleners op vlak van informatietechnologie~\autocite{Creeger2009}.

\lipsum[7-20]
