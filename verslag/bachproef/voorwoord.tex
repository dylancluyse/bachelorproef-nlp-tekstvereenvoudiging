%%=============================================================================
%% Voorwoord
%%=============================================================================

\chapter*{\IfLanguageName{dutch}{Woord vooraf}{Preface}}%
\label{ch:voorwoord}

%% TODO:
%% Het voorwoord is het enige deel van de bachelorproef waar je vanuit je
%% eigen standpunt (``ik-vorm'') mag schrijven. Je kan hier bv. motiveren
%% waarom jij het onderwerp wil bespreken.
%% Vergeet ook niet te bedanken wie je geholpen/gesteund/... heeft

\lipsum[1-2]

% todo aanvullen naargelang wat het hart zegt

% Deze scriptie zou niet dezelfde hoogte behaald kunnen hebben zonder de volgende peers. Deze wil ik nadrukkelijk bedanken voor hun bijdrage aan dit onderzoek.

% Lena De Mol leverde een uitmuntende bijdrage als promotor voor het onderzoek. Met een affiniteit voor technologie, taal en onderwijs was zij een \textit{match-made-in-heaven} om dit onderzoek te begeleiden.

% Johan Decorte voor de input en bijdrage omtrent Machine Learning. Iedere wekelijkse sessie bracht mij een nieuwe inkijk hoe ik het technologische component moest aanpakken. Dit bracht de ambitie naar meer met zich mee. De begeleiding van Jana Van Damme verbreedde mijn horizon over het logopedisch vakdomein. Zowel Johan als Jana maakten tijd vrij voor dit project, ondanks hun druk schema.

% Emmanuel Vercruysse en Johannes Nijs van Hogeschool Vives en Sofie Smet en Sophie Vyncke van Arteveldehogeschool voor hun bijdrage van de referentieteksten aan het experiment. Emmanuel en Sophie waren aangenaam om tijd vrij te maken voor een oud-student, alsook Johannes en Sofie om deze taak op hen te nemen tijdens de paasvakantie.

% Lobke als steunpunt en oprecht goede vriend.