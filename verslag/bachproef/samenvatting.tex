%%=============================================================================
%% Samenvatting
%%=============================================================================

% TODO: De "abstract" of samenvatting is een kernachtige (~ 1 blz. voor een
% thesis) synthese van het document.
%
% Een goede abstract biedt een kernachtig antwoord op volgende vragen:
%
% 1. Waarover gaat de bachelorproef?
% 2. Waarom heb je er over geschreven?
% 3. Hoe heb je het onderzoek uitgevoerd?
% 4. Wat waren de resultaten? Wat blijkt uit je onderzoek?
% 5. Wat betekenen je resultaten? Wat is de relevantie voor het werkveld?
%
% Daarom bestaat een abstract uit volgende componenten:
%
% - inleiding + kaderen thema
% - probleemstelling
% - (centrale) onderzoeksvraag
% - onderzoeksdoelstelling
% - methodologie
% - resultaten (beperk tot de belangrijkste, relevant voor de onderzoeksvraag)
% - conclusies, aanbevelingen, beperkingen
%
% LET OP! Een samenvatting is GEEN voorwoord!

%%---------- Nederlandse samenvatting -----------------------------------------
%
% TODO: Als je je bachelorproef in het Engels schrijft, moet je eerst een
% Nederlandse samenvatting invoegen. Haal daarvoor onderstaande code uit
% commentaar.
% Wie zijn bachelorproef in het Nederlands schrijft, kan dit negeren, de inhoud
% wordt niet in het document ingevoegd.

\IfLanguageName{english}{%
\selectlanguage{dutch}
\chapter*{Samenvatting}
\lipsum[1-4]
\selectlanguage{english}
}{}

%%---------- Samenvatting -----------------------------------------------------
% De samenvatting in de hoofdtaal van het document

\chapter*{\IfLanguageName{dutch}{Samenvatting}{Abstract}}

Ingewikkelde woordenschat en zinsbouw hinderen scholieren met dyslexie in het derde graad middelbaar onderwijs bij het lezen van wetenschappelijke artikelen. Adaptieve tekstvereenvoudiging helpt deze scholieren bij hun lees- en verwerkingssnelheid. Daarnaast kan artificiële intelligentie (AI) dit proces automatiseren om de werkdruk bij leraren en scholieren te verminderen. Dit onderzoek achterhaalt met welke technologische en logopedische aspecten AI-ontwikkelaars rekening moeten houden bij de ontwikkeling van een AI-toepassing voor adaptieve en geautomatiseerde tekstvereenvoudiging. Hiervoor is de volgende onderzoeksvraag opgesteld: "Hoe kan een wetenschappelijk artikel automatisch worden vereenvoudigd, gericht op de unieke noden van scholieren met dyslexie in het derde graad middelbaar onderwijs?". Een vergelijkende studie beantwoordt deze onderzoeksvraag en is uitgevoerd met bestaande toepassingen en een prototype voor adaptieve en geautomatiseerde tekstvereenvoudiging. Uit de vergelijkende studie blijkt dat toepassingen om wetenschappelijke artikelen te vereenvoudigen, gemaakt zijn voor een centrale doelgroep en geen rekening houden met de unieke noden van een scholier met dyslexie in het derde graad middelbaar onderwijs. Adaptieve software voor geautomatiseerde tekstvereenvoudiging is mogelijk, maar ontwikkelaars moeten meer inzetten op de unieke noden van deze scholieren. 
