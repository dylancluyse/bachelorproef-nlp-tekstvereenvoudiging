%===============================================================================
% LaTeX sjabloon voor de bachelorproef toegepaste informatica aan HOGENT
% Meer info op https://github.com/HoGentTIN/latex-hogent-report
%===============================================================================

\documentclass[dutch,dit,thesis]{hogentreport}

% TODO:
% - If necessary, replace the option `dit`' with your own department!
%   Valid entries are dbo, dbt, dgz, dit, dlo, dog, dsa, soa
% - If you write your thesis in English (remark: only possible after getting
%   explicit approval!), remove the option "dutch," or replace with "english".

\usepackage{lipsum} % For blind text, can be removed after adding actual content
\usepackage{array}
\usepackage{listings}

%% Pictures to include in the text can be put in the graphics/ folder
\graphicspath{{graphics/}}

%% For source code highlighting, requires pygments to be installed
%% Compile with the -shell-escape flag!
\usepackage[section]{minted}
\usemintedstyle{solarized-light}
\definecolor{bg}{RGB}{253,246,227} %% Set the background color of the codeframe

%% Change this line to edit the line numbering style:
\renewcommand{\theFancyVerbLine}{\ttfamily\scriptsize\arabic{FancyVerbLine}}

%% Macro definition to load external java source files with \javacode{filename}:
\newmintedfile[javacode]{java}{
    bgcolor=bg,
    fontfamily=tt,
    linenos=true,
    numberblanklines=true,
    numbersep=5pt,
    gobble=0,
    framesep=2mm,
    funcnamehighlighting=true,
    tabsize=4,
    obeytabs=false,
    breaklines=true,
    mathescape=false
    samepage=false,
    showspaces=false,
    showtabs =false,
    texcl=false,
}

% Other packages not already included can be imported here

%%---------- Document metadata -------------------------------------------------
% TODO: Replace this with your own information
\author{Dylan Cluyse}
\supervisor{Mevr. L. De Mol}
\cosupervisor{J. Decorte; J. Van Damme;}
\title[De opbouw van een LLM-gedreven prototype voor geautomatiseerde en gepersonaliseerde tekstvereenvoudiging.]%
    {Scholieren met dyslexie van de derde graad middelbaar onderwijs ondersteunen bij het intensief lezen van wetenschappelijke artikelen via geautomatiseerde en gepersonaliseerde tekstvereenvoudiging}
\academicyear{\advance\year by -1 \the\year--\advance\year by 1 \the\year}
\examperiod{1}
\degreesought{\IfLanguageName{dutch}{Professionele bachelor in de toegepaste informatica}{Bachelor of applied computer science}}
\partialthesis{false} %% To display 'in partial fulfilment'
%\institution{Internshipcompany BVBA.}

%% Add global exceptions to the hyphenation here
\hyphenation{back-slash}

%% The bibliography (style and settings are  found in hogentthesis.cls)
\addbibresource{bachproef.bib}            %% Bibliography file
\addbibresource{../voorstel/voorstel.bib} %% Bibliography research proposal
\defbibheading{bibempty}{}

%% Prevent empty pages for right-handed chapter starts in twoside mode
\renewcommand{\cleardoublepage}{\clearpage}

\renewcommand{\arraystretch}{1.2}

%% Content starts here.
\begin{document}

%---------- Front matter -------------------------------------------------------

\frontmatter

\hypersetup{pageanchor=false} %% Disable page numbering references
%% Render a Dutch outer title page if the main language is English
\IfLanguageName{english}{%
    %% If necessary, information can be changed here
    \degreesought{Professionele Bachelor toegepaste informatica}%
    \begin{otherlanguage}{dutch}%
       \maketitle%
    \end{otherlanguage}%
}{}

%% Generates title page content
\maketitle
\hypersetup{pageanchor=true}

%%=============================================================================
%% Voorwoord
%%=============================================================================

\chapter*{\IfLanguageName{dutch}{Woord vooraf}{Preface}}%
\label{ch:voorwoord}

%% TODO:
%% Het voorwoord is het enige deel van de bachelorproef waar je vanuit je
%% eigen standpunt (``ik-vorm'') mag schrijven. Je kan hier bv. motiveren
%% waarom jij het onderwerp wil bespreken.
%% Vergeet ook niet te bedanken wie je geholpen/gesteund/... heeft

\lipsum[1-2]

% todo aanvullen naargelang wat het hart zegt

% Dit onderzoek zou nooit dezelfde hoogtes kunnen gehaald hebben zonder de volgende betrokkenen.

% Lena De Mol leverde een uitmuntende bijdrage als promotor voor het onderzoek. Met een affiniteit voor technologie, taal en onderwijs was zij een \textit{match-made-in-heaven} om dit onderzoek te begeleiden.

% Johan Decorte voor de input en bijdrage omtrent Machine Learning. Iedere wekelijkse sessie bracht mij een nieuwe inkijk hoe ik het technologische component moest aanpakken. Dit bracht de ambitie naar meer met zich mee. De begeleiding van Jana Van Damme verbreedde mijn horizon over het logopedisch vakdomein. Zowel Johan als Jana maakten tijd vrij voor dit project, ondanks hun druk schema.

% Emmanuel Vercruysse, ..., Sofie Smet en Sophie Vyncke voor hun bijdrage aan het experiment. Emmanuel en Sophie waren aangenaam om tijd vrij te maken voor een oud-student. 

% Lobke voor de hartverwarmende steun.
%%=============================================================================
%% Samenvatting
%%=============================================================================

% TODO: De "abstract" of samenvatting is een kernachtige (~ 1 blz. voor een
% thesis) synthese van het document.
%
% Een goede abstract biedt een kernachtig antwoord op volgende vragen:
%
% 1. Waarover gaat de bachelorproef?
% 2. Waarom heb je er over geschreven?
% 3. Hoe heb je het onderzoek uitgevoerd?
% 4. Wat waren de resultaten? Wat blijkt uit je onderzoek?
% 5. Wat betekenen je resultaten? Wat is de relevantie voor het werkveld?
%
% Daarom bestaat een abstract uit volgende componenten:
%
% - inleiding + kaderen thema
% - probleemstelling
% - (centrale) onderzoeksvraag
% - onderzoeksdoelstelling
% - methodologie
% - resultaten (beperk tot de belangrijkste, relevant voor de onderzoeksvraag)
% - conclusies, aanbevelingen, beperkingen
%
% LET OP! Een samenvatting is GEEN voorwoord!

%%---------- Nederlandse samenvatting -----------------------------------------
%
% TODO: Als je je bachelorproef in het Engels schrijft, moet je eerst een
% Nederlandse samenvatting invoegen. Haal daarvoor onderstaande code uit
% commentaar.
% Wie zijn bachelorproef in het Nederlands schrijft, kan dit negeren, de inhoud
% wordt niet in het document ingevoegd.

\IfLanguageName{english}{%
\selectlanguage{dutch}
\chapter*{Samenvatting}
\lipsum[1-4]
\selectlanguage{english}
}{}

%%---------- Samenvatting -----------------------------------------------------
% De samenvatting in de hoofdtaal van het document

\chapter*{\IfLanguageName{dutch}{Samenvatting}{Abstract}}

Ingewikkelde woordenschat en zinsbouw hinderen scholieren met dyslexie in het derde graad middelbaar onderwijs bij het lezen van wetenschappelijke artikelen. Adaptieve tekstvereenvoudiging helpt deze scholieren bij hun lees- en verwerkingssnelheid. Daarnaast kan artificiële intelligentie (AI) dit proces automatiseren om de werkdruk bij leraren en scholieren te verminderen. Dit onderzoek achterhaalt met welke technologische en logopedische aspecten AI-ontwikkelaars rekening moeten houden bij de ontwikkeling van een AI-toepassing voor adaptieve en geautomatiseerde tekstvereenvoudiging. Hiervoor is de volgende onderzoeksvraag opgesteld: "Hoe kan een wetenschappelijk artikel automatisch worden vereenvoudigd, gericht op de unieke noden van scholieren met dyslexie in het derde graad middelbaar onderwijs?". Een vergelijkende studie beantwoordt deze onderzoeksvraag en is uitgevoerd met bestaande toepassingen en een prototype voor adaptieve en geautomatiseerde tekstvereenvoudiging. Uit de vergelijkende studie blijkt dat toepassingen om wetenschappelijke artikelen te vereenvoudigen, gemaakt zijn voor een centrale doelgroep en geen rekening houden met de unieke noden van een scholier met dyslexie in het derde graad middelbaar onderwijs. Adaptieve software voor geautomatiseerde tekstvereenvoudiging is mogelijk, maar ontwikkelaars moeten meer inzetten op de unieke noden van deze scholieren. 


%---------- Inhoud, lijst figuren, ... -----------------------------------------

\tableofcontents

% In a list of figures, the complete caption will be included. To prevent this,
% ALWAYS add a short description in the caption!
%
%  \caption[short description]{elaborate description}
%
% If you do, only the short description will be used in the list of figures

\listoffigures

% If you included tables and/or source code listings, uncomment the appropriate
% lines.
\listoftables
\listoflistings

% Als je een lijst van afkortingen of termen wil toevoegen, dan hoort die
% hier thuis. Gebruik bijvoorbeeld de ``glossaries'' package.
% https://www.overleaf.com/learn/latex/Glossaries
\glossary{}

%---------- Kern ---------------------------------------------------------------

\mainmatter{}

% De eerste hoofdstukken van een bachelorproef zijn meestal een inleiding op
% het onderwerp, literatuurstudie en verantwoording methodologie.
% Aarzel niet om een meer beschrijvende titel aan deze hoofdstukken te geven of
% om bijvoorbeeld de inleiding en/of stand van zaken over meerdere hoofdstukken
% te verspreiden!

%%=============================================================================
%% Inleiding
%%=============================================================================

\chapter{\IfLanguageName{dutch}{Inleiding}{Introduction}}%
\label{ch:inleiding}

% De inleiding moet de lezer net genoeg informatie verschaffen om het onderwerp te begrijpen en in te zien waarom de onderzoeksvraag de moeite waard is om te onderzoeken. In de inleiding ga je literatuurverwijzingen beperken, zodat de tekst vlot leesbaar blijft. Je kan de inleiding verder onderverdelen in secties als dit de tekst verduidelijkt. Zaken die aan bod kunnen komen in de inleiding~\autocite{Pollefliet2011}:

%\begin{itemize}
%  \item context, achtergrond
%  \item afbakenen van het onderwerp
%  \item verantwoording van het onderwerp, methodologie
%  \item probleemstelling
%  \item onderzoeksdoelstelling
%  \item onderzoeksvraag
%  \item \ldots
%\end{itemize}

Het middelbaar onderwijs staat op springen. Dagelijks sneuvelen leerkrachten en leerlingen van het middelbaar onderwijs onder de te harde werkdruk. Daarnaast is taal vrijwel onmogelijk om aan te ontsnappen. Dagelijks komen mensen in aanraking met taal, van Nederlandse nieuwsartikelen tot de ondertiteling van Koreaanse Netflix-series, ongeacht de doelgroep. Lerarenopleidingen richt zich de afgelopen tien jaar sterk op het gebruik van gevarieerde bronnen in lessen. De moeilijkheidsgraad van deze bronnen verandert echter niet, want de noodzaak aan verscheidenheid brengt ook de noodzaak aan uitdagingen met zich mee. STEM-leerkrachten in een derde graad middelbaar onderwijs moeten volgens het leerplan van zowel het katholiek\footnote{https://pro.katholiekonderwijs.vlaanderen/basisoptie-stem/ondersteunend-materiaal} als het gemeenschapsonderwijs\footnote{https://g-o.be/stem/} hun theorielessen op een toegankelijke manier aanbieden, zodat iedereen betrokken is bij het verhaal.

\newline

Met een jaarlijks budget van 32 miljoen is België een pionier \autocite{Crevits2022} in het vakgebied kunstmatige intelligentie (AI) op de werkvloer.  Zo stampte de Vlaamse overheid verschillende AI-projecten uit de grond, om Vlaamse AI-ontwikkelingen te ondersteunen en inspireren. Het amai!-project\footnote{https://amai.vlaanderen/}  brengt AI-softwarebedrijven samen uit verschillende domeinen. Dit project leidt tot het ontstaan van AI-toepassingen die processen automatiseren om de werkdruk te verminderen, zoals binnen het onderwijs \textit{real-time} ondertiteling en een taalassistent voor leerkrachten in meertalige klasgroepen.

\section{\IfLanguageName{dutch}{Probleemstelling}{Problem Statement}}%
\label{sec:probleemstelling}

% Uit je probleemstelling moet duidelijk zijn dat je onderzoek een meerwaarde heeft voor een concrete doelgroep. De doelgroep moet goed gedefinieerd en afgelijnd zijn. Doelgroepen als ``bedrijven,'' ``KMO's'', systeembeheerders, enz.~zijn nog te vaag. Als je een lijstje kan maken van de personen/organisaties die een meerwaarde zullen vinden in deze bachelorproef (dit is eigenlijk je steekproefkader), dan is dat een indicatie dat de doelgroep goed gedefinieerd is. Dit kan een enkel bedrijf zijn of zelfs één persoon (je co-promotor/opdrachtgever).

% Uitleg over dyslexie.

Volgens \textcite{Ghesquiere2018} heeft ongeveer 5 tot 9 \% van de Nederlandstalige bevolking\footnote{Deze schatting is gebaseerd op de Vlaamse en Nederlandse bevolking.} te maken met dyslexie. Verder benadrukt de studie van \textcite{Lissens2020} dat de impact van leerstoornissen niet stopt na het middelbaar onderwijs. Scholieren met dyslexie in het middelbaar onderwijs krijgen te maken met unieke uitdagingen. Gelukkig worden ze niet aan hun lot overgelaten en kunnen ze rekenen op ondersteuning van coaches en beschikbare hulpmiddelen om hun achterstand te beperken. Het leerplan voor STEM-vakken stimuleert het gebruik van wetenschappelijke artikelen, maar houdt niet altijd rekening met de moeilijkheidsgraad ervan. De complexe woordenschat en zinsopbouw in deze artikelen vormen een barrière voor de begrijpelijkheid van een tekst, waardoor de scholieren de kerninhoud moeilijk kunnen doorgronden. Een oplossing hiervoor is om de tekst te vereenvoudigen, waardoor de kerninhoud wordt behouden.

\newline

\begin{quote}
	“Ik vind het zeker de moeite waard om te onderzoeken. Leerlingen met dyslexie hebben naast problemen met technisch lezen (maken meer leesfouten en lezen vaak trager) ook problemen met begrijpend lezen. Samengestelde zinnen en complexere zinnen worden niet altijd even makkelijk begrepen (belangrijke signaalwoorden niet herkend, woordbetekenissen worden moeizamer uit de tekstcontext gehaald, enz...) Een toegankelijke tekststructuur kan voor veel doelgroepen soelaas brengen (dat is trouwens wat organisaties zoals 'Wablieft' voor ogen hebben.) “
\end{quote}

\newline

Wetenschappelijke artikelen vereenvoudigen vraagt tijd en energie van docenten in de derde graad middelbaar onderwijs. Het middelbaar onderwijs staat onder druk en docenten hebben moeite om met deze werkdruk boven water te blijven. Daarom is er nood aan software die wetenschappelijke artikelen automatisch kan vereenvoudigen, specifiek gericht op de noden van scholieren met dyslexie. Een dergelijke toepassing vermindert het routinematige werk van STEM-docenten en biedt scholieren met dyslexie in de derde graad middelbare onderwijs de mogelijkheid om de kern van een tekst sneller te begrijpen.

\section{\IfLanguageName{dutch}{Onderzoeksvraag}{Research question}}%
\label{sec:onderzoeksvraag}

% Wees zo concreet mogelijk bij het formuleren van je onderzoeksvraag. Een onderzoeksvraag is trouwens iets waar nog niemand op dit moment een antwoord heeft (voor zover je kan nagaan). Het opzoeken van bestaande informatie (bv. ``welke tools bestaan er voor deze toepassing?'') is dus geen onderzoeksvraag. Je kan de onderzoeksvraag verder specifiëren in deelvragen. Bv.~als je onderzoek gaat over performantiemetingen, dan 

% Dit onderzoek toont aan hoe de inhoud van wetenschappelijke artikelen met kunstmatige intelligentie automatisch vereenvoudigd kan worden, specifiek gericht op de noden van een scholier met dyslexie in het derde graad middelbaar onderwijs. Om een antwoord op deze onderzoeksvraag te vinden, moet het onderzoek eerst zeven fasen doorlopen. Het doel van dit onderzoek is om te achterhalen met welke technologische en logopedische aspecten AI ontwikkelaars rekening moeten houden bij het een de ontwikkeling van een adaptieve AI toepassing voor geautomatiseerde tekstvereenvoudiging. 

De volgende onderzoeksvraag is opgesteld: ”Hoe kan een wetenschappelijke artikel automatisch vereenvoudigd worden, gericht op de unieken noden van scholieren met dyslexie in de derde graad middelbaar onderwijs?”. Daarnaast worden de volgende deelvragen beantwoord.

\begin{itemize}
\item Welke aanpakken zijn er voor geautomatiseerde tekstvereenvoudiging? Aansluitende vraag: "Hoe worden teksten handmatig vereenvoudigd voor scholieren met dyslexie?"
\item Welke specifieke noden hebben scholieren van de derde graad middelbaar onderwijs bij het begrijpen van complexere teksten?
\item Wat zijn de specifieke kenmerken van wetenschappelijke artikelen?
\item Met welke valkuilen bij taalverwerking met AI moeten ontwikkelaars rekening houden?
\item Welke toepassingen, tools en modellen zijn er beschikbaar om Nederlandse geautomatiseerde tekstvereenvoudiging met AI mogelijk te maken?
\item Welke functies ontbreken AI-toepassingen om geautomatiseerde tekstvereenvoudiging mogelijk te maken voor scholieren met dyslexie in de derde graad middelbaar onderwijs? Aansluitende vraag: ”Welke manuele methoden voor tekstverereenvoudiging komen niet in deze tools voor?"
\end{itemize}


\section{\IfLanguageName{dutch}{Onderzoeksdoelstelling}{Research objective}}%
\label{sec:onderzoeksdoelstelling}

% Wat is het beoogde resultaat van je bachelorproef? Wat zijn de criteria voor succes? Beschrijf die zo concreet mogelijk. Gaat het bv.\ om een proof-of-concept, een prototype, een verslag met aanbevelingen, een vergelijkende studie, enz.

Het doel van dit onderzoek is om te achterhalen met welke technologische en logopedische aspecten AI ontwikkelaars rekening moeten houden bij het een de ontwikkeling van een adaptieve AI toepassing voor geautomatiseerde tekstvereenvoudiging. Het resultaat van dit onderzoek is een prototype voor een toepassing die de tekstinhoud van een wetenschappelijke paper zal vereenvoudigen, naargelang de specifieke noden van een scholier met dyslexie in de derde graad middelbaar onderwijs. Het prototype houdt rekening met de transformatie van het bronbestand, bijvoorbeeld een PDF of een afbeelding, naar de tekstinhoud. Hiervoor bestaan er kant-en-klare pakketten die het omzettingswerk al voor de ontwikkelaar doen. De invoer van dit prototype is een wetenschappelijk artikel van minstens 500 woorden lang.

% Als tweede onderdeel wordt er een prototype ontwikkeld om wetenschappelijke artikelen automatisch te vereenvoudigen, specifiek gericht op de noden van een scholier in de derde graad middelbaar onderwijs. Het prototype houdt geen rekening met de transformatie van het bronbestand, bijvoorbeeld een PDF of een afbeelding, naar de tekstinhoud. Dergelijke AI-toepassingen of AI-modellen die tekst uit afbeeldingen of PDF-bestanden halen, bestaan al. De invoer van dit prototype is een wetenschappelijk artikel van 300 tot 500 woorden lang. De uitvoer van dit prototype is een vereenvoudigde versie van ditzelfde wetenschappelijk artikel. Metrieken, indien mogelijk per zin, worden weergegeven. Verdere concretisering volgt...

\section{\IfLanguageName{dutch}{Opzet van deze bachelorproef}{Structure of this bachelor thesis}}%
\label{sec:opzet-bachelorproef}

% Het is gebruikelijk aan het einde van de inleiding een overzicht te
% geven van de opbouw van de rest van de tekst. Deze sectie bevat al een aanzet die je kan aanvullen/aanpassen in functie van je eigen tekst.

De rest van deze bachelorproef is als volgt opgebouwd:

In Hoofdstuk~\ref{ch:stand-van-zaken} wordt een overzicht gegeven van de stand van zaken binnen het onderzoeksdomein, op basis van een literatuurstudie.

In Hoofdstuk~\ref{ch:methodologie} wordt de methodologie toegelicht en worden de gebruikte onderzoekstechnieken besproken om een antwoord te kunnen formuleren op de onderzoeksvragen.

% TODO: Vul hier aan voor je eigen hoofstukken, één of twee zinnen per hoofdstuk

\begin{itemize}
	\item Welke aanpakken zijn er voor geautomatiseerde tekstvereenvoudiging? Aansluitende vraag: "Hoe worden teksten handmatig vereenvoudigd voor scholieren met dyslexie?"
	\item Welke specifieke noden hebben scholieren van de derde graad middelbaar onderwijs bij het begrijpen van complexere teksten?
	\item Wat zijn de specifieke kenmerken van wetenschappelijke artikelen? 
	\item Met welke valkuilen bij taalverwerking met AI moeten ontwikkelaars rekening houden?
	\item Welke toepassingen, tools en modellen zijn er beschikbaar om Nederlandstalige geautomatiseerde tekstvereenvoudiging met AI mogelijk te maken?
	\item Welke functies ontbreken AI-toepassingen om geautomatiseerde én adaptieve tekstvereenvoudiging mogelijk te maken voor \newline scholieren met dyslexie in de derde graad \newline middelbaar onderwijs? Aansluitende vraag: "Welke manuele methoden voor tekstvereenvoudiging komen niet in deze tools voor?"
\end{itemize}

In Hoofdstuk~\ref{ch:conclusie}, tenslotte, wordt de conclusie gegeven en een antwoord geformuleerd op de onderzoeksvragen. Daarbij wordt ook een aanzet gegeven voor toekomstig onderzoek binnen dit domein.
\chapter{\IfLanguageName{dutch}{Stand van zaken}{State of the art}}%
\label{ch:stand-van-zaken}

\section{Tekstvereenvoudiging}

Tekstvereenvoudiging is het proces waarin het technisch leesniveau en/of woordgebruik van een geschreven tekst wordt verminderd. Belangrijk hierbij is dat de vereenvoudiging geen effect mag hebben op de kerninhoud. Een complete vereenvoudiging van een tekst bestaat uit minstens drie transformaties \autocite{Siddharthan2014}. Daarnaast is tekstvereenvoudiging een taalbewerking dat geautomatiseerd kan worden. Tekstvereenvoudiging is namelijk een zijtak van natuurlijke taalverwerking.

\subsection{Natural Language Processing}

Natuurlijke taalverwerking of NLP is een brede term die zich richt op het verwerken en analyseren van menselijke taal door computers en andere technologieën. Het omvat verschillende technieken, zoals tekstanalyse, taalherkenning en -generatie, spraakherkenning en -synthese, en semantische analyse. Computers zijn ertoe in staat om op een menselijke manier te communiceren en begrijpen wat er wordt gezegd. Vooraleer het onderzoek zich verdiept in hoe teksten worden vereenvoudigd, moeten er eerst begrippen worden aangehaald die noodzakelijk zijn om de volgende fasen te kunnen uitleggen. \textcite{Sohom2019} haalt de volgende begrippen aan.

\begin{itemize}
	\item \textbf{Tokenisatie} splitst de stam of basisvorm van woorden in een tekst. Gebruikelijk zetten ontwikkelaars deze stap in om een woordenschat voor een taalmodel op te bouwen. Bij tokenisatie wordt er geen rekening gehouden met de betekenis achter ieder woord.
	\item \textbf{Lemmatiseren} in NLP bouwt verder op \textit{stemming}, maar de betekenis van ieder woord wordt in acht genomen. Voor het lemmatiseren bestaan er Nederlandstalige modellen, waaronder JohnSnow\footnote{https://nlp.johnsnowlabs.com/2020/05/03/lemma\_nl.html}. Bij \textbf{omgekeerd lemmatiseren} wordt er een afgeleide achterhaald vanuit de stam. Bijvoorbeeld voor het werkwoord 'zijn' zou dit 'is', 'was' of 'ben' zijn. Voor zelfstandige naamwoorden, zoals 'hond', is dit dan enkelvoud of meervoud.
	\item Bij een \textbf{parsing}-fase wordt er een label aan ieder woord of zinsdeel toegekend. Voorbeelden van labels zijn zelfstandig naamwoord, bijwoord, werkwoord, bijzin of stopwoord. Het herkennen van zinsdelen wordt \textit{chunking} genoemd. Parsing heeft een dubbelzinnigheidsprobleem, want een 'plant' staat niet gelijk aan de vervoeging van werkwoord 'planten'.
\end{itemize}

\section{De verschillende soorten tekstvereenvoudiging}

Inleiding op de soorten tekstvereenvoudiging. Tekstvereenvoudiging bestaat uit vier soorten transformaties: lexicale, syntactische en semantische vereenvoudiging en samenvatten.

\subsection{Lexicale vereenvoudiging}

% todo bron
Bij lexicale vereenvoudiging worden complexe woorden vervangen door eenvoudigere synoniemen. Bijvoorbeeld, het woord 'adhesief' kan worden vervangen door 'klevend'. De zinsstructuur verandert niet en er is garantie dat de kerninhoud en benadrukking hetzelfde blijft. Het doel van lexicale vereenvoudiging is om de moeilijkheidsgraad van de woordenschat in een zin of tekst te verlagen.

Een onderzoek van \textcite{Bulte2018} ging met dit concept aan de slag. Het resultaat van hun onderzoek was een \textit{pipeline} ontworpen om moeilijke woordenschat naar simpele synoniemen te vervangen. Eerst ging de tekstinhoud door een \textit{preprocessing}-fase, samen met het uitvoeren van WSE. Daarna werd de moeilijkheidsgraad van ieder token overlopen. De moeilijkheidsgraad is gebaseerd op hoe vaak een woord voorkomt in SONAR500\footnote{https://taalmaterialen.ivdnt.org/download/tstc-sonar-corpus/} een corpus met eenvoudige Nederlandstalige woorden. Synoniemen werden teruggevonden met Cornetto\footnote{https://github.com/emsrc/pycornetto}, een lexicale databank met Nederlandstalige woorden. Hiervoor gebruikten de onderzoekers een \textit{reverse lemmatization} fase. Lexicale vereenvoudiging is ingewikkeld wanneer er geen eenvoudigere synoniemen zijn. In dat geval blijft een moeilijk woord voor wat het is.

\subsection{Syntactische vereenvoudiging}

Syntactische vereenvoudiging transformeert de grammatica en zinsstructuur van een tekst om de complexiteit van een zin te verlagen. Bijvoorbeeld, twee afzonderlijke zinnen kunnen worden samengevoegd tot één eenvoudigere zin. Syntactische vereenvoudiging richt zich op het verminderen van complexe of onduidelijke zinsconstructies, terwijl de inhoud en betekenis van de tekst behouden blijft. Dergelijke transformaties zijn het vereenvoudigen van de syntax of door de zinnen korter te maken. Zinnen worden toegankelijker, zonder de kerninhoud of relevante inhoud te verliezen.

\subsection{Conceptuele vereenvoudiging}

Conceptuele vereenvoudiging lost dit probleem op. Theoretische kennis hierover is schaars, maar \textcite{Siddharthan2006} bestudeerde dit concept verder. Dit type vereenvoudiging betreft het opdelen van complexe concepten in eenvoudigere delen, het gebruik van duidelijke en bondige taal en het vermijden van technische jargon en abstracte uitdrukkingen. Het doel is om de inhoud begrijpelijker te maken, zonder dat hierbij de betekenis of nauwkeurigheid wordt aangetast. \textcite{Siddharthan2006} noemt deze transformatie een vorm van elaboratie of het uiteenzetten van een begrip.


\subsection{Tekstvereenvoudiging automatiseren}

Geautomatiseerde tekstvereenvoudiging is geen nieuwe zaak. Volgens het onderzoek van ... en ... waren de eerste aanpakken op geautomatiseerde tekstvereenvoudiging gebouwd op rule-based modellen. Deze modellen waren gericht op het bewerken van de syntax door onder meer zinnen te splitsen, te verwijderen of in een andere volgorde te plaatsen. Pas met recentere onderzoeken van ... en ... werd het duidelijk hoe lexicale en syntactische vereenvoudiging gecombineerd kon worden.

\subsection{Discourse edits}

\textit{Discourse}

\subsection{Combineren tot het geheel van tekstvereenvoudiging}

% Tekstvereenvoudiging omvat drie transformaties. 

Het onderzoek van \textcite{DeBelder2010} richt zich op tekstvereenvoudiging voor kinderen. De doelgroep ligt echter jonger dan deze casus, maar het onderzoek haalt aan hoe de onderzoekers een methode opzetten voor lexicale en syntactische vereenvoudiging.

\subsection{Samenvatten}

Lexicale, conceptuele en syntactische vereenvoudiging is er geen garantie dat de tekstinhoud korter zal worden. Eenvoudigere woordenschat 

\textcite{Wafaa2021} deed verder onderzoek op geautomatiseerd samenvatten.


\section{Voordelen van tekstvereenvoudiging}

\section{Struikelblokken}

\subsection{Evaluatie van de toepassing}

\subsection{Datasets}

\subsection{Meaning distortion}



\subsection{Paternalisme}
De doelstelling van assisterende software is om gelijke kansen te bieden aan iedereen. Zoals eerder vermeld, zorgt tekstvereenvoudiging voor een simpelere syntax en woordenschat in een tekst. Volgens \textcite{Niemeijer2010} zijn de ethische overwegingen die samenhangen met tekstvereenvoudiging via implicaties voor assistieve technologie niet gemakkelijk te scheiden van de technologie die wordt gebruikt om het resultaat te bereiken. Ontwikkelaars moeten, volgens deze auteur, rekening houden met de doelgroep waarvoor ze een toepassing maken.

% lenker bron
Het onderzoek van \textcite{Gooding2022} richtte zich op dit probleem. Ontwikkelaars moeten zich meer bewust worden van de behoeften en verwachtingen van de eindgebruiker bij het ontwikkelen van een tekstvereenvoudigingstoepassing. Haar onderzoek benadrukt de paternalistische en afhankelijke aard van assisterende technologieën. Tekstvereenvoudiging omvat drie transformaties, maar de moeilijkheidsgraad is niet statisch. Een adaptieve tekstvereenvoudigingstoepassing moet de eindgebruiker de keuze bieden om aan te passen wat vereenvoudigd wordt, afhankelijk van zijn of haar specifieke behoeften.

% Xu bron
Volgens \textcite{Punardeep2020}, maken de meeste AI-toepassingen voor tekstvereenvoudiging gebruik van \textit{black-box} modellen. Een \textit{black-box} model maakt het onmogelijk om transparant te zijn over waarom bepaalde transformaties worden uitgevoerd, bijvoorbeeld het vervangen van een woord door een eenvoudiger synoniem. Het model kan dus niet aangeven waarom het juist dat woord heeft vervangen door dat specifieke synoniem. Deze AI-toepassingen vallen onder de categorie van \textit{supervised learning} en het model leert handelingen uit de data waarop het is getraind. Dit is echter problematisch, aangezien \textcite{Xu2015} benadrukt dat veel toepassingen voor tekstvereenvoudiging geen rekening houden met de doelgroep waarvoor ze zijn ontwikkeld.

Om dit probleem op te lossen, is het belangrijk om de eindgebruiker, in dit geval scholieren met dyslexie in het derde graad middelbaar onderwijs, de keuze te geven. Zoals beschreven in \textcite{Gooding2022}, zijn er verschillende mogelijkheden. Bijvoorbeeld, de eindgebruiker moet de mogelijkheid hebben om te kiezen welke synoniemen de tekst lexicaal zullen aanpassen. Een alternatieve aanpak voor syntactische vereenvoudiging is om de scholier zelf zinnen te laten markeren die moeilijk te begrijpen zijn, zodat het systeem alleen de door de eindgebruiker aangegeven zinnen vereenvoudigt.


\subsection{Problemen bij lexicale vereenvoudiging}
\subsection{Problemen bij syntactische vereenvoudiging}
% Volgende zaken aanhalen:
\begin{itemize}
	\item Acroniemen
	\item Homoniemen
	\item Kerninhoud verliezen
	\item Ethisch aspect
	\item Word Sense Ambiguity
\end{itemize}


\section{Tekstvereenvoudigingssoftware in het onderwijs}

\section{Beschikbare tekstvereenvoudigingssoftware}

\section{Tekstvereenvoudigingspipeline opbouwen}

\section{Metrieken om de transformatie van tekstvereenvoudiging te beoordelen}

% Tip: Begin elk hoofdstuk met een paragraaf inleiding die beschrijft hoe
% dit hoofdstuk past binnen het geheel van de bachelorproef. Geef in het
% bijzonder aan wat de link is met het vorige en volgende hoofdstuk.

% Pas na deze inleidende paragraaf komt de eerste sectiehoofding.

% Dit hoofdstuk bevat je literatuurstudie. De inhoud gaat verder op de inleiding, maar zal het onderwerp van de bachelorproef *diepgaand* uitspitten. De bedoeling is dat de lezer na lezing van dit hoofdstuk helemaal op de hoogte is van de huidige stand van zaken (state-of-the-art) in het onderzoeksdomein. Iemand die niet vertrouwd is met het onderwerp, weet nu voldoende om de rest van het verhaal te kunnen volgen, zonder dat die er nog andere informatie moet over opzoeken \autocite{Pollefliet2011}.

% Je verwijst bij elke bewering die je doet, vakterm die je introduceert, enz.\ naar je bronnen. In \LaTeX{} kan dat met het commando \texttt{$\backslash${textcite\{\}}} of \texttt{$\backslash${autocite\{\}}}. Als argument van het commando geef je de ``sleutel'' van een ``record'' in een bibliografische databank in het Bib\LaTeX{}-formaat (een tekstbestand). Als je expliciet naar de auteur verwijst in de zin, gebruik je \texttt{$\backslash${}textcite\{\}}.
% Soms wil je de auteur niet expliciet vernoemen, dan gebruik je \texttt{$\backslash${}autocite\{\}}. In de volgende paragraaf een voorbeeld van elk.

% \textcite{Knuth1998} schreef een van de standaardwerken over sorteer- en zoekalgoritmen. Experten zijn het erover eens dat cloud computing een interessante opportuniteit vormen, zowel voor gebruikers als voor dienstverleners op vlak van informatietechnologie~\autocite{Creeger2009}.

\lipsum[7-20]

%%=============================================================================
%% Methodologie
%%=============================================================================

\chapter{\IfLanguageName{dutch}{Methodologie}{Methodology}}%
\label{ch:methodologie}

%% TODO: Hoe ben je te werk gegaan? Verdeel je onderzoek in grote fasen, en
%% licht in elke fase toe welke stappen je gevolgd hebt. Verantwoord waarom je
%% op deze manier te werk gegaan bent. Je moet kunnen aantonen dat je de best
%% mogelijke manier toegepast hebt om een antwoord te vinden op de
%% onderzoeksvraag.

Het onderzoek wordt aangevat met een grondige literatuurstudie, welke in Hoofdstuk 2 van de methodologie uiteengezet wordt. Deze literatuurstudie omvat een overzicht van de beschikbare technologieën en methoden voor het vereenvoudigen van teksten voor leerlingen met dyslexie in de derde graad van het middelbaar onderwijs. De literatuurstudie biedt een duidelijke uiteenzetting van de verschillende aspecten van het onderzoek, met als doel de lezer de vereiste kennis bij te brengen om de resultaten van de analyse te begrijpen.

Hoofdstuk 4 wordt er gekeken naar beschikbare tools die in staat zijn om teksten te vereenvoudigen voor scholieren met dyslexie. Er wordt gekeken naar de functionaliteiten en eigenschappen van de tools, alsook naar de doelgroep waarvoor de tool geschikt is. Er wordt ook gekeken naar de prijs, de gebruiksvriendelijkheid en de compatibiliteit met bestaande software en hardware. Op basis van deze criteria wordt er een shortlist opgesteld van tools die geschikt zijn voor het vereenvoudigen van teksten voor leerlingen met dyslexie in de derde graad van het middelbaar onderwijs.

Hoofdstuk 5 beschrijft vervolgens de ontwikkeling van een prototype voor tekstvereenvoudiging. Dit prototype wordt geprogrammeerd op basis van de requirementsanalyse, waarbij er rekening wordt gehouden met de functionaliteiten en eigenschappen die uit de shortlist van tools naar voren zijn gekomen. Het prototype wordt stap voor stap opgebouwd, waarbij er aandacht wordt besteed aan de gebruiksvriendelijkheid, de snelheid en de nauwkeurigheid van de tool. Er wordt ook gekeken naar de compatibiliteit van het prototype met bestaande software en hardware, zodat de tool naadloos kan integreren in het onderwijs voor scholieren met dyslexie in de derde graad van het middelbaar onderwijs.

In Hoofdstuk 6 worden de verschillende tools met elkaar vergeleken door middel van een mixed-methods vergelijkende studie. De tools worden gebruikt om een oorspronkelijk wetenschappelijk artikel in PDF-formaat te uploaden en deze te laten vereenvoudigen of samenvatten. Op deze manier kan bepaald worden welke tools en middelen het meest geschikt zijn voor het vereenvoudigen van wetenschappelijke artikelen op maat van leerlingen met dyslexie in de derde graad van het middelbaar onderwijs. De vergelijkende studie richt zich op de metrieken en vereisten die in Hoofdstuk 2 besproken zijn, met als doel vast te stellen aan welke criteria een vereenvoudigde tekst moet voldoen om leerlingen met dyslexie in de derde graad van het middelbaar onderwijs te ondersteunen

\chapter{Requirementsanalyse}

In deze fase van het onderzoek worden de tools uitgetest. Functionaliteiten met een bevorderend effect uitgewezen in Hoofdstuk 2, worden genoteerd. Daarnaast worden aspecten waarmee ontwikkelaars rekening mee moeten houden ook betrokken in de requirementsanalyse. Eerst wordt software uitgetest die nu in het onderwijs wordt ingezet. Vervolgens worden online beschikbare tools uitgetest die lectoren in het onderwijs kan gebruiken. De uitvoer van de wetenschappelijke artikelen wordt vergeleken met geautomatiseerde en handmatige tekstanalyse. Voor de geautomatiseerde analyse zijn er pakketten beschikbaar zoals \textit{readability} of \textit{textstat}. Met behulp van Pandas worden de statistieken in tabelvorm weergegeven. Tien teksten hiervan werden op semantisch vlak vergeleken met de oorspronkelijke teksten vereenvoudigd door een mens.

\subsection{Tekstanalyse}

Geen enkel softwarepakket of hulpmiddel biedt standaard een visuele weergave van waarom een taal- of AI-model een zin als moeilijk of belangrijk beschouwt, of waarom het model een kernwoord heeft gekozen. Dit komt overeen met de bevindingen van \textcite{Gooding2019}. Het GPT3-model en het verwante Bing-model doen dit echter wel wanneer het taalmodel hier expliciet om wordt gevraagd. SciSpace houdt hier geen rekening mee en verwerpt de vraag. Het stellen van vragen aan het taalmodel biedt weliswaar een alternatief, maar valt buiten het bereik en de capaciteiten van de gemiddelde gebruiker. Deze prompt kan worden aangeboden in de vorm van een intuïtieve knop. 

\subsection{Lexicale vereenvoudiging}

Simplish geeft nadien een vergelijkende weergave met de oorspronkelijke tekst en de vereenvoudigde tekst. Met gebruik van kleurcodes worden de verschillende transformaties aangeduid.

\begin{figure}
	\includegraphics{img/simplish-input.png}
	\includegraphics{img/simplish-output.png}
\end{figure}

\subsubsection{Syntactische vereenvoudiging}

\subsubsection{Samenvatten}

De experimenten met teksten wijzen uit dat GPT en Bing AI de nadruk legt op het behouden van bronreferenties. Wanneer expliciet gevraagd aan de Bing chatbot, geeft het model bronnen terug die buiten het oorspronkelijke artikel te vinden zijn.

\subsection{Conclusie}

\chapter{Prototype voor tekstvereenvoudiging}

\section{Voorbereiding}

\subsubsection{Python-notebooks}

De werking van tekstvereenvoudiging via Python-code wordt optimaal weergegeven in Python notebooks. Het gebruik van API's vormt geen hindernis en voorziet een snelle weergave zonder dat code uitgevoerd moet worden. 

\subsubsection{Keuze back-end en front-end}

Het prototype maakt gebruik van een Flask en het Jinja-framework. Aanvullend maakt het prototype gebruik van de nodige HTML- en CSS bestanden om de nodige visuele ondersteuning te kunnen aanbieden aan zowel lectoren als scholieren met dyslexie in de derde graad van het middelbaar onderwijs. Het aanspreken van de back-end vanuit de HTML-pagina's gebeurt met JavaScript-calls.

\subsubsection{Docker-omgeving}

Voor een optimale opzet als ontwikkelaar wordt er gebruik gemaakt van Docker. Een bat-scriptbestand maakt de opstart van deze lokale webapplicatie intuïtiever dan de opstart via een terminal. Daarnaast worden de benodigde Python-bibliotheken en taalmodellen alvorens opgehaald.

\subsubsection{PDF text mining}

De meest gebruikte vorm van tekstbestanden is in de vorm van PDF-bestanden. Methoden om deze bestanden uit te lezen en om te zetten naar tekstdata bestaan reeds in de vorm van Python-bibliotheken. De keuze van de bibliotheek speelt wel een rol. Sommige Python-bibliotheken bieden meer analyse- en verwerkingsfuncties aan. PDF Miner biedt extra functies aan om de metadata van een PDF-bestand op te halen of om classificatie op basis van lettertypes of pagina's per titel op te halen. Deze functies bevatten een intuïtieve naamgeving en besparen ontwikkelaars de overbodige werklast om extra logica te schrijven.

\subsubsection{Data pre-processing}

% todo stopword removal voor keyword extraction

\subsection{Extraherende samenvatting}

\subsubsection{BERT Extractive Summarization}

Indien niet aangegeven, wordt het aantal zinnen bepaald door het BERT-model.

\subsubsection{Vertaling}

BERT houdt geen rekening met vertaling. Het model staat enkel paraat om zinnen te markeren en de belangrijkste zinnen terug te geven. Voor de vertaling wordt de Google Translate Python-package gebruikt. Deze is minder accuraat vergeleken met DeepL, maar biedt een gratis beschikbaar en aanvaardbaar alternatief aan. Factoren zoals topic diversity en semantische redundantie moeten overwogen worden bij het kiezen van een taalmodel voor extraherend samenvatten.

\begin{lstlisting}[language=Python]
def extractive_summarization(full_text):

try:    
    from summarizer import Summarizer
    model = Summarizer()

    """determining optimal number of sentences based on MMR"""
    res = model.calculate_optimal_k(
        full_text, 
        k_max=10
    )

    """extracting key sentences"""
    result = model(
        body=full_text,
        max_length=700,
        min_length=100,
        num_sentences=res,
        return_as_list=True
    )

    new = []
    try:
        for i in result:
            if detect(i) != LANG:
                new.append(translate_sentence(i))
            else:
                new.append(i)
        return ' '.join(new)
    except Exception as e:
        return f'Problemen met Google Translate {e}'

except Exception as e:
    return f'Problemen met BERT {e}'
\end{lstlisting}

\subsection{Integreren naar Flask}

Flask biedt een snelle en toegankelijke opzet aan voor Python-ontwikkelaars. De combinatie van een robuuste front-end en back-end maakt dit binnen de schaal van een prototype ideaal.

\begin{lstlisting}

@app.route('/', methods=['GET'])
def home():
	return render_template('index.html')

if __name__ == "__main__":
	app.run()
\end{lstlisting}

% \subsection{title}



% Voeg hier je eigen hoofdstukken toe die de ``corpus'' van je bachelorproef
% vormen. De structuur en titels hangen af van je eigen onderzoek. Je kan bv.
% elke fase in je onderzoek in een apart hoofdstuk bespreken.
%%=============================================================================
%% Discussie
%%=============================================================================

\chapter{\IfLanguageName{dutch}{Discussie}{Discussion}}%
\label{ch:discussie}

In dit hoofdstuk worden de resultaten uit de requirementsanalyse, vergelijkende studie en de ontwikkeling van het prototype besproken. 

\section{Requirementsanalyse}

% Startende zin over lexicale vereenvoudiging
Woorden- en synoniemenlijsten kunnen een ondersteunend middel aanbieden voor zowel scholieren met dyslexie als zonder bij het lezen van wetenschappelijke artikelen en wordt aangeboden in Kurzweil. Automatisch genereren is enkel prevalent binnen ChatGPT en de Bing chatbot, maar de tools houden geen rekening met de doelgroep, tenzij expliciet aangegeven met een one-shot summary. Andere tools houden helemaal geen rekening met de doelgroep en kunnen enkel woordenlijsten genereren op basis van gekozen woorden. 

\medspace


Bij de geteste tools is PDF-upload de standaardmethode. Het inlezen van tekstinhoud is beschikbaar bij de webtoepassingen, maar niet bij de erkende software in het onderwijs. ChatGPT en Bing chatbot genereren mensachtige teksten, maar kunnen geen PDF's verwerken. Het kopiëren en plakken van tekst uit het originele document kan leiden tot weinig fouten, maar is omslachtig en moet verbeterd worden. Bestaande tools hebben moeite met oudere PDF's waarbij niet alle tekst kan worden geëxtraheerd. Een geavanceerde optie voor het inlezen van PDF's is vereist.

\medspace

Tools spenderen weinig tijd op de analyse over het ingegeven document, alsook het vereenvoudigde of samengevatte document. Simplish doet dit wel en geeft aan de hand van kleurcodes bepaalde data over de vereenvoudigde tekst mee, waaronder niet-veranderende woorden, adequate vertalingen, uitleg naar de voetnoot, homoniemen of woorden waarvan er geen eenvoudig synoniem is. Zoals aangegeven in \ref{img:simplish-output} duidt de vergelijkende weergave de verschillen aan tussen de oorspronkelijke en vereenvoudigde tekst en met behulp van kleurcodes worden de verschillende transformaties aangegeven.

\begin{figure}[H]
	\includegraphics[width=\linewidth]{img/simplish-output.png}
	\label{img:simplish-output}
\end{figure}

\begin{center}
	\begin{tabular}{ | m{4cm} | m{12cm} | } 
		\hline
		\textbf{MoSCoW-principe} & Functionaliteit \\
		\hline
		Must-have & Gepersonaliseerde vereenvoudiging aanbieden, waaronder lexicale en syntactische vereenvoudiging aanbieden, na het toevoegen van een respectievelijke API-sleutel. \\
		& Wetenschappelijke artikelen in PDF-vorm opladen. \\
		& Personaliseerbare site: lettertype -en grootte aanpassen, tekstformaat aanpassen, achtergrondkleur aanpassen \\
		& Lokale opzet \\
		\hline
		Should-have & Glossary genereren na handmatige selectie van moeilijke woorden \\
		& Personaliseerbare PDF- of Worddocumentlay-out \\
		& Uitvoer als PDF of Word-bestand teruggeven. \\
		\hline
		Could-have & Glossary genereren na automatische selectie van moeilijke woorden \\
		\hline
		Wont-have & Beschikbaarheid tot de tool zonder Docker Desktop, in de vorm van online webtoepassing of browserextensie. \\
		& Beschikbaarheid tot de standaard- en gepersonaliseerde opties zonder API-sleutels \\
		\hline
	\end{tabular}
\end{center}

\section{Vergelijkende studie}

De huidige softwaretools die worden gebruikt in het middelbaar onderwijs zijn niet in staat om de oorspronkelijke tekst te transformeren, wat betekent dat syntactische vereenvoudiging momenteel niet haalbaar is. Hoewel er online webtoepassingen beschikbaar zijn, bieden ze minder functionaliteiten om de moeilijkheidsgraad van zinsyntaxis te verlagen en zijn ze voornamelijk gericht op het verkorten van de oorspronkelijke tekst ofwel samenvatting. Het aanpassen van tangconstructies, verwijswoorden, voorzetseluitdrukkingen, samengestelde werkwoorden en onregelmatige werkwoorden blijft daarom een uitdaging voor deze toepassingen. Zelfs het schrijven in de actieve stem kan problematisch zijn, en er zijn alleen vooraf gedefinieerde prompts beschikbaar om deze transformaties uit te voeren.

\medspace

Hoewel taalmodellen zoals GPT-3 in staat zijn om zinsyntaxtransformaties uit te voeren, kunnen ze soms problemen ondervinden bij het verwerken van alle meegegeven transformaties, en er is geen garantie dat deze modellen alle transformaties met slechts één prompt kunnen uitvoeren. Om deze uitdagingen aan te pakken, kunnen bestaande pipelines voor tekstvereenvoudiging gebruikmaken van verschillende transformers, waarbij de tekst meerdere keren aan het GPT-3 model wordt gegeven maar met verschillende prompts. Het moet echter opgemerkt worden dat taalmodellen van HuggingFace minder gericht zijn op het aanpassen van de zinsyntaxis en vaak vrijwel identieke tekst genereren.

\medspace

\subsection{Conclusie}

Ontwikkelaars kunnen voor algemene samenvattings- en vereenvoudigingstaken gebruik maken van algemene taalmodellen die vrij beschikbaar op \textit{HuggingFace} of dergelijke platforms terug te vinden zijn. GPT-3 blinkt uit in gepersonaliseerde vereenvoudigings- en samenvattingstaken. Engelstalige prompts die expliciet de uitvoertaal vermelden zijn nauwkeuriger dan Nederlandstalige prompts. 

\section{Opbouw van het prototype}

Deze ontwerpkeuze bespaart geheugenruimte voor ontwikkelaars en vermindert de benodigde rekenkracht voor een prototype. Eenmaal ontwikkelaars de toepassing willen uitrollen naar het grote publiek, wordt er net zoals bij (...) aangeraden om de taalmodellen zelf te hosten.



%%=============================================================================
%% Conclusie
%%=============================================================================

\chapter{Conclusie}%
\label{ch:conclusie}

Deze scriptie tracht een antwoord te bieden op de volgende onderzoeksvraag:

\begin{itemize}
	\item Hoe kan een wetenschappelijk artikel automatisch vereenvoudigd worden, gericht op de unieke noden van scholieren met dyslexie in de derde graad middelbaar onderwijs?
\end{itemize}

% ontbrekende mts en ontbrekende
Allereerst geeft de requirementsanalyse nieuwe inzichten in de huidige toepassingen voor ATS. Zo beschikken online tools te weinig over gepersonaliseerde ATS-functionaliteiten, zoals gebleken in sectie \ref{sec:requirementsanalyse}. Daarnaast maken geen tools buiten software specifiek voor scholieren met dyslexie geen gebruik van gepersonaliseerde opmaakopties. Toepassingen die wél wetenschappelijke artikelen kunnen opladen, beschikken over onvoldoende functionaliteiten om gepersonaliseerde ATS mogelijk te maken. Daarnaast ontbreken toepassingen die wél gepersonaliseerde ATS kunnen aanreiken over de nodige middelen om wetenschappelijke artikelen op een eenduidige manier op te laden. Om deze tools te kunnen gebruiken, moeten gebruikers \textit{commandline-interfaces} of chatbots gebruiken.

\medspace

% welk taalmodel gebruiken?
De vergelijkende studie wijst uit dat de geteste taalmodellen in staat zijn om LS mogelijk te maken. SS is enkel beschikbaar bij het GPT-3 model. Dit taalmodel kan doelgroepen in grote lijnen inschatten, alsook formaatwijzigingen toepassen zoals het herschrijven van een tekst als opsomming of in tabelvorm. Andere geteste HF-taalmodellen behalen zwakkere resultaten en vereisen een extra vertaalfase, die niet nodig is bij het aanspreken van de GPT-3 API.

\medspace

% hoe een prototype opzetten?
Uit de ontwikkeling van het prototype voor gepersonaliseerde ATS bleek dat \textit{open-source} AI en NLP-technologieën voldoende hoogstaand genoeg zijn om kwaliteitsvolle tekstvereenvoudigingssoftware te ontwikkelen. Zo kunnen ontwikkelaars gebruikmaken van PDFMiner om tekstinhoud uit wetenschappelijke artikelen te extraheren, van OpenAI's GPT-3 model via de API om gepersonaliseerde ATS mogelijk te maken en ten slotte van Pandoc om dynamische en gepersonaliseerde PDF-documenten automatisch te genereren. Binnen een webapplicatie kunnen eenduidige handelingen, gebouwd in JavaScript en HTML\&CSS, complexe commandlinehandelingen afhandelen. Ontwikkelaars kunnen met eenvoudige en open-source tools een webpagina opbouwen die voldoet aan de noden beschreven in \textcite{Rello2012a}. Het prototype maakt gebruik van duidelijke symbolen, 

\medspace

Ontwikkelaars hebben toegang tot T1, T2 en T3 via HuggingFace voor lexicale vereenvoudigingstaken. Deze taalmodellen zijn echter ontoereikend voor gepersonaliseerde ATS, want ze ontbreken SS-technieken om de tekst op een syntactisch niveau te vereenvoudigen. Daarnaast beschikken ze over een ingebakken doelgroep. T4 is een geschikter model voor het vereenvoudigen van wetenschappelijke artikelen op maat van scholieren met dyslexie in de derde graad van het middelbaar onderwijs. Zo presteert T4 goed op gepersonaliseerde LS en SS-technieken, maar het is belangrijk om op te merken dat geen enkel taalmodel de doelgroep altijd nauwkeurig kan inschatten. Extra trainingsdata in de vorm van leerstof op leesniveau van de doelgroep kan het model helpen bij de doelgroepsinschatting. Het gebruik van Engelstalige prompts met expliciete vermelding van de gewenste uitvoertaal, resulteert in coherentere teksten dan bij een Nederlandstalige prompt.

%---------- Bijlagen -----------------------------------------------------------

\appendix

\chapter{Onderzoeksvoorstel}

\section*{Samenvatting}

% Kopieer en plak hier de samenvatting (abstract) van je onderzoeksvoorstel.
Ingewikkelde woordenschat en zinsbouw hinderen scholieren met dyslexie in het derde graad middelbaar onderwijs bij het lezen van wetenschappelijke artikelen. Gepersonaliseerde tekstvereenvoudiging helpt deze scholieren bij hun leesbegrip. Daarnaast kan artificiële intelligentie (AI) dit proces automatiseren om de werkdruk bij leraren en scholieren te verminderen. Dit onderzoek achterhaalt met welke technologische en logopedische aspecten AI-ontwikkelaars rekening moeten houden bij de ontwikkeling van een AI-toepassing voor geautomatiseerde en gepersonaliseerde tekstvereenvoudiging. Hiervoor is de volgende onderzoeksvraag opgesteld: "Hoe kan een wetenschappelijk artikel automatisch worden vereenvoudigd, gericht op de unieke noden van scholieren met dyslexie in het derde graad middelbaar onderwijs?". Een requirementsanalyse achterhaalt de benodigde functionaliteiten om gepersonaliseerde en geautomatiseerde tekstvereenvoudiging mogelijk te maken. Vervolgens wijst de vergelijkende studie uit welk taalmodel kan worden ingezet om de taak van gepersonaliseerde en geautomatiseerde tekstvereenvoudiging mogelijk te maken. De requirementsanalyse wijst uit dat toepassingen om wetenschappelijke artikelen te vereenvoudigen, gemaakt zijn voor een centrale doelgroep en geen rekening houden met de unieke noden van een scholier met dyslexie in het derde graad middelbaar onderwijs. Adaptieve software voor geautomatiseerde tekstvereenvoudiging is mogelijk, maar ontwikkelaars moeten meer inzetten op de unieke noden van deze scholieren. 

% Verwijzing naar het bestand met de inhoud van het onderzoeksvoorstel
%---------- Inleiding ---------------------------------------------------------

\section{Introductie}%
\label{sec:introductie}

% Met een jaarlijks budget van 32 miljoen in het vakgebied kunstmatige intelligentie (AI) op de werkvloer is België een pionier \autocite{Crevits2022}.  Zo zijn er verschillende projecten, om taalgerelateerde AI-ontwikkelingen op te starten, uit de grond gestampt. Het amai!-project \footnote{https://amai.vlaanderen/}  verenigt AI-softwarebedrijven uit verschillende domeinen om zo met AI-toepassingen te maken die processen automatiseren om de werkdruk te verminderen, zoals binnen het onderwijs \textit{real-time} ondertiteling en een taalassistent voor leerkrachten in meertalige klasgroepen.

Het Vlaamse middelbaar onderwijs staat nu op barsten, want de leraren en scholieren in het Vlaamse middelbaar onderwijs ondergaan aan de werkdruk en stress. Daarnaast is de derde graad in het middelbaar onderwijs is een cruciale stap voor de verdere loopbaan van scholieren, al hebben scholieren moeite met de gekregen vakliteratuur \autocite{Dapaah2022}. Het STEM-agenda\footnote{https://www.vlaanderen.be/publicaties/stem-agenda-2030-stem-competenties-voor-een-toekomst-en-missiegericht-beleid} van de Vlaamse Overheid bestaat uit aandachtspunten om het STEM-onderwijs tegen 2030 aantrekkelijker te maken door de ondersteuning voor zowel leerkrachten als scholieren te verbeteren. Echter, het overbruggen van de steeds complexere wetenschappelijke jargon is niet opgenomen in de prioriteiten van de STEM-agenda, ondanks het feit dat geautomatiseerde en adaptieve tekstvereenvoudiging een revolutionaire oplossing aanbiedt.

Dit onderzoek achterhaalt hoe de inhoud van een wetenschappelijke artikel op een geautomatiseerde wijze vereenvoudigd kan worden, gericht op de behoeften van scholieren met dyslexie in de derde graad middelbaar onderwijs. Hierbij wordt gestart met een theoretische basis voor tekstvereenvoudiging en een literatuurstudie naar de uitdagingen die een toepassing in acht moet nemen. In een vervolgstap wordt met een veldonderzoek gekeken naar bestaande AI toepassingen voor tekstvereenvoudiging in Nederlandstalige en Engelstalige teksten. Hierna beschrijft het onderzoek een pipeline voor geautomatiseerde tekstvereenvoudiging en staat het stil bij de verschillende metrieken om een vereenvoudigde tekst te beoordelen. Daarna vindt een vergelijkende studie plaats tussen de vereenvoudigde tekstinhoud van verschillende aangehaalde toepassingen, die beoordeeld wordt met behulp van enquêtes en statistische metrieken. Tot slot worden de resultaten van het onderzoek gebruikt om inzicht te krijgen in hoe wetenschappelijke artikelen op een geautomatiseerde en adaptieve manier vereenvoudigd kunnen worden, specifiek voor scholieren met dyslexie in het derde graad middelbaar onderwijs. Dit leidt tot verdere ontwikkeling voor AI-ontwikkelaars om een bruikbare toepassing te creëren voor gebruik in het onderwijs.


% Als laatste beschrijving haalt het onderzoek de verschillende evaluatietechnieken aan die nodig zijn om een vereenvoudigde tekst te beoordelen, alsook welke ethische aspecten ontwikkelaars in acht moeten houden bij het opzetten van een dergelijke toepassing. 

%---------- Stand van zaken ---------------------------------------------------

\section{State-of-the-art}%
\label{sec:state-of-the-art}

% Deelvraag: Wat is tekstsimplificatie
De voorbije tien jaar is kunstmatige intelligentie (AI) sterk verder ontwikkeld. De toename in kennis zorgde voor nieuwe toepassingen \autocite{Vasista2022}. Tekstvereenvoudiging vloeide hier uit voort. Momenteel bestaan er al robuuste toepassingen die teksten kunnen vereenvoudigen, zoals Resoomer\footnote{https://resoomer.com/nl/}, Paraphraser\footnote{https://www.paraphraser.io/nl/tekst-samenvatting} en Prepostseo\footnote{https://www.prepostseo.com/tool/nl/text-summarizer}. Binnen het kader van tekstvereenvoudiging is er bestaande documentatie beschikbaar waar onderzoekers het voordeel van toegankelijkheid aanhalen, maar volgens \textcite{Gooding2022} ontbreken deze toepassingen de extra noden die scholieren met dyslexie in het derde graad middelbaar onderwijs vereisen.

\textcite{Shardlow2014} haalt aan dat het algemene doel van tekstvereenvoudiging is om ingewikkelde bronnen toegankelijker te maken. Het zorgt voor verkorte teksten zonder de kernboodschap te verliezen. \textcite{Siddharthan2014} haalt verder aan dat tekstvereenvoudiging op één van drie manieren gebeurt. Er is conceptuele vereenvoudiging waarbij documenten naar een compacter formaat worden getransformeerd. Daarnaast is er uitgebreide modificatie die kernwoorden aanduidt door gebruik van redundantie. Als laatste is er samenvatting die documenten verandert in kortere teksten met alleen de topische zinnen. Met deze concepten zijn ontwikkelaars volgens \textcite{Siddharthan2014} in staat om ingewikkelde woorden te vervangen door eenvoudigere synoniemen of zinnen te verkorten zodat ze sneller leesbaar zijn.

Tekstvereenvoudiging behoort tot de zijtak van natuurlijke taalverwerking (NLP) in kunstmatige intelligentie. NLP omvat methodes om, door machinaal leren, menselijke teksten om te zetten in tekst voor machines. Documenten vereenvoudigen met NLP kan volgens \textcite{Chowdhary2020} op twee manieren: extract of abstract. Bij extractieve simplificatie worden zinnen gelezen zoals ze zijn neergeschreven. Vervolgens bewaart een document de belangrijkste taalelementen om de tekst te kunnen hervormen. Deze vorm van tekstvereenvoudiging komt volgens \autocite{Sciforce2020} het meeste voor. Daarnaast is er abstracte simplificatie die de kernboodschap van de zin bewaart en daarmee een nieuwe zin opbouwt. Volgens het onderzoek van \textcite{Chowdhary2020} heeft deze vorm potentieel dankzij de menselijke interpretatie, maar zit nog in de kinderschoenen.

% Deelvraag 2: Bewezen voordelen van tekstsimplificatie bij scholieren met dyslexie
Voor kinderen met dyslexie bestaan digitale hulpmiddelen die voor een betere visuele presentatie zorgen van teksten. Zo haalt het onderzoek van \textcite{Rello2012} tips aan waarmee teksten en documenten rekening moeten houden bij scholieren met dyslexie in het derde graad middelbaar onderwijs. Het gaat over speciale lettertypes, spreiding tussen woorden en het gebruik van inzoomen op aparte zinnen. Het onderzoek haalt verder AIaan dat teksten voor deze unieke noden aanpassen tijdrovend is, dus tekstvereenvoudiging door kunstmatige intelligentie kan een revolutionaire oplossing bieden. 

Het onderzoek van Franse wetenschappers \newline \textcite{Gala2016} illustreert dat manuele tekstvereenvoudiging schoolteksten toegankelijker \newline maakt voor kinderen met dyslexie. Dit deden ze door simpelere synoniemen en zinsstructuren te gebruiken. Verwijswoorden werden vermeden en woorden kort gehouden. De resultaten waren veelbelovend. Het leestempo lag hoger en de kinderen maakten minder leesfouten. Ook bleek er geen verlies van begrip in de tekst bij geteste kinderen. Resultaten van de studie werden gebundeld voor de mogelijke ontwikkeling van een AI hulpmiddel.

De Universiteit van Kopenhagen is met bovenstaande idee aan de slag gegaan. Onderzoekers \textcite{Bingel2018} hebben gratis software ontwikkeld, genaamd Hero\footnote{https://beta.heroapp.ai/}, om tekstvereenvoudiging voor scholieren in het middelbaar onderwijs met dyslexie te automatiseren. De software bestudeert met welke woorden de gebruiker moeite heeft, en vervangt die door simpelere alternatieven. Hero bevindt zich in beta-vorm en wordt enkel in het Engels en het Deens ondersteund. 

% Deelvraag: Waarop moet er gefixeerd worden bij een wetenschappelijke paper
\textcite{PlavenSigray2017} halen aan hoe onderzoekers in hun taalbubbel blijven, wat gevolgen voor de lezers met zich meebrengt. Daarnaast brengt de stijging aan het gebruik van acroniemen volgens \textcite{Barnett2020} een extra obstakel met zich mee. Het onderzoek van \textcite{Donato2022} wijst uit dat scholieren met dyslexie in het middelbaar onderwijs die uit hun richting vallen, te wijten zijn aan ondoorgrondelijke teksten. Dit bleek vooral bij STEM-richtingen het geval. 

% Deelvraag: Uitdagingen van AI-software met tekstsimplificatie
\textcite{Roldos2020} haalt aan dat NLP in de laatste decennia volop in ontwikkeling is, maar ontwikkelaars botsen nog op uitdagingen. Het gaat om zowel interpretatie- als dataproblemen bij AI machines. Het onderzoek haalt twee punten aan. Allereerst is het voor een machine moeilijk om de context van homoniemen te achterhalen. Bijvoorbeeld bij het woord ‘bank’ is het niet duidelijk voor de machine of het gaat over de geldinstelling of het meubel. Daarnaast zijn synoniemen geen probleem voor tekstverwerking.

Het onderzoek van \textcite{Sciforce2020} haalt aan dat het merendeel van NLP-toepassingen Engelstalige invoer gebruikt. Niet-Engelstalige toepassingen zijn zeldzaam. De opkomst van AI technologieën die twee datasets gebruiken, biedt een oplossing voor dit probleem. De software vertaalt eerst de oorspronkelijke tekst naar de gewenste taal, voordat de tekst wordt herwerkt. Hetzelfde onderzoek bewijst dat het vertalen van gelijkaardige talen, zoals Duits en Nederlands, een minimaal verschil opleverd.

% Deelvraag: Stand van zaken bij Belgische secundaire scholen
De Vlaamse overheid leent gratis abonnementen uit voor voorlees- en schrijfsoftware. De voornaamste zijn SprintPlus\footnote{https://www.sprintplus.be/}, Alinea\footnote{https://sensotec.be/product/alinea-suite/} en Kurzweil3000\footnote{https://sensotec.be/product/kurzweil-3000/}. Vlaamse scholieren met dyslexie in het middelbaar onderwijs kunnen voor deze software een gratis abonnement of licentie aanvragen. Al bieden de vijf softwarepakketten elk een samenvattingsfunctie aan, echter ligt de focus op spreek- en luisterfuncties waarbij het samenvatten en markeren van tekst als extra wordt gehouden.

ChatGPT\footnote{https://chat.openai.com/chat} van OpenAI is een \textit{chatbot} gebouwd op het GPT-3 model. Het GPT-3 model omvat meer dan vijf miljard verschillende woorden, wat het revolutionair maakt voor AI taaltoepassingen. Nadelig moet de \textit{chatbot} via de online toepassing expliciet gevraagd worden om tekst te kunnen vereenvoudigen. \textcite{Verhoeven2023} haalt aan dat toepassingen zoals ChatGPT een wondermiddel zijn om de werklast van routinematig en boilerplate werk te verminderen in het onderwijs. Toepassingen ontwikkelen met het GPT-3 model is mogelijk, al is de API van GPT-3 enkel tegen betaling beschikbaar. Readable\footnote{https://readable.com/} is een Engelstalige AI toepassing dat zinnen beoordeeld met leesbaarheidsformules. Bij beide tools is het enkel mogelijk om tekst op de webpagina te plakken, dus er kunnen geen PDF-documenten of scans worden geüpload en eenzelfde werking verwachten.

Vlaanderen heeft weinig zicht op de geïmplementeerde AI software in scholen. Dit werd vastgesteld door \autocite{Martens2021}, een samenwerking tussen de Vlaamse universiteiten en overheid voor kunstmatige intelligentie. Vergeleken met andere Europese landen, maakt België het minst gebruik van leerling-georiënteerde hulpmiddelen. Degenen die wel gebruikt worden, zijn vooral online leerplatformen voor zelfstandig werken. Ook maakt België amper gebruik van beschikbare software die de leermethoden en -noden van leerlingen evalueert \autocite{Martens2021a}. 

% Deelvraag: Wat is er nodig voor tekstsimplificatie? 
Python staat bovenaan de lijst van programmeertalen voor NLP-toepassingen. Volgens het onderzoek van \textcite{Thangarajah2019} is dit te wijten aan de eenvoudige syntax, kleine leercurve en grote beschikbaarheid van kant-en-klare bibliotheken. Moeilijke wiskundige berekeningen of statistische analyses kunnen worden uitgevoerd door middel van één lijn code. Een artikel van \textcite{Malik2022} haalt de twee meest voorkomende aan, namelijk NLTK\footnote{https://www.nltk.org/} en Spacy\footnote{https://spacy.io/}.

Iedere soort tekstvereenvoudiging omvat verschillende fases. Het onderzoek van \textcite{Shardlow2014} wijst uit dat een pipeline voor lexicale vereenvoudiging uit vier fases bestaat. Een \textit{proof-of-concept} genaamd \textit{Deep Martin}\footnote{https://github.com/chrislemke/deep-martin} bouwt verder op dit theoretisch concept. Hun pipeline maakt gebruik van \textit{custom transformers} om invoertekst om te zetten naar een vereenvoudigde versie van de tekstinhoud.

\textcite{Garbacea2021} benadrukken dat AI ontwikkelaars te weinig aandacht besteden aan het achterhalen waarom een woord of zin moet worden aangepast. Zij halen twee ethische aspecten van AI taaltoepassingen aan de eindgebruiker moet worden meegegeven. Allereerst moet de toepassing meegeven waarom een zin of woord is aangepast. De moeilijkheidsgraad van de woord of de zin moet worden bewezen door het model. \textcite{Iavarone2021} haalt zo een methode aan om de moeilijkheidsgraad te bepalen. In dit onderzoek werden regressiemodellen ingezet om een gemiddelde moeilijkheidspercentage te berekenen per zin. Verder haalt \textcite{Garbacea2021} om de complexe delen van een tekst te markeren. Hiervoor worden \textit{lexical of deep learning} methoden aangehaald.

Er is een tactvolle aanpak nodig om een vereenvoudigde tekst met AI te beoordelen. De studie van \textcite{Swayamdipta2019} haalt aan dat er extra nood is aan NLP-modellen waarbij de tekst zijn kernboodschap behoudt. Samen met Microsoft Research bouwden ze NLP-modellen die gericht waren op de bewaring van zinsstructuur en -context door \emph{scaffolded learning}. Hiervoor maakten de onderzoekers gebruik van een voorspellingsmethode die de positie van woorden en zinnen in een document beoordeelde. Daarnaast wijst het onderzoek van \textcite{Readable2021} uit dat de Flesch-Kincaid leesbaarheidstest een manier aanbiedt om vereenvoudigde tekstinhoud te beoordelen, zonder de nood van vooraf getrainde modellen. Met de Python-library \textit{textstat}\footnote{https://pypi.org/project/textstat/} kan deze score eenvoudig worden berekend.

\begin{figure}
	\includegraphics[width=\linewidth]{img/Screenshot_302.png}
	\caption{\autocite{Readable2021}}
\end{figure}

%---------- Methodologie ------------------------------------------------------
\section{Methodologie}%
\label{sec:methodologie}

Er wordt een \textit{mixed-methods} onderzoek uitgevoerd om te bepalen of een AI toepassing de tekstinhoud van een wetenschappelijke paper op maat van de noden voor een scholier met dyslexie in het derde graad middelbaar onderwijs kan vereenvoudigen. Het onderzoek houdt zes fases in. 

De eerste fase is het proces van tekstvereenvoudiging beschrijven, waaronder een omschrijving van het begrip en de verschillende soorten van tekstvereenvoudiging met AI. Dit gebeurt via een grondige studie van vakliteratuur en wetenschappelijke teksten. Ook blogs van experten komen hier aan bod. Na het verwerven van de nodige inzichten wordt er een verklarende tekst opgesteld.

De tweede fase bestaat uit het analyseren van wetenschappelijke werken over de bewezen voordelen van tekstvereenvoudiging bij scholieren met dyslexie van het derde graad middelbaar onderwijs. Hiervoor zijn geringe thesissen beschikbaar, die zorgvuldigheid vragen tijdens interpretatie. De resulterende tekst bevat de voordelen samen met hun wetenschappelijke onderbouwing.

De derde fase is opnieuw een beschrijving. Hier worden de valkuilen bij taalverwerking met AI software nagegaan. Deze fase van het onderzoek brengt, aan de hand van een technische uitleg, mogelijke nadelen en tekortkomingen van AI software bij tekstvereenvoudiging aan het licht.

De vierde fase omvat een toelichting over beschikbare AI toepassingen voor tekstvereenvoudiging. Aan de hand van een veldonderzoek op het internet en bij bedrijven wordt er op zoek gegaan naar dergelijke software. Er wordt niet gezocht naar vertaalsoftware of toepassingen die de inhoud van een afbeelding of tekstbestand omzet naar tekstinhoud. Het resultaat van deze fase is een longlist van alle beschikbare AI toepassingen die teksten kunnen vereenvoudigen.

De vijfde fase omschrijft de technische uitwerking van een pipeline voor tekstvereenvoudiging, alsook een shortlist van metrieken om de vereenvoudigde tekstinhoud te evalueren. Er zal een tekstvereenvoudigingspipeline worden ontwikkeld met beschikbare kant-en-klare bibliotheken, \textit{transformers} en algoritmen. Het resultaat van deze fase is een pipeline opgebouwd in de programmeertaal Python. 

De zesde fase bestaat uit een toelichting van de beschikbare evaluatiemetrieken om vereenvoudigde tekst te kunnen beoordelen. Het resultaat is een shortlist van alle evaluatiecriteria waaraan de uitvoertekst van een tekstvereenvoudigingstoepassing moet voldoen.

De zevende en laatste fase omvat een vergelijkende studie van de gevonden AI toepassingen die tekst vereenvoudigen en de pipeline. De tekstinhoud van wetenschappelijke papers, die in een derde graad middelbaar onderwijs worden gebruikt, dienen hier als invoertekst voor de evaluatie. De subjectieve test gebeurt aan de hand van een enquête en een \textit{think-aloudtest}. De objectieve testen gebeuren op basis van de shortlist uit de derde fase en de shortlist van metrieken uit de zesde fase. Ten slotte volgt er een persoonlijk advies over de nodige ontwikkelingen in het vak op vlak van Nederlandstalige tekstvereenvoudiging.

%---------- Verwachte resultaten ----------------------------------------------
\section{Verwacht resultaat, conclusie}
\label{sec:verwachte_resultaten}

Er wordt verwacht dat de software, die nu in het onderwijs wordt ingezet, niet voldoet aan de noden van een scholier met dyslexie in het derde graad middelbaar onderwijs. Er wordt onvoldoende rekening gehouden met het adaptieve aspect. Bestaande internationale AI toepassingen bieden een gelijkwaardige oplossing, al steekt ChatGPT met het GPT-3 model boven de rest uit. Met dit model kan er een krachtige applicatie worden opgebouwd. Het vertalen van de vereenvoudigde tekstinhoud bij een internationale AI toepassing kan afwijken van de oorspronkelijke context.

Er zijn te weinig kant-en-klare algoritmen en modellen beschikbaar om een pipeline voor tekstvereenvoudiging op te zetten, gericht op scholieren met dyslexie in het middelbaar onderwijs. De pipeline is moeilijk af te stemmen op de specifieke noden van deze doelgroep. Er is een behoefte aan aangepaste transformers om bevredigende resultaten te bereiken. Bovendien is er een gebrek aan Nederlandstalige word embeddings die de complexiteit van elk woord kunnen bijhouden en aan kant-en-klare modellen die de inhoud van wetenschappelijke papers kunnen vereenvoudigen. Word embeddings uit een Germaanse taal gebruiken, gevolgd door vertaling naar het Nederlands is wel een acceptabel alternatief.

%%---------- Andere bijlagen --------------------------------------------------
% TODO: Voeg hier eventuele andere bijlagen toe. Bv. als je deze BP voor de
% tweede keer indient, een overzicht van de verbeteringen t.o.v. het origineel.
%\input{...}

%%---------- Backmatter, referentielijst ---------------------------------------

\backmatter{}

\setlength\bibitemsep{2pt} %% Add Some space between the bibliograpy entries
\printbibliography[heading=bibintoc]

\end{document}
