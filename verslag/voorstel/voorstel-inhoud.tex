%---------- Inleiding ---------------------------------------------------------

\section{Introductie}%
\label{sec:introductie}

België is een koploper in het gebruik van artificiële intelligentie (AI) op de werkvloer. Jaarlijks investeert de Vlaamse overheid 32 miljoen in het vakgebied \autocite{Crevits2022}. Soortgelijke technologieën worden amper toegepast in het onderwijs. Er is terughoudendheid door enerzijds ouders van leerlingen \autocite{Martens2021}, anderzijds door de weinige ontwikkelingen in schoolgerelateerde AI-software. Toch zijn er reeds bewijzen dat artificiële intelligentie ook op school nuttig kan zijn. Machinaal leren wordt alsmaar toegankelijker, en toch is er een schaarste aan applicaties die hierop inspelen.

Dit onderzoek beschrijft het gebruik van artificiële intelligentie in de vorm van tekstsimplificatie, als advies voor implementatie in het onderwijs. Specifiek om kinderen met dyslexie te ondersteunen bij het verwerken van leerstof.

Het doel wordt bereikt door eerst het proces van tekstvereenvoudiging te beschrijven. Nadien volgt er een analyse van de bewezen voordelen bij kinderen met dyslexie. Vervolgens worden de valkuilen bij taalverwerking met artificiële intelligentie onderzocht. Uiteindelijk volgt er de ontwikkeling van een \textit{proof-of-concept} in Python. 

%---------- Stand van zaken ---------------------------------------------------

\section{State-of-the-art}%
\label{sec:state-of-the-art}

% Deelvraag 1

De voorbije tien jaar is artificiële intelligentie sterk verder ontwikkeld. De toename in kennis zorgde voor nieuwe toepassingen. Tekstsimplificatie vloeide hier uit voort. Momenteel bestaan er al robuuste applicaties voor tekstsimplificatie. Toch houdt de meerderheid niet genoeg rekening met het menselijk aspect van taalverwerking. Binnen het kader van tekstsimplificatie is er bestaande documentatie beschikbaar waar onderzoekers het voordeel van toegankelijkheid aanhalen, maar deze toepassingen ontbreken de extra noden die mensen met een leeraandoening vereisen.

Het algemene doel van tekstsimplificatie is om ingewikkelde bronnen toegankelijker te maken. Het zorgt voor verkorte teksten zonder de oorspronkelijke context te verliezen. Tekstsimplificatie gebeurt doorgaans op één van drie manieren. Er is conceptuele simplificatie waarbij documenten naar een compacter formaat worden getransformeerd. Daarnaast is er uitgebreide modificatie die kernwoorden aanduidt door gebruik van redundantie. Als laatste is er samenvatting die documenten verandert in kortere teksten met alleen de topische zinnen. Met deze concepten zijn ontwikkelaars in staat om ingewikkelde woorden te vervangen door eenvoudigere synoniemen of zinnen te verkorten zodat ze sneller leesbaar zijn \autocite{Siddharthan2014}.

Tekstsimplificatie behoort tot de zijtak van natuurlijke taalverwerking (NLP) in artificiële intelligentie. NLP omvat methodes om, door machinaal leren, menselijke teksten om te zetten in tekst voor machines. Documenten vereenvoudigen met NLP kan op twee manieren: extract of abstract. Bij extractieve simplificatie worden zinnen gelezen zoals ze zijn neergeschreven. Vervolgens bewaart een document de belangrijkste taalelementen om de tekst te kunnen hervormen. Deze vorm van tekstsimplificatie komt het meeste voor \autocite{Sciforce2020}. Daarnaast is er abstracte simplificatie die de oorspronkelijke context van de zin bewaart en daarmee een nieuwe zin opbouwt. Deze vorm heeft potentieel dankzij de menselijke interpretatie, maar zit nog in de kinderschoenen \autocite{Chowdhary2020}.

% Deelvraag 2

Voor kinderen met dyslexie bestaan digitale hulpmiddelen die voor een betere visuele presentatie zorgen van teksten. Het gaat over speciale lettertypes, spreiding tussen woorden en het gebruik van inzoomen op aparte zinnen. Weinig aandacht wordt besteed aan het veranderen van de tekst zelf, want dit kost tijd. Tekstsimplificatie door artificiële intelligentie kan een revolutionaire oplossing bieden. 

Het onderzoek van Franse wetenschappers \textcite{Gala2016} illustreert dat manuele tekstsimplificatie schoolteksten toegankelijker maakt voor kinderen met dyslexie. Dit deden ze door simpelere synoniemen en zinsstructuren te gebruiken. Verwijswoorden werden vermeden en woorden kort gehouden. De resultaten waren veelbelovend. Het leestempo lag hoger en de kinderen maakten minder leesfouten. Ook bleek er geen verlies van begrip in de tekst bij geteste kinderen. Resultaten van de studie werden gebundeld voor de mogelijke ontwikkeling van een AI-hulpmiddel.

De Universiteit van Kopenhagen is met bovenstaande idee aan de slag gegaan. Onderzoekers \textcite{Bingel2018} hebben gratis software ontwikkeld, genaamd Lexi, om tekstsimplificatie voor mensen met dyslexie te automatiseren. De software bestudeert met welke woorden de gebruiker moeite heeft, en vervangt die door simpelere alternatieven. Hoe meer de software gebruikt wordt, hoe beter hij op maat van de gebruiker zal werken. Dit is de eerste en momenteel enige software van zijn soort. Voorheen bestond alleen generieke AI-software voor tekstsimplificatie. Lexi is beschikbaar als een browserextensie en tot nu toe enkel in het Deens. 

% Deelvraag 3

NLP is de laatste decennia volop in ontwikkeling, maar ontwikkelaars botsen nog op uitdagingen. Het gaat om zowel interpretatie- als dataproblemen bij AI-machines. Allereerst is het voor een machine moeilijk om de context van homoniemen te achterhalen. Bijvoorbeeld bij het woord ‘bank’ is het niet duidelijk voor de machine of het gaat over de geldinstelling of het meubel. Daarnaast zijn synoniemen geen probleem voor tekstverwerking \autocite{Roldos2020}.

Het merendeel van NLP-toepassingen maakt gebruik van Engelstalige invoer. Niet-Engelstalige toepassingen zijn zeldzaam. De opkomst van AI-technologieën die twee datasets gebruiken, biedt een oplossing voor dit probleem. De software vertaalt eerst de oorspronkelijke tekst naar de gewenste taal, voordat de tekst wordt herwerkt \autocite{Sciforce2020}.

Om tekstsimplificatiemethoden te beoordelen, is er een tactvolle aanpak nodig. De studie van \textcite{Swayamdipta2019} haalt aan dat er extra nood is aan NLP-modellen waarbij de tekst zijn oorspronkelijke betekenis behoudt. Samen met Microsoft Research bouwden ze NLP-modellen die gericht waren op de bewaring van zinsstructuur en -context door \emph{scaffolded learning}. Hiervoor maakten de onderzoekers gebruik van een voorspellingsmethode die de positie van woorden en zinnen in een document beoordeelde.

% Deelvraag 4 (herwerking)
Vlaanderen heeft weinig zicht op de geïmplementeerde AI-software in scholen. Dit werd geconstateerd door \autocite{Martens2021}, een samenwerking tussen de Vlaamse universiteiten en overheid voor artificiële intelligentie. Vergeleken met andere Europese landen, maakt België het minst gebruik van leerling-georiënteerde hulpmiddelen. Degenen die wel gebruikt worden, zijn voornamelijk online leerplatformen voor zelfstandig werken. Ook maakt België amper gebruik van beschikbare software die de leermethoden en -noden van leerlingen evalueert \autocite{Martens2021a}. 



%---------- Methodologie ------------------------------------------------------
\section{Methodologie}%
\label{sec:methodologie}

Het onderzoek houdt vijf fases in. De eerste fase is het proces van tekstvereenvoudiging beschrijven. Dit gebeurt via een grondige studie van vakliteratuur en wetenschappelijke teksten. Ook blogs van experten komen hier aan bod. Na het verwerven van de nodige inzichten wordt er een verklarende tekst opgesteld.

De tweede fase bestaat uit het analyseren van wetenschappelijke werken over de bewezen voordelen van tekstsimplificatie bij kinderen met dyslexie. Hiervoor zijn geringe thesissen beschikbaar, die zorgvuldigheid vragen tijdens interpretatie. De resulterende tekst bevat de voordelen samen met hun wetenschappelijke onderbouwing.

De derde fase is opnieuw een beschrijving. Hier worden de valkuilen bij taalverwerking met AI-software nagegaan. Deze fase van het onderzoek brengt mogelijke nadelen en tekortkomingen van AI-software bij tekstsimplificatie aan het licht. Dit gebeurt aan de hand van een technische uitleg.

De vierde fase omvat een toelichting en advies over de beschikbare Nederlandstalige AI-tools voor tekstsimplificatie. Aan de hand van een kort veldonderzoek op het internet en bij bedrijven wordt er op zoek gegaan naar dergelijke software. Het opzoekingswerk leidt uiteindelijk tot testen van de applicaties. Ten slotte volgt er een persoonlijk advies over de nodige ontwikkelingen in het vak op vlak van (Nederlandstalige) tekstsimplificatie.

Als vijfde en laatste fase is de ontwikkeling van een \textit{proof-of-concept} gepland. Met gebruik van de Python programmeertaal wordt een korte applicatie ontwikkeld. De applicatie is kleinschalig, maar specifiek gericht op de noden van een kind met dyslexie. 


%---------- Verwachte resultaten ----------------------------------------------
\section{Verwacht resultaat, conclusie}%
\label{sec:verwachte_resultaten}

Er wordt verwacht dat er nog geen geschikte software beschikbaar is voor toepassing in onderwijsinstellingen. Dit is omdat de technische aspecten verdere ontwikkeling vereisen voor algemene tekstsimplificatie. Ook houdt bestaande software onvoldoende rekening met de unieke uitdagingen omtrent leerstoornissen. Daarnaast is Engels de gebruikelijke voertaal van AI-software, wat niet gepast is voor implementatie in Vlaamse scholen. Nederlandstalige applicaties zullen te vinden zijn, maar voor generiek gebruik zonder ondersteunende functie en in schaarse aantallen. Anderzijds wordt er verwacht dat de software ontoegankelijk is. Hiermee wordt bedoeld dat ze enkel te verkrijgen is bij gespecialiseerde IT-bedrijven achter een hoge prijs. \emph{Open-source} versies zullen moeilijk of niet te vinden zijn.